%%%%%%%%%%%%%%%%%%%%%%%%%%%%%%%%%%%%%%%%%%%%%%%%%%%%%%%%%%%
%% Master's Thesis of Jukka Pajarinen
%%%%%%%%%%%%%%%%%%%%%%%%%%%%%%%%%%%%%%%%%%%%%%%%%%%%%%%%%%%
\documentclass{config/tauthesis}
\usepackage{pgfplots}
\pgfplotsset{compat=1.15}
\usepackage{subcaption}
\usepackage{amsmath, amssymb, amsthm, bm}
\usepackage{float}
\usepackage{soul}
\numberwithin{equation}{section}
\newcommand{\verbcommand}[1]{\texttt{\textbackslash #1}}
\newtheorem{lause}{Lause}[chapter]
\newtheorem{theorem}[lause]{Theorem}
\newtheorem{apulause}[lause]{Apulause}
\newtheorem{lemma}[lause]{Lemma}
\theoremstyle{definition}
\newtheorem{maaritelma}{Määritelmä}[chapter]
\newtheorem{definition}[maaritelma]{Definition}
\loadglsentries[main]{tex/04_sanasto.tex}
\makeglossaries
\addbibresource{tex/00_referenssit.bib}

%%%%%%%%%%%%%%%%%%%%%%%%%%%%%%%%%%%%%%%%%%%%%%%%%%%%%%%%%%%
%% Front matter
%%%%%%%%%%%%%%%%%%%%%%%%%%%%%%%%%%%%%%%%%%%%%%%%%%%%%%%%%%%
\begin{document}
\frontmatter
\title
  {Web-käyttöliittymän hyväksymistestauksen priorisointi painotetun verkon avulla}
  {Web User Interface Acceptance Testing Prioritization with a Weighted Graph}
\subtitle{}{}
\author{Jukka Pajarinen}
\finishdate{2019}{12}{31}
\thesistype{Diplomityö}{Master's Thesis}
\facultyname
  {Informaatioteknologian ja viestinnän tiedekunta}
  {Faculty of Information Technology and Communication Sciences}
\programmename{Tietotekniikan DI-ohjelma}{Degree Programme in Information Technology}
\keywords
  {hyväksymistestaus, painotettu verkko, priorisointi, jatkuva integrointi, web-sovellukset, testiautomaatio}
  {acceptance testing, weighted graph, prioritization, continuous integration, web applications, test automation}
\maketitle
\abstract{tex/01_tiivistelma.tex}
\otherabstract{tex/02_abstract.tex}
\preface{tex/03_alkusanat.tex}{Tampereella}
\tableofcontents
\listoffigures
\listoftables
\glossary

%%%%%%%%%%%%%%%%%%%%%%%%%%%%%%%%%%%%%%%%%%%%%%%%%%%%%%%%%%%
%% Main matter
%%%%%%%%%%%%%%%%%%%%%%%%%%%%%%%%%%%%%%%%%%%%%%%%%%%%%%%%%%%
\mainmatter
\chapter{Johdanto} \label{ch:05_johdanto}
  Tässä luvussa ...

\begin{figure}[ht!]
  \centering
  \includegraphics[width=0.5\textwidth]{assets/good-example.png}
  \caption{<Lisää kuvateksti tähän.>}
  \label{fig:kuvaesimerkki}
\end{figure}

\begin{table}[ht!]
  \centering
  \caption{<Lisää taulukkoteksti tähän.>}
  \label{tab:taulukkoesimerkki}
  \begin{tabular}{p{0.2\linewidth} | p{0.3\linewidth} | p{0.3\linewidth}}
    \hline
    & \textbf{Otsikko 1} & \textbf{Otsikko 2} \\
    \hline
    \textbf{Otsikko 3} & Teksti 1 & Teksti 2\\
    \hline
    \textbf{Otsikko 4} & Teksti 3 & Teksti 4\\
    \hline
  \end{tabular}
\end{table}

\parencite{nawar_multi-heuristic_2014}

\parencite{zhang_test_2007}
\chapter{Tutkimusasetelma} \label{ch:06_tutkimusasetelma}
  Tässä luvussa esitetään diplomityön taustaa, tutkimuskysymykset, tutkimusmenetelmä, tutkimuksen rajaus sekä tavoitteet.
Tutkimuskysymykset liittyvät vahvasti yhteiseen priorisoinnin teemaan, johon tässä työssä erityisesti paneudutaan.
Lisäksi työn lopussa on myös toteutuksellinen osuus, joka on tehty diplomityön asiakasyrityksen tarpeita varten.
Toteutuksellisessa osuudessa on paljon muutakin sisältöä, joka on varsinaisen priorisointiteeman ulkopuolella, mutta pysyy kuitenkin työn kokonaiskontekstissa.

\section{Tausta}

Diplomityö tehtiin WordDive nimiselle yritykselle. WordDive on Tampereella toimiva, suomalainen kieltenoppimiseen keskittyvä yritys. WordDivellä oli kirjoitushetkellä kieltenoppimissovellus sekä mobiilialustalle että web-alustalle. Tämän diplomityön sisältö koskettaa web-alustalla toimivaa sovellusta. Hyväksymistestauksen osalta mobiilisovellukselle oli yrityksessä toteutettu testiautomaatio, mutta web-alustalle sitä ei vielä oltu tehty.

% Lisää tekstiä tähän

\section{Tutkimuskysymykset} \label{tutkimuskysymykset}

Tutkimuksen tarkoituksena on pohjimmiltaan tarkoitus löytää ja kehittää toistettavissa oleva menetelmä hyväksymistestauksen testitapauksien priorisoimiseen.
Testitapauksien laatimisen yleisenä ongelmanakohtana on erityisesti niiden priorisointi, joka usein johtaa liian suppean tai ylikattavan testiautomaation rakentamiseen.
Tutkimuskysymykset on laadittu siten, että niihin vastaaminen antaa ratkaisun edellä mainittuun testiautomaation ongelmaan.

Työlle asetettiin seuraavat tutkimuskysymykset:
\begin{itemize}
  \item T1: \emph{Miten painotettua verkkoa voidaan käyttää testitapauksien priorisoimiseen?}
  \item T2: \emph{Mitkä muuttujat vaikuttavat web-käyttöliittymän hyväksymistestauksen testitapauksien priorisointiin?}
  \item T3: \emph{Kuinka prioriteetein painotetusta verkosta valitaan toteutettavat testitapaukset?}
  \item (T4: \emph{Miten painotetun verkon avulla tehty priorisointi liitetään yhteen jatkuvan integraation ja testiautomaation kanssa?)}
\end{itemize}

\section{Tutkimusmenetelmä}

Tutkimuskysymyksiin vastaamiseksi työn tutkimusmenetelmäksi valittiin diplomitöissä yleisesti käytetty design science menetelmä.
Tarkoituksena oli muodostaa uudenlainen toistettavissa oleva menetelmä tutkimuksen kohteena olevan ongelman ratkaisemiseksi.
Tutkimusidean hahmottelemisen ja ratkaisua kaipaavan ongelman identifoinnin jälkeen, valittua tutkimusenetelmää käyttäen ensin määriteltiin tutkimuskysymykset \ref{tutkimuskysymykset}.
Seuraavaksi kartoitettiin ratkaisuvaihtoehto tutkimuskysymyksiin ja esitetään perustelut siihen päätymiseen.
Asiakasyrityksen ohjelmistotuotetta silmälläpitäen toteutettiin kokonaisratkaisu hyödyntäen kehitettyä ratkaisumenetelmää.
Lopuksi vielä evaluoitiin ratkaisun toimivuus ja esitetään yhteenveto tutkimuksesta.

% Tutkimusaiheen suunnittelemisen alkuvaiheessa huomattiin, että olemassa olevat käsitteelliset mallit ja teoria eivät olleet kovin vakiintunutta ja jäsenneltyä, joka antoi lisää painoarvoa työn tekemiselle.
% Tutkittavaan asiaan liittyvää aineistoa muun muassa priorisoinnin ja painotettujen verkkojen osalta on saatavilla runsaasti, joten aineiston valitseminen perustui harkintaan.
% Aineston hallintaan käytettiin tietokoneohjelmistoa, jossa aineisto kategorisoitiin eri loogisiin kokonaisuuksiin muun muassa priorisoinnin ja painotetun verkon osalta.
% Kerätyn aineiston avulla pyrittiin luomaan mahdollisimman vahva teoreettinen pohja tutkimuskysymyksiin vastaamiseksi mahdollisimman kattavasti.

\section{Tutkimuksen rajaus}

% Teksti tähän

\section{Tavoitteet}

% Teksti tähän

\chapter{Testiautomaatio} \label{ch:07_testiautomaatio}
  Tässä luvussa pyritään esittämään perusteet ja tarvittavat tiedot testiautomaatiosta.
Perusteiden ymmärtämistä tarvitaan työn myöhemmässä vaiheessa, jossa esitetään testitapauksien priorisointi painotetun verkon avulla.

\section{Testiautomaation tarkoitus}

<Lisää teksti tähän>

\section{Testauksen lähestymistavat}

<Lisää teksti tähän>

\section{Testiautomaatio prosessina}

<Lisää teksti tähän>

\section{Testitapauksien määrittäminen}

<Lisää teksti tähän>

\section{Testitapauksien priorisointi}

<Lisää teksti tähän>

\section{Web-käyttöliittymien erityispiirteet}

<Lisää teksti tähän>

\section{Hyväksymistestaus}

<Lisää teksti tähän>

\chapter{Hyväksymistestaus} \label{ch:08_hyvaksymistestaus}
  Tässä luvussa esitetään perusteet ja tarvittavat tiedot hyväksymistestauksesta..

\section{Hyväksymistestauksen tarkoitus} \label{ch:08_hyvaksymistestauksen_tarkoitus}

  Hyväksymistestauksen tarkoituksena on varmistaa toteutettavan ohjelmiston vaatimusten toimivuus erityisesti käytännön tilanteissa siten, että voidaan varmistaa vastaako ohjelmisto loppukäyttäjän tarpeita.
  Hyväksymistestaus antaa vastauksen siihen, toimiiko toteutettu järjestelmä loppukäyttäjän tarpeiden mukaisesti ja loppukäyttäjän näkökulmasta oikein.
  Hyväksymistestauksen sanotaan olevan muodollista testaamista, jossa käyttäjän tarpeet, vaatimukset ja liiketoimintaprosessit otetaan huomioon selvittäessä täyttääkö järjestelmä hyväksymisen kriteerit ja sallii käyttäjän, asikkaiden tai muun autorisoidun tahon päättää hyväksytäänkö järjestelmä \parencite{istqb_glossary_nodate}.
  Ohjelmistotestauksen tekniikoiden näkökulmasta hyväksymistestaus on mustalaatikkotestausta, eli sitä testataan tietämättä sen teknisestä toteutuksesta.
  Hyväksymistestauksen painoarvo on asiakaperusteisessa vaatimusmäärittelyssä ja loppukäyttäjän tarpeiden kartoittamisessa.
  Testiautomaation osalta hyväksymistestausta varten voidaan rakentaa testitapaukset, joiden avulla voidaan keskittyä varmistamaan loppukäyttäjille tarpeellisten toimintojen toteutuminen testitapauksien suorittamisen jälkeen.
  Hyväksymistestauksen osalta testitapauksia voidaan toteuttaa niin sanotulla päästä päähän -periaatteella, jossa testattavaa järjestlemää testataan siten kuin loppukäyttäjä sitä käyttää.
  Hyväksymistestauksessa ei anneta suurta painoarvoa kosmeettisille tai kirjoitusvirheille, vaan pyritään selvittämään loppukäyttäjille oleellisten ja tarpeellisten toimintojen toteutuminen.

  Hyväksymistestaus on aiemmin esitetyistä testauksen tasoista \ref{ch:07_testauksen_tasot} viimeinen ja sen suorittamisen jälkeen saadaan tieto siitä onko järjestlemä toteutuksen osalta sellaisenaan valmis julkaistavaksi.
  Perinteisesti hyväksymistestauksen lähtökohtia ovat selvät hyväksymisvaatimukset sekä julkaisukelpoinen toteutus joka voi sisältää vain kosmeettisia virheitä.
  Hyväksymisvaatimukset voivat olla esimerkiksi liiketoiminnallisia käyttötapauksia, prosessivirtauskaavioita sekä ohjelmiston vaatimusmäärittely.
  Testiautomaatiota varten käytettävästä testialustasta riippuen hyväksymistestauksen käyttötapaukset voidaan muodostaa joko osittain tai suoraan testitapauksiksi.
  Hyväksymistestaukseen usein osallistuu ohjelmistokehittäjien lisäksi myös muut sidosryhmät ja loppukäyttäjät.
  Keskeistä on, että loppukäyttäjiltä hankitaan tieto tarvittavista ja toteutettavista ominaisuuksista, kun taas muut sidosryhmät kuten esimerkiksi johtoryhmä voivat tehdä liiketoiminnallisia päätöksiä hyväksymistestauksen onnistumisen osalta ja esimerkiksi peruuttaa julkaisun.
  Hyväksymistestaus antaa mahdollisuuden korjata usein liiketoiminnalisestakin näkökulmasta merkittävät toiminalliset virheet ennen järjestelmän julkaisua loppukäyttäjille.

  Kehittäjien käsitys järjestelmän toiminnallisuudesta ja sen vaatimuksista voi olla usein hyvinkin erilainen kuin loppukäyttäjien.
  Hyväksymistestauksen avulla voidaan tätä lievittää tätä ongelmaa, ja saattaa ohjelmistokehittäjät loppukäyttäjien kanssa vaatimusmäärittelyn suhteen samalle sivulle.
  Testiautomaation avulla toteutettavalla toistuvalla hyväksymistestauksella varmistetaan, että järjestelmä toteuttaa loppukäyttäjän tarpeet vielä järjestlemään tehtyjen muutoksien jälkeenkin.
  Hyväksysmistestauksen testitapaukset tarkoituksenmukaisesti heijastavat suoraan loppykäyttäjien tarpeita, joka on iso etu sillä sen avulla ohjelmistokehittäjät ja muut sidosryhmät voivat tehokkaasti varmistaa järjestlemän valmiuden ja tilan.
  Hyväksymistestauksella siis saadaan katsaus ohjelmiston valmiudesta sen vaatimuksiin ja loppukäyttäjien toiminnallisiin tarpeisiin nähden.

\section{Hyväksymistestausvetoinen kehitys} \label{ch:08_hyvaksymistestausvetoinen_kehitys}

  Hyväksymistestausvetoisen kehityksen sanotaan olevan yhteistyöllinen lähestymistapa kehitykseen, jossa tiimi ja asiakkaat käyttävät asiakkaiden oman ympäristön kieltä ymmärtääkseen heidän vaatimukset, jotka muodostavat pohjan komponentin tai järjestelmän testaamiseen \parencite{istqb_glossary_nodate}.

  Hyväksymistestausvetoisen kehityksen (englanniksi: ATDD, acceptance test driven development) tarkoituksena, kuten testausvetoisessakin kehityksessä on toteuttaa ohjelmistotuotannollinen prosessi testaaminen edellä.

  Tämä tarkoittaa käytännössä, sitä että ohjelmistokehittäjät laativat ohjelmiston vaatimusten ja suunnitelman mukaisia iteratiivisesti suoritettavia testitapauksia, ennen niitä käyttävän varsinaisen ohjelmakoodin toteuttamista.

  Hyväksymistestausvetoisessa kehityksessä luodaan ennen toteutusta tarvittavat ohjelmiston asiakasvaatimuksia palvelevat hyväksymistestit, joiden ohjelmiston on tarkoitus läpäistä.

  Tarvittavat ohjelmiston hyväksymistestit suoritetaan iteratiivisesti ohjelmistokehitysprosessin aikana, ja se tarkoittaa käytännössä jatkuvan integraation \ref{ch:07_jatkuva_integrointi} ottamista käyttöön ohjelmistokehityksessä.

  Hyväksysmistestausvetoinen kehitys on erittäin hyödyllinen ohjelmistokehityksessä käytetty menetelmä, sillä kehitysvaiheessa on aina tarkasti tiedossa vastaako ohjelmiston tila asiakasvaatimuksia ja kuinka hyvin se niiden täyttämisessä onnistuu.


  % TODO: Lisää kuva tähän ATDD:stä (kuten ch:07_testausvetoinen_kehitys)

  \begin{itemize}
    \item ketterä ohjelmistokehitysmenetelmä
    \item kuten tdd, mutta ennen ohjelmistokehityksen aloitusta asiakasvaatimukset kartoitetaan ja hyväksyttävyys määritetään
    \item ohjelmistokehitystä ohjaa vaatimukset ja loppukäyttäjien tarpeiden toteutuminen
    \item hyväksymistestit kirjoitetaan tdd:n mukaisesti ensin, ja ohjelmistokehitys noudattaa iteratiivisesti tdd:tä, vaikkakin hyväksymistestaus itsessään on perinteisesti vaatinut lähes valmista järjestelmää.
    \item hyväksymistestit pilkotaan pieniin hallittaviin kokonaisuuksiin, jolloin voidaan iteratiivisesti toteuttaa valmiiksi tietyn testitapauksen mukainen ominaisuus joka vastaa jotakin loppukäyttäjän tarvetta.
    \item testitapaus voi olla esimerkiksi käyttäjän tietojen muuttuminen varmistaminen, kuten tason läpäiseminen pelisovelluksessa, joka muuttaa käyttäjän edistystä.
    \item esimerkki käyttötapaus: as a user, I want to be able to unlock premium features by in-app purchase
    \item Perus workflow: 1) kirjoita vaatimus testitapauksen muotoon, 2) toteuta testitapaus ja ominaisuus, 3) refaktoroi ja iteroi takaisin 1 vaiheeseen tarvittaessa, 4) aja testitapaus ja validoi ominaisuuden toimivuus
    \item menetelmän tarkoituksena on onnistua vastaamaan loppukäyttäjän tarpeisiin tehokkaasti ja hyvin ottamalla tarpeet huomioon jo ennen toteutuksen aloittamista.
    \item menetelmän avulla myös luodaan ymmärrystä ohjelmistotuotteen valmiuden määritelmästä kun eri sidosryhmät pääsevät samalle aaltopituudelle.
    \item jatkuva testaaminen haluttujen ominaisuuksien toteutumisen validoimiselle menetelmän jokaisen iteraation koontiversiossa.
    \item hyväksymistestausvetoinen kehitys lainaa samoja perusperiaatteita testausvetoisesta kehityksestä, mutta keskittyy käyttötapauksien muodossa validoitavien haluttujen ominaisuuksien toteutumista.
  \end{itemize}

\section{Robot Framework} \label{ch:08_robot_framework}

  Robot framework on geneerinen avoimen lähdekoodin testausalusta hyväksymistestaukseen, hyväksymistestausvetoiseen kehitykseen ja robotisten prosessien automaatioon.

  \begin{figure}[H]
    \centering
    \includegraphics[width=0.4\textwidth]{assets/robot-architecture.png}
    \caption{Robot framework alustan arkkitehtuuri}
    \label{fig:robot-architecture}
  \end{figure}

  \begin{itemize}
    \item Helposti ymmärrettävä, luettava ja selkeä avainsanaperustainen syntaksi
    \item Etuna helposti lähestyttävyys. Helppo asentaa, ymmärtää ja ottaa käyttöön.
    \item Testikehystä voi helppouden ja avainsanaperustaisuuden vuoksi käyttää muutkin kuin sovelluskehittäjät.
    \item Tuki ulkoisille kirjastoille ja useita käyttövalmiita ulkoisia kirjastoja
    \item Tukee muuttujia testitapauksissa, joilla voi lisätä hieman kompleksisuutta testitapauksiin
    \item Tukee dataperustaisia testitapauksia, joille annetaan eri syötteitä sisältävää testidataa
    \item Testitapauksia voidaan ryhmitellä tageillä.
    \item Kattavat ja selkeät testiraportit ajetuille testitapauksille.
    \item Heikkoutena tuen puuttuminen ohjelmointikieliperustaisissa testikehyksissä löytyville kontrollirakenteille, joita esiintyy esimerkiksi yksikkötestaukseen tarkoitetuissa testikehyksissä.
    \item käyttötapaus: 1) tilanne, 2) motivaatio, 3) haluttu lopputulos (esim. Kun tämä, niin haluan tätä, jotta saavutan tämän)
  \end{itemize}

\section{Testitapauksien määrittäminen} \label{ch:08_testitapauksien_maarittaminen}

  \begin{itemize}
    \item Testitapaus on testiautomaation näkökulmasta, määritelty toimenpiteiden, ehtojen ja muuttujien joukko, joka suorittamalla voidaan verifioida ominaisuus tai toiminnallisuus ohjelmistosta.
    \item Testisuite tai testikokoelma on samaan kontekstiin kuuluvista testitapauksista muodostettu joukko.
    \item Testitapaukset kirjoitetaan hyväksymistestauksen mukaisesti käyttötapauksien muodossa.
    \item Testiformaatti: 1) Oletetaan tämä ja tätä (setup) => 2) Kun tämä tapahtuu (trigger) => 3) Niin tämä seuraa (verification) (todo: tee tästä kuva)
    \item Yleisiä tavoitteita: yksinkertaisuus, läpinäkyvyys, käyttäjätietoisuus, epätoistuvuus, olettamattomuus, kattavuus, tunnistettavuus, jälkensä puhdistava, toistettava, syvyyttömyys, atomisuus.
    \item Taulukon/listan muodossa esimerkkejä käyttöliittymien testitapauksista.
    \item Robot framework: avainsana, käyttäytyminen tai data-pohjainen kirjoitustyyli testitapauksille
    % \item <todo: lisää kuva/koodi esimerkkitestitapauksessa robot frameworkillä>
    \item https://github.com/robotframework/HowToWriteGoodTestCases/blob/master/HowToWriteGoodTestCases.rst
  \end{itemize}

\section{Web-käyttöliittymien erityispiirteet} \label{ch:08_webkayttoliittymien_erityispiirteet}

  Web-käyttöliittymillä on myös omia erityispiirteitä, jotka vaikuttavat testitapauksien laatimiseen.

  \begin{itemize}
    \item Käyttöliittymä ja DOM
    \item Hosting
    \item Näyttöresoluutiot
    \item Navigointi
    \item Syötteet
    \item Syntaksi
    \item Selainasetukset
    \item Moniselaimellinen testaus
    \item Päätteetön testaus
    \item Selenium
  \end{itemize}

\section{Priorisointiongelma} \label{ch:08_priorisointiongelma}

  Testitapauksien priorisointi on kustannussyistä tai resurssien optimoinnin kannalta erittäin tärkeää.
  Ohjelmistotestauksessa on hyvä tiedostaa, että ohjelmistotuotetta ei usein voida testata täydellisesti, joka nostaa esiin tarpeen tärkeimpien testitapauksien löytämisestä.
  Testitapauksia voidaan priorisoida monella tavalla, joihin tämä diplomityö tuo yhden uudenlaisen painottua verkkoa hyödyntävän lähestymistavan.

  \begin{itemize}
    \item Painotetun verkon hyödyntäminen
    \item Muut priorisointitavat
  \end{itemize}
\chapter{Priorisointi painotetun verkon avulla} \label{ch:09_priorisointi_painotetun_verkon_avulla}
  Tässä luvussa käsitellään ensin työhön keskeisesti kuuluvan verkkoteorian perusteita ja käydään huolellisesti läpi niistä tässä työssä käytettävät osat.
Työssä sovelletaan erityisesti verkkoteorian painotettua verkkoa sekä verkkoteoriassa esiintyvän lyhimmän polun ongelmaan kehitettyä Dijkstran algoritmia.
Verkkoteoria itsessään on osa diskeettiä matematiikkaa.

Verkkoteorian jälkeen tässä luvussa esitetään vaiheittain työn tuloksena kehitetty priorisointimenetelmä.
Priorisointia varten esitetään harkintaa käyttäen valitut priorisointiin vaikuttavat muuttujat, niitä käyttävät painofunktiot, verkon rakentaminen ja karsiminen sekä verkon ja testitapauksien yhteys.
Lisäksi käydään läpi miten menetelmää käyttäen tuotetun painotetun verkon sisältämää informaatiota voidaan hyödyntää prioriteeteiltaan tärkeimmän polun löytämiseen \ref{ch:10_dijkstran_algoritmin_hyodyntaminen}.

\section{Matemaattisten verkkojen tarkoitus} \label{ch:09_matemaattisten_verkkojen_tarkoitus}

  Matemaattisten verkkojen tarkoituksena on mallintaan parittaisia riippuvuuksia verkkomaisessa objektijoukossa.
  Verkkoteoriassa peruskäsitteitä ovat itse \emph{verkko} eli \emph{graafi}, joka muodostuu \emph{solmuista} ja niiden välisiä riippuvuuksia esittävistä \emph{kaarista} tai \emph{nuolista}.
  Verkkoteorialla on lukuisia käytännön sovellutuksia. Verkkoteoriaa sovelletaan muun muassa tietokonetieteissä, kielitieteissä, fysiikan ja kemian sovellutuksissa, sosiaalisissa tieteissä ja biologiassa.
  Alun perin verkkoteoria katsotaan syntyneen 1700-luvulla esiintyneestä niin sanotusta Königsbergin siltaongelmasta, johon Leonhard Euler esitti todistuksensa.

  Matemaattisten verkkojen käyttöön päädyttiin tässä työssä siksi, että niiden avulla on hyväksymistestauksen kohteena oleva käyttöliittymä mahdollistaa mallintaa verkoksi.
  Käyttöliittymän verkkomuotoiseen esitykseen voidaan vielä lisätä painot, jotka tässä tapauksessa kuvaavat prioriteetteja, mahdollistaen testikokoelmien priorisoinnin.

\section{Perusmerkinnät ja käsitteet} \label{ch:09_perusmerkinnat_ja_kasitteet}

  Verkkoteoriassa käytetään muun muassa seuraavia perusmerkintöjä ja käsitteitä:

  \begin{itemize}
    \item \(V := \{v_1, v_2, v_3\}\) Solmujoukko joka sisältää \emph{solmut} \(v_1\), \(v_2\) ja \(v_3\).
    \item \(E := \{e_1, e_2, e_3\}\) Kaarijoukko joka sisältää \emph{kaaret} \(e_1\), \(e_2\) ja \(e_3\).
    \item \(\phi(e_1) := \langle v_1, v_2 \rangle\) Kaariparin \(v_1\) ja \(v_2\) yhdistävän \emph{kaaren} \(e_1\) kuvaaja.
    \item \(d_G(x)\) Solmun asteluku, eli solmuun liittyvien \emph{kaarten} määrä.
    \item \(G_2 \subset G_1\) Aliverkko, eli \emph{verkko} \(G_2\) joka koostuu osasta \emph{verkon} \(G_1\) \emph{solmuja} ja \emph{kaaria}.
    \item \(v_1 \neq v_2, v_1 \rightarrow v_2\) Verkon yhtenäisyys, eli jokaiselle solmuparille \(v_1 \neq v_2\) on olemassa niitä yhdistävä \emph{kaari}.
    \item \(P = \{v_0, v_1, ..., v_n\}, v_0 \rightarrow v_n\) Polku, eli \emph{suunnattu solmujono} jota pitkin voidaan kulkea \emph{solmusta} \(v_0\) \emph{solmuun} \(v_n\).
    \item \(P = \{v_0, v_1, ..., v_n| e \in E_P, e \notin \{E_P \setminus \{e\} \}\}\) Sykli, eli \emph{polku}, jonka aloitus \(v_0\) ja lopetussolmu \(v_n\) on sama, mutta polun jokaista kaarta \(e\) kuljetaan vain kerran.
    \item \(d_G(v_1) = 0\) Eristetty solmu, eli \emph{solmu} jonka \emph{asteluku} on nolla.
    \item \(v_1 \rightarrow v_2, d_G(v_1) = 1 \lor d_G(v_2) = 1\) Silta, eli \emph{kaari} johon yhdistyvän \emph{solmun asteluku} on yksi ja jonka poistaminen epäyhteinäistää \emph{verkon}.
    \item \(v_x \rightarrow v_x\) Silmukka, eli \emph{kaari} jonka \emph{aloitussolmut} ja \emph{lopetussolmu} ovat sama \emph{solmu}.
    \item \(\alpha := V(G), E(G) \rightarrow \mathbb{N}\) Painofunktion yleinen kuvaus verkossa \(G\), solmuille \(V\) ja kaarille \(E\).
  \end{itemize}

\section{Priorisointiin vaikuttavat muuttujat} \label{ch:10_priorisointiin_vaikuttavat_muuttujat}

  Näkymä- ja siirtymäperustaiseen priorisointiin vaikuttavat monet eri asiat, joista osa kasvattaa prioriteettia ja osa laskee sitä.
  Prioriteettia kasvattava muuttuja on esimerkiksi liiketoiminnallinen arvo ja laskeva muuttuja on esimerkiksi projektin muutosherkkyys.
  Muuttujat ovat kuitenkin hyvin kontekstiriippuvaisia, joten yleispätevää ja kaikkiin tilanteisiin soveltuvaa listaa muuttujista on hankala antaa.
  Kontekstiriippuvaisuuden takia muuttujiin ja myöhemmin esitettäviin painofunktioihin on varattu paikka omille lisämuuttujille.

  Tässä diplomityössä esiteltävää priorisointimenetelmää varten jokainen priorisointiin vaikuttava muuttuja arvioidaan asteikolla 1-10, paria poikkeusta lukuun ottamatta.
  Numeerisella asteikolla on tarkoitus antaa korkea numero, jos muuttuja on prioriteetiltaan tärkeä kyseisen näkymän, eli verkon solmun kohdalla.
  Jos jokin muuttuja ei ole kelpoinen siinä kontekstissa, jossa menetelmää yritetään hyödyntää, tulee muuttujan arvo asettaa nollaksi, jolloin se sivuutetaan painofunktiossa \ref{ch:10_painofunktiot_priorisointiin}.

  Poikkeukselliset muuttujat ovat käyttötapauksien määrä ja siirtymien määrä, joissa numeerisen asteikon sijaan käytetään kyseisten muuttujien määrää suhteessa koko verkkoon.
  Esimerkiksi siirtymien määrää ilmaiseva suhde määritetään laskemalla solmun asteluku \(d_G(v)\), eli solmuun liittyneiden kaarien määrä, jaettuna kaikilla verkossa olevien kaarien määrällä.
  Lisäksi siirtymien määrän suhde vielä kerrotaan luvulla 10, jotta se saadaan skaalautumaan muiden muuttujien kanssa samalle tasolle.

  \begin{table}[H]
    \caption{Priorisointiin vaikuttavat muuttujat}
    \label{tab:priorisointiin_vaikuttavat_muuttujat}
    \centering
    \begin{tabular}{lllll} \hline
    \(m\) & \textbf{Muuttuja} & \textbf{Etumerkki} & \textbf{Asteikko} &  \\ \hline
    \textbf{1} & Liiketoiminnallinen arvo & \(+\) & 1 - 10 &  \\
    \textbf{2} & Liiketoiminnallinen visio & \(+\) & 1 - 10 &  \\
    \textbf{3} & Negatiivinen käyttäjäpalaute & \(+\) & 1 - 5 &  \\
    \textbf{4} & Käyttötapauksien määrä & \(+\) & 10 \(\cdot\) suhde &  \\
    \textbf{5} & Siirtymien määrä & \(+\) & 10 \(\cdot\) suhde &  \\
    \textbf{6} & Positiivinen käyttäjäpalaute & \(-\) & 1 - 5 &  \\
    \textbf{7} & Muutosherkkyys & \(-\) & 1 - 10 &  \\
    \textbf{8} & Toteuttamisen kompleksisuus & \(-\) & 1 - 5 &  \\
    \textbf{9} & Toteutuksen virheherkkyys & \(-\) & 1 - 5 &  \\
    \textbf{10} & Omat lisämuuttujat & \(\pm\) & 1 - 10 & \\ \hline
    \end{tabular}
  \end{table}

\section{Painofunktiot priorisointiin} \label{ch:10_painofunktiot_priorisointiin}

  Painofunktioiden määrittäminen on tärkeä osa painotetun verkon avulla priorisointia, sillä niiden avulla määritetään verkon solmujen ja kaarien prioriteetit.
  Tavanomaisesti numeerinen prioriteetti usein mielletään olevan korkea, jos priorisoitu muuttuja on tärkeä.
  Painotettujen verkkojen tapauksessa on kuitenkin järkevää vaihtaa numeerisen prioriteetin suuntaa, jotta painotettuun verkoon sovellettavat lyhimmän polun algoritmit toimisivat etsien prioriteetiltaan tärkeitä polkuja. Ennen prioriteetin suunnanvaihtoa, voidaan kokonaisprioriteetti \(p\) yksittäiselle solmulle \(v\), eli näkymälle määrittää seuraavasti.

  \[p(v) = \sum\limits_{i=1}^{5} m_i - \sum\limits_{j=6}^{9} m_j \pm m_{10}\]

  Prioriteetin suunnan vaihtamiseksi suuresta pieneen, säilyttäen kuitenkin prioriteetin sisältämän informaation, voi hoitaa käänteislukujen avulla.
  Ennen käänteisluvuksi muuttamista, prioriteettiin vaikuttavien muuttujien yhteenlaskettu summa voi olla ongelmallisesti negatiivinen tai nolla.
  Negatiiviset arvot eivät ole painotetun verkon kannalta erityisen järkeviä, sillä tässä diplomityössä hyödynnettävää Dijkstran algoritmia ei voida käyttää negatiivisien painojen kanssa.
  Dijkstran algoritmin toiminta nollan tapauksessa voi myös kuulostaa epäilyttävältä, kuten esimerkiksi tilanne, jossa painotetun verkon kaikki painot olisivat nollia.
  Dijkstran algoritmin tapauksessa tällainen verkko on kuitenkin sallittu, koska silloin lyhimmän polun ratkaisu on verkon kaikki solmut.
  Lyhimmän polun ongelman erityisvaatimusten lisäksi käänteislukua varten nolla on huono arvo siinä mielessä, että sille ei ole olemassa lainkaan käänteislukua.
  Tämä johtuu siitä, että jos nollalle yrittäisi etsiä käänteislukua, tulisi eteen nollalla jakaminen jota ei voi tehdä.
  Nämä molemmat ongelmatapaukset voidaan kuitenkin painofunktiossa ratkaista siten, että käänteisfunktiota ei etsitä, vaan korvataan painofunktion tulos yhdellä.

  Painofunktio yksittäiselle solmulle \(v\), eli näkymälle saadaan solmun kokonaisprioriteetin \(p(v)\) käänteislukuna.

  \[\alpha(v) = \begin{cases}
    p^{-1}(v) & p(v) > 0 \\
    1 & p(v) \leq 0
  \end{cases}\]

  Painofunktio yksittäiselle solmut \(v_x\) ja \(v_y\) yhdistävälle kaarelle \(e_{xy}\), eli siirtymälle saadan myös käänteislukuna.
  Kaaren painofunktiota varten pitää kuitenkin huomioida, että sen kokonaisprioriteetti on kaaren solmujen kokonaisprioriteetin summa \(p(v_x) + p(v_y)\).
  Kaaren kokonaisprioriteetti \(p(v_1) + p(v_2)\) pitää laskea ennen käänteisluvuksi muuttamista.

  \[\beta(e_{xy}) = \begin{cases}
    (p(v_x) + p(v_y))^{-1} & p(v_x) + p(v_y) > 0 \\
    1 & p(v_x) + p(v_y) \leq 0
  \end{cases}\]

\section{Verkon rakentaminen} \label{ch:10_verkon_rakentaminen}

  Tässä diplomityössä on aiemmin moneen otteeseen kerrottu näkymä ja siirtymäperusteisesta testiautomaation toteuttamisesta ja priorisoinnista.
  Painotetun verkon rakentamista varten tulee tarvittavat näkymät ja niiden väliset siirtymät muodostavat testauskohteen käyttöliittymästä.
  Web-sovelluksen käyttöliittymän näkymiä ovat muun muassa sivut, sivujen sisältämät säiliö-elementit ja dialogit.
  Siirtymät ovat usein sivujen välisiä linkkejä tai vaihtoehtoisesti jotakin sellaista toiminnallisuutta, joka muuttaa nykyisen näkymän tai osan siitä toiseksi näkymäksi.

  Seuraavassa taulukossa on esitetty kuvitteellisen web-sovelluksen mukainen näkymien ja siirtymien mukaan laadittu esimerkki \ref{tab:esimerkki_verkon_priorisointi_muuttujat}.
  Taulukossa esitetään näkymät kirjautumisnäkymästä ohjenäkymään ja jokaisen näkymän siirtymät eli yhteydet toisiin näkymiin.
  Näkymät ja siirtymät luovat matemaattisen verkon laatimisen perusedellytykset, eli datan jonka avulla myöhemmin esitettävä painomatriisi voidaan laatia.
  Taulukossa on lisäksi esitetty jokainen näkymään liittyvä priorisointiin vaikuttava muuttuja.
  Priorisointiin vaikuttavien muuttujien arvot on laadittu subjektiivisesti kuvitteellisen esimerkin muodossa.
  Priorisointiin vaikuttavien muuttujien yhteenlaskettu prioriteetti yksittäiselle näkymälle on laskettu taulukkoon valmiiksi käyttäen aiemmin esitettyä prioriteettifunktiota \(p(n)\), jossa \(n\) tarkoittaa sitä näkymää jolle prioriteetti lasketaan.

  \begin{table}[H]
    \caption{Esimerkkiverkon näkymät, siirtymät ja priorisointimuuttujat}
    \label{tab:esimerkki_verkon_priorisointi_muuttujat}
    \centering
    \begin{tabular}{lllllllllllll} \hline
    \(n\) & \textbf{Näkymä} & \textbf{Siirtymät} & \(m_1\) & \(m_2\) & \(m_3\) & \(m_4\) & \(m_5\) & \(m_6\) & \(m_7\) & \(m_8\) & \(m_9\) & \(p(n)\) \\ \hline
    \textbf{A} & Kirjautuminen & B & 10 & 10 & 0 & 2 & 1 & 0 & 5 & 5 & 5 & 8 \\
    \textbf{B} & Pelivalikko & A, C, D, G & 8 & 10 & 1 & 2 & 4 & 4 & 5 & 5 & 5 & 6 \\
    \textbf{C} & Asetukset & A, B & 4 & 6 & 5 & 2 & 2 & 2 & 5 & 5 & 5 & 2 \\
    \textbf{D} & Peli & B, E, G & 10 & 10 & 4 & 2 & 3 & 4 & 4 & 5 & 5 & 11 \\
    \textbf{E} & Tulokset & B, D, F & 6 & 8 & 0 & 2 & 3 & 5 & 5 & 4 & 5 & 2 \\
    \textbf{F} & Onnittelu & B, E & 1 & 8 & 0 & 0 & 2 & 2 & 5 & 2 & 5 & -3 \\
    \textbf{G} & Ohje & B, D & 1 & 10 & 2 & 0 & 2 & 0 & 8 & 0 & 0 & 7 \\ \hline
    \end{tabular}
  \end{table}

  Painotetun verkon rakentamisen syötteeksi täytyy käyttöliittymän näkymät ja siirtymät sekä niiden painoarvot esittää painomatriisin muodossa.
  Painoarvot saadaan aiemmin esitetyn painofunktion \(\beta\) avulla \ref{ch:10_painofunktiot_priorisointiin}.
  Painoarvo lasketaan \(\beta\) funktion avulla jokaiselle kahta näkymää yhdistävälle siirtymälle, eli painotetun verkon solmujen väliselle kaarelle.
  Painofunktio \(\beta\) käyttää kaaren molempien päätepisteiden \(v_A\) ja \(v_B\) yhteenlaskettua prioriteettia, josta käänteisluku otetaan.
  Näin saadaan laskettua kaarelle sellainen painoarvo, joka tarkoittaa painotetussa verkossa siirtymän näkymiin sidottua prioriteettia.

  \[\beta(e_{AB}) = (p(v_A) + p(v_B))^{-1} = (8 + 6)^{-1} = \frac{1}{14} \approx 0.071\]

  Painomatriisi, esimerkin mukaiselle datalle \ref{tab:esimerkki_verkon_priorisointi_muuttujat} ja siitä lasketuille painoarvoille on esitetty seuraavassa matriisissa \(M_G\).
  Painomatriisissa tulee väistämättä esiin tilanne, jossa pitää määrittää painoarvo kaarelle jonka aloitussolmu ja lopetussolmu ovat sama solmu itsessään.
  Tällaisissa tapauksissa, tilanteesta riippuen painomatriiseihin usein merkitään \(0\), \(\infty\) tai \(-\).
  Tässä diplomityössä esitettävän menetelmän painomatriiseissa solmuun itseensä johtuvan kaaren, eli silmukan painoksi merkitään aina \(-\), koska käyttöliittymän näkymästä siirtymät itseensä ei menetelmässä käsitellä aitoina siirtyminä.
  Tämän lisäksi luonnollisesti jokainen sellainen solmupari, jolla ei ole niitä yhdistävää kaarta merkitään painomatriisiin käyttäen \(-\) merkintää.
  Sellaisien siirtymien toiminnallisuuden testaaminen on tarkoitus kattaa näkymän mukaisen testikokelman testitapauksissa ja ne tulee priorisoiduiksi näkymä ja siirtymäperusteisesti.
  Painotetun verkon kaaret voivat verkkoteorian mukaan olla suunnattuja tai suuntaamattomia.
  Tässä esimerkkitapauksessa jokainen siirtymä näkymien välillä on suuntaamaton, eli toisin sanoen käyttöliittymässä kaksisuuntainen ja se priorisoidaan sen mukaisesti.
  Painomatriisissa suuntaamattomien kaarien johdosta voidaan huomata, että painomatriisin diagonaalin erottamat puoliskot ovat toistensa peilikuvia.

  \[
    M_G \approx
    \bordermatrix{
      G   & v_A   & v_B   & v_C   & v_D   & v_E   & v_F   & v_G   \cr
      v_A & -     & 0.071 & 0.100 & -     & -     & -     & -     \cr
      v_B & 0.071 & -     & 0.125 & 0.059 & 0.125 & 0.333 & 1.000 \cr
      v_C & 0.100 & 0.125 & -     & -     & -     & -     & -     \cr
      v_D & -     & 0.059 & -     & -     & 0.077 & -     & 0.250 \cr
      v_E & -     & 0.125 & -     & 0.077 & -     & 1.000 & -     \cr
      v_F & -     & 0.333 & -     & -     & 1.000 & -     & -     \cr
      v_G & -     & 1.000 & -     & 0.250 & -     & -     & -     \cr
    }
  \]

  Painomatriisin avulla voidaan siis rakentaa matemaattinen painotettu verkko, joka kuvaa näkymiä ja siirtymiä sekä niiden prioriteetteja.
  Painotetun verkon kuvaamiseen piirretään jokainen erilaista käyttöliittymän näkymää vastaava ja esimerkkidatan \ref{tab:esimerkki_verkon_priorisointi_muuttujat} mukainen solmu ja niiden välisiä siirtymiä kuvaavat yhteydet eli kaaret.
  Kaarien yhteyteen lisätään kuvaajassa kaaren prioriteettia kuvaava painoarvo.
  Seuraavassa on esitetty painomatriisin dataa vastaava painotetun verkon kuvaaja \ref{fig:painotettu-verkko-ennen} sellaisena, kuin se on ennen siihen tehtäviä prioriteettileikkauksia.
  Priorisoimista varten tehtävien leikkauksien tekeminen esitetään myöhemmin omassa kappaleessaan \ref{ch:10_verkon_karsiminen}.

  \begin{figure}[H]
    \centering
    \includegraphics[width=0.6\textwidth]{assets/painotettu-verkko-ennen.png}
    \caption{Esimerkki painotetusta verkosta ennen leikkauksia}
    \label{fig:painotettu-verkko-ennen}
  \end{figure}

  Perinteisesti painotetuissa verkoissa ei esitetä yksittäisiä solmupainoja vaan painotetun verkon painoilla tarkoitetaan solmujen välisien kaarien painoarvoja.
  Tässä diplomityössä kehitettyä menetelmää käytettäessä edellä esitettyyn painotettuun verkkoon \ref{fig:painotettu-verkko-ennen} on kuitenkin lisätty painomatriisin sisältämän informaation lisäksi painofunktion \(\alpha(v)\) avulla lasketut yksittäisten solmujen eli näkymien painoarvot.

  \[\alpha(v_A) = p^{-1}(v_A) = 8^{-1} = \frac{1}{8} = 0.125\]

  Yksittäisten solmujen prioriteettia kuvaavat painoarvot ovat erittäin merkittäviä ja hyödyllisiä, sillä niiden avulla voidaan järjestää itse solmut, eli näkymät prioriteettien mukaiseen järjestykseen.
  Tämän lisäksi solmujen prioriteettien avulla voidaan verkkoon muun muassa soveltaa lyhimmän polun ratkaisemiseen kehitettyjä algoritmeja, kuten myöhemmin Dijkstran algoritmin osalta esitetään omassa kappaleessaan \ref{ch:10_dijkstran_algoritmin_hyodyntaminen}.

\section{Verkon karsiminen} \label{ch:10_verkon_karsiminen}

  Painotetun verkon karsiminen eli leikkaaminen on prioriteeillä painotetun verkon tärkeä ominaisuus.
  Verkkoteorian soveltaminen prioriteettien avulla painotettuun verkkoon on erityisen hyödyllistä, kun verkon kaarissa alhainen paino tarkoittaa suurta prioriteettia.
  Verkon karsimista varten valitaan kattavuus, joka vastaa minimirajaa ja jonka jälkeen karsiminen lopetetaan.
  Kattavuus tarkoittaa myös testikattavuutta testikokoelmien näkökulmasta, sillä painotetussa verkossa jokainen solmu, eli näkymä vastaa näkymän mukaan kategorisoitua testikokoelmaa.

  Verkkoon tehtäviä leikkauksia varten tarvitsee määrittää haluttu kattavuus \(0 \leq c \leq 100\), joka on prosentuaalinen luku siitä kuinka suuri osa verkon solmuista eli näkymistä tai testikokoelmista täytyy verkkoon jäädä karsimisen jälkeenkin. Leikkauksien tekeminen ja toistaminen suoritetaan käyttäen seuraavia toimenpiteitä \(n\)-kertaa, niin kauan kunnes karsittu aliverkko on suurempi kuin kattavuuden mukaan laskettu osuus alkuperäisestä verkosta tai jos iteraatiokerralla ei enää löydy toimenpiteillä poistettavia solmuja.
  \[|V(G_s)| > c \cdot \frac{|V(G)|}{100}, G_s \subset G\]

  Tässä verkon karsimisen esimerkissä kattavuutena käytetään \(c = 80\), joka tarkoittaa esimerkin solmujen määrän \(7\) karsimista \(80 \cdot \frac{7}{100} = 5.6\), eli lukumäärään \(5\) asti.

  \begin{enumerate}
    \item Poistetaan verkosta löytyvä eristetty solmu, eli solmu jonka asteluku on nolla.
    \[d_G(v) = 0\]
    \item Poistetaan verkosta löytyvä sillattu solmu, eli solmu jonka asteluku on yksi ja paino on pienempi kuin solmujen painojen keskiarvo.
    \[d_G(v) = 1  \land \alpha(v) > \frac{1}{|V(G)|} \cdot \sum\limits_{v \in V(G)} \alpha(v)\]
    \item Poistetaan verkosta sellainen alhaisimman prioriteetin solmu, jonka asteluku on pienempi kuin solmujen astelukujen keskiarvo ja paino on pienempi kuin solmujen painojen keskiarvo.
    \[d_G(v) < max\{d_G(x) | x \in V(G)\} \land \alpha(v) > \frac{1}{|V(G)|} \cdot \sum\limits_{v \in V(G)} \alpha(v)\]
  \end{enumerate}

  \begin{figure}[H]
    \centering
    \includegraphics[width=0.6\textwidth]{assets/painotettu-verkko-jalkeen.png}
    \caption{Esimerkki painotetusta verkosta leikkauksien jälkeen}
    \label{fig:painotettu-verkko-jalkeen}
  \end{figure}

\section{Dijkstran algoritmin hyödyntäminen} \label{ch:10_dijkstran_algoritmin_hyodyntaminen}

  Priorisointimenetelmän mukaan karsittuun painotettuun verkkoon on mahdollista soveltaa lyhimmän polun ongelman ratkaisemiseen kehitettyjä algoritmeja, jolloin ne toimivat etsien alhaisimman, eli korkeimman prioriteetin polkuja.
  Lyhimmän polun etsimiseen on tarkoitus valita aina sellaiset aloitus ja lopetuspisteet, joiden välille lyhin polku verkossa halutaan etsiä.
  Lyhimmän polun löytymisen yhtenä perusedellytyksenä on verkon yhtenäisyys, joka tarkoittaa sitä, että verkon kaikkia solmuja tulee yhdistää vähintään yksi kaari ja että verkon jokaisesta solmusta on löydettävissä yhteys mihin tahansa verkon solmuun.
  Lyhimmän polun ongelma johon muun muassa Dijkstran algoritmi antaa ratkaisun on matemaattisessa muodossaan seuraavanlainen.

  \[d_G^\alpha(v_1, v_2) = min\{\alpha(P) | P:v_1 \rightarrow v_2 | v_1, v_2 \in V(G)\}\]

  Prioriteeteiltaan tärkeimmän polun löytämiseksi, valitaan ensin \(min\{\alpha(v_1) | v_1 \in V(G)\}\) ja \(min\{\alpha(v_2) | v_2 \in V(G), v_1 \neq v_2\}\) sekä etsitään sitten Dijkstran algoritmin avulla niiden välinen lyhin polku.
  Dijkstran algoritmin sisäinen toimintaperiaate ei ole tämän diplomityön näkökulmasta oleellista, mutta se on kuitenkin esitetty tarkemmin pseudokoodina liitteessä \ref{ch:13_liite_dijkstran_algoritmi}.
  Dijkstran algoritmi kahdelle painoltaan pienimmälle, eli prioriteetiltaan korkeimmalle antaa tuloksena \(v_1\) ja \(v_2\) solmuja yhdistävän polun, jonka sisältävät solmut eli näkymät ovat prioriteetiltaan tärkeimmät.
  Koska painotetun verkon painofunktiot on laadittu käänteislukuja hyödyntäen, Dijkstran algoritmi löytää painoarvoltaan matalimman, mutta prioriteetiltaan tärkeimmän polun solmujen \(v_1\) ja \(v_2\) välille.
  Näin ollen saadaan helposti ja vaivattomasti tietää sellaiset solmut eli käyttöliittymän näkymät jotka kuuluvat tärkeimpiin ja joista testiautomaation rakentaminen kannattaa aloittaa.

\section{Verkon ja testitapauksien yhteys} \label{ch:10_verkon_ja_testitapauksien_yhteys}

  Ennen testitapauksien suunnittelua tehtävä painotetun verkon avulla tehty priorisointi havainnollistaa käyttöliittymän näkymiä ja niiden välisiä siirtymiä.
  Tällaisesta painotetusta verkosta saadaan priorisoitua näkymät ja siirtymät, mutta lopulliset testitapauksien prioriteetit ovat kuitenkin testitapaukseen kuuluvien näkymien tai siirtymien prioriteetteja.
  Tämä tarkoittaa käytännössä sitä, että kun näkymät ja siirtymät on priorisoitu, on esimerkiksi yhden yksittäisen tarkasteltavana olevan näkymän toiminnoilla sama keskenään prioriteetti.
  Painotetun verkon näkymät ovatkin suoraan yhteydessä testiautomaatiota varten rakennettaviin testikokoelmiin, jotka sisältävät kokoelman testitapauksia kyseiselle näkymälle.
  Toisin sanoen, painotetun verkon näkymiä vastaavat testikokoelmat ovat varsinaisen priorisoinnin kohteena.

  Testitapaukset ja testikokoelmat kappaleessa \ref{ch:07_testitapaus_ja_testikokoelmat} on esitetty niiden välistä eroa ja sitä kuinka testikokoelmat koostuvat yhteen liittyvistä testitapauksista.
  Painotetun verkon avulla tehtävää priorisointia käyttäessä on tarkoitus ajatella testiautomaation testitapauksien kategorisoimista testikokoelmiksi käyttöliittymän näkymiä vastaavalla tavalla.
  Kun käyttöliittymän näkymillä on niitä vastaavat testikokoelmat, toimii tässä diplomityössä kehitetty painotetun verkon avulla toteutattava priorisointi oikein ja siten kuin se on tarkoitettu.
  Jos testiautomaation halutaan lisätä testitapauksia tai testikokoelmia, jotka eivät ole luettavissa painotetusta verkosta, niille ei luonnollisesti ole olemassa prioriteettia ja sellaiset täytyy käsitellä ylimääräisinä, täydentävinä testitapauksina.

  Painotetun verkon kuvaamisen seurauksena, voidaan verkosta nähdä myös paljon hyödyllistä informaatiota, kuten muun muassa siinä esiintyviä sillattuja solmuja sekä syklejä.
  Sillatut solmut ovat sellaisia käyttöliittymän näkymiä, joihin käyttäjä ei kovinkaan usein päädy ja näin ollen jos niitä lopullisessa karsitussa verkossa esiintyy, ne ovat testiautomaatin rakentamisen kannalta usein vain vähän merkitseviä.
  Eristetyt solmut ovat samaan tapaan vain vähän merkitseviä kuin sillatut solmut.
  Syklit puolestaan ovat erittäin merkittävä osa painotetussa verkossa ja testiautomaation rakentamisessa, sillä ne ovat sellaisia käyttöliittymän näkymiä ja niiden välisiä siirtymiä, jotka toistuvat käyttäjälle usein käyttöliittymää käyttäessään.
  Solmujen asteluvut kertovat myös paljon solmujen merkitsevyydestä.
  Sellainen solmu jonka asteluku, eli siihen liittyvien kaarien lukumäärä on korkea, on testiautomaation rakentamisen kannalta yhtälailla erittäin merkittävä osa testiautomaatiota.

\chapter{Tulosten tarkastelu ja arviointi} \label{ch:10_tulosten_tarkastelu_ja_arviointi}
  Tässä kappaleessa esitetään yhteenveto tutkimuksen tuloksista, ja evaluoidaan testausjärjestelmän ja priorisointimenetelmän toteutuksien onnistumista.
Ensin evaluoidaan diplomityössä suunnitellun ja asiakasyritykselle toteutetun testausjärjestelmän positiivisia sekä negatiivisia puolia.
Seuraavaksi evaluoidaan diplomityössä kehitetyn priorisointimenetelmän positiivisia ja negatiivisia puolia ja pohditaan muun muassa sitä, että kuinka hyvin soveltuva ja toistettavissa oleva kyseinen kehitetty priorisointimenetelmä on.
Lopuksi vielä esitetään toteutuksen jälkeen esiin tulleita jatkokehitysehdotuksia testausjärjestelmälle sekä priorisointimenetelmälle.

\section{Tutkimuksen konkreettiset tulokset} \label{ch:12_tutkimuksen_konkreettiset_tulokset}

  Työn tuloksena kehitetty  web-käyttöliittymien hyväksymistestauksen automatisoimisen mahdollistava testausjärjestelmä on integroitu onnistuneesti osaksi GoCD-palvelimen avulla suoritettavaa jatkuvaa integrointia.
  Lyhyesti sanottuna testausjärjestelmän osalta konkreettinen tulos koostuu järjestelmästä, joka mahdollistaa testitapauksien luomisen Robot Framework -alustalle käyttäen Selenium-kirjastoa, Xvfb-virtualisointipalvelinta ja Docker-säiliöintiohjelmistoa.
  Testausjärjestelmän toimivuus todettiin käytännössä esimerkkitestitapauksien muodossa oikeassa ympäristössään, ja testausjärjestelmän mahdollistamat ominaisuudet ovat jo itsessään oikeassa ja lopullisessa käyttöympäristössä tarvittavia.

  Web-sovelluksien näkymä- ja siirtymäperustainen, painotettua verkkoa hyödyntävä priorisointimenetelmä on myös todettu toimivaksi lähestysmistavaksi priorisointiin.
  Lyhyesti sanottuna priorisointimenetelmän tuloksena on painotettuja verkkoja hyödyntävä menetelmä, jossa määritetään priorisointiin vaikuttavat muuttujat, painofunktiot, painomatriisi, prioriteetteihin perustuvien leikkauksien tekeminen sekä prioriteettien löytäminen ja niiden lukeminen verkosta.
  Priorisointimenetelmän toimivuus käytännössä todettiin aidosta ympäristöstä yksinkertaistaen poimitusta web-sovelluksesta.
  Priorisointimenetelmän avulla todettiin, että priorisoinnin aikana toteutetut painotetun verkon leikkaukset olivat juuri niitä näkymiä, jotka vaistonvaraisesti myös ilman menetelmän käyttöä karsittaisiin.

\section{Toteutuksen evaluointi} \label{ch:12_toteutuksen_evaluointi}

  Kokonaisuutena hyväksymistestausjärjestelmän toteutus onnistui erittäin hyvin ja sen avulla on mahdollista jopa geneerisesti rakentaa web-sovelluksesta riippumattomasti hyväksymistestaus testauskohteena olevalle web-sovellukselle.

  Testausjärjestelmän positiivisia puolia ovat muun muassa Docker-säiliöinnin avulla saatava tuki myös manuaaliselle testitapauksien ajamiselle.
  Docker-säiliö, joka mahdollistaa testitapauksien ajamisen, voidaan pystyttää periaatteessa mihin tahansa ympäristöön, jossa Docker on saatavilla.
  Docker-säiliöinnin avulla ohjelmistokehittäjät myös saavat valmiin hyväksymistestausjärjestelmän helposti käyttöönsä.
  Docker-säiliöinnin ja Docker-composen avulla rakennettu järjestelmä ei myöskään ole sidottu mihinkään ennalta määritettyyn jatkuvan integroinnin palvelimeen, mikä huomattavasti helpottaa testausjärjestelmän käyttöönottoa osaksi uutta tai muuttuvaa ohjelmistotuotannon prosessia.
  Testausjärjestelmä mahdollistaa päätteettömän testauksen virtuaalisen Xfvb-näyttöpalvelimen avulla, joka on itsessään erittäin tarvittu ominaisuus jatkuvan integroinnin ja testausjärjestelmän yhdistämiseen.
  Xvfb-näyttöpalvelimen tarjoaman virtualisoinnin avulla voidaan myös lisätä uusia WebDriver-rajapinnan toteuttavia verkkoselaimia päätteettömän testauksen alaisuuteen erittäin helposti ja käytännössä rajoituksetta.
  Ainoa vaatimus on, että verkkoselain on saatavilla siihen ympäristöön, jossa Xvfb-näyttöpalvelinta ajetaan.

  Testausjärjestelmässä on kuitenkin myös negatiivisia puolia, joiden osalta järjestelmän käyttö on rajattua.
  Xvfb-näyttöpalvelimen avulla voidaan päätteettömästi testata periaatteessa mitä tahansa GUI-ohjelmia, mutta rajoitteena on kuitenkin se, että niiden täytyy olla saatavilla siihen ympäristöön, jossa Xvfb-näyttöpalvelinta ajetaan.
  Tämä tarkoittaa käytännössä sitä, että web-sovelluksien hyväksymistestauksen automatisoimisesta on jätettävä pois vain Window-ympäristöön saatavien verkkoselainten, kuten Internet Explorer -verkkoselaimen testaaminen.
  Robot Framework takaa helpon testitapauksien luettavuuden kenelle tahansa, mutta ohjelmistokehittäjille se voi olla turhan rajatun tuntuinen.
  Ohjelmistokehittäjänä testitapauksien laatimisen yhteyteen olisi hyvä saada mahdollisuus yksikkötestauskehyksissä käytettävistä ohjelmointikielistä tuttuihin kontrollirakenteisiin, joilla testitapauksien monipuolisuutta voisi kasvattaa perinteisesti yksikkötestauksessa mahdollisten rakenteiden tasolle.
  Tämä ei kuitenkaan ole mahdollista Robot Frameworkissä, jossa testitapauksien laatimiseen käytetään Robot Frameworkin omaa, rajattua syntaksia.

\section{Menetelmän evaluointi} \label{ch:12_menetelman_evaluointi}

  Tässä diplomityössä kehitetyn testitapauksien priorisointimenetelmän kehittäminen onnistui erittäin hyvin ja sen käyttämisellä saavutetaan lisäarvoa etenkin keskisuurien ja suurien web-sovelluksien käyttöliittymien hyväksymistestaukseen.

  Priorisointimenetelmän positiivisia puolia ovat muun muassa sen ominaisuuksiin liittyviä asioita, kuten priorisointimenetelmän toistettavuus, ja mahdollisuus priorisoida käyttöliittymien näkymiä ja siirtymiä.
  Näkymä- ja siirtymäperusteisen priorisoinnin tarkoituksena on mahdollistaa näkymiin perustuvien testikokoelmien priorisoiminen, jolloin niiden tärkeysjärjestys saadaan selville, ja testitapauksien kirjoittaminen voidaan aloittaa prioriteetiltaan tärkeimmästä näkymästä.
  Esimerkkinä menetelmän käyttämisen tuomasta resurssien säästöstä voidaan ottaa tarkasteluun kattavuudet \(c=90\) ja \(c=50\), keskisuurelle käyttöliittymälle, jossa on yhteensä kymmenen erilaista näkymää, ja joista jokaista varten laadittaisiin esimerkin omaisesti kolme testitapausta.
  Tällaisessa tilanteessa kaikkien testitapauksien lukumääräksi tulee kolmekymmentä, joista prioriteetein painotettua verkkoa karsimalla voidaan kuitenkin tiputtaa alhaisimman prioriteetin näkymien avulla osa testitapauksista pois.
  Tämä tarkoittaa kattavuudella \(c=90\) yhden alhaisimmalla prioriteetilla painotetun näkymän jättämistä pois testauksesta, jolloin myös kolme testitapausta voidaan jättää pois toteutuksesta.
  Vastaavasti varsin häikäilemättömällä kattavuudella \(c=50\) voitaisiin jättää pois peräti viisi näkymää, mikä tarkoittaa esimerkissä viittätoista testitapausta, jotka jätettäisiin alhaisen prioriteettinsa myötä pois toteutuksesta.
  Testitapauksien toteuttamatta jättämisellä voidaan väistämättä säästää resursseja, mutta sitä ei kuitenkaan voida tehdä täysin ilman priorisointia.
  Yksi tämän menetelmän suurimmista hyödyistä tulee nimenomaan priorisoimisesta, joka mahdollistaa resurssien säästämisen edellä mainitulla tavalla.

  Menetelmän käyttäminen on tehokkainta, kun testitapaukset kategorisoidaan näkymittäin laadittuihin testikokoelmiin, sillä menetelmän kehittämisen taustalla on ollut ajatus, jossa näkymät vastaavat testikokoelmia.
  Priorisointimenetelmä perustuu matemaattisiin painotettuihin verkkoihin, jotka tuovat hyötynä lyhimmän polun ongelman ratkaisemiseen kehitettyjen algoritmien käyttämisen mahdollistamisen prioriteetiltaan korkeimpien polkujen löytämiseen kahden solmun, eli näkymän, välille.
  Lisäksi painotetun verkon ja matemaattisen lähestymistavan käyttäminen tuo hyötynä sen, että menetelmä on kohtalaisen pienellä vaivalla muunnettavissa tietokoneohjelmaksi.
  Painotettujen verkkojen käyttäminen priorisointiin pakottaa myös menetelmän käyttäjät piirtämään näkymä- ja siirtymäperusteisen painotetun verkon, jolloin se kasvattaa käyttäjien ymmärrystä testauskohteena olevasta järjestelmästä.
  Priorisointimenetelmä on tässä diplomityössä esitetyn esimerkin \ref{tab:esimerkki_verkon_priorisointi_muuttujat} mukaan todettavissa toimivaksi, ja sen avulla on suoritettu priorisointi varsinaisesta testauskohteesta yksinkertaistetulle käyttöliittymälle.

  Priorisointimenetelmässä on myös negatiivisia puolia, mutta ne eivät tässäkään tapauksessa ylitä menetelmän käytöstä saatavaa hyötyä.
  Menetelmässä esittävien toistuvien leikkauksien määrää rajaavan testikattavuuden päättäminen näkymä- ja siirtymäperusteisesti voi olla haastavaa.
  Toinen menetelmään kohdistuva kritiikki koskee menetelmän geneerisyyttä, eli käyttöönottamisen mahdollisuutta ilman muutoksia, jota on vaikea arvioida.
  Priorisointimenetelmässä käytettävät priorisointiin vaikuttavat muuttujat ovat varsin subjektiivisia ja voivat olla testausta toteuttavan tahon mukaan muuttuvia, minkä takia muuttujiin joudutaan mahdollisesti tekemään muutoksia.
  Lisäksi menetelmässä esitetty, priorisointiin vaikuttavia muuttujia hyödyntävä, funktio \(p(v)\) kokonaisprioriteetin laskemiseen ei ota muuttujien määrittämisessä mahdollisesti esiintyvää epälineaarisuutta lainkaan huomioon.
  Tämä tarkoittaa käytännössä sitä, että kokonaisprioriteetin määrittämisen on voitava olla ilmaistavissa siihen vaikuttavien osiensa summana, mikä rajoittaa menetelmän käyttöä epälineaarisissa tapauksissa.
  Negatiivista on myös se, että menetelmän käyttö on soveltuva painotettujen verkkojen luonteen mukaisesti käyttöliittymien tapauksessa soveltuvat vain kokonaisia näkymiä peilaavien testikokoelmien priorisointiin yksittäisten testitapauksien sijaan.
  Priorisointimenetelmän käyttö voi olla turhan aikaa vievää, jos käyttöliittymä on yksinkertainen.

\section{Jatkokehitysehdotukset} \label{ch:12_jatkokehitysehdotukset}

  Tämän diplomityön konkreettisina tuloksina syntyneet testausjärjestelmä ja priorisointimenetelmä ovat sellaisenaan käyttövalmiita ja toimivaksi todettuja, mutta jatkokehittelylle on luonnollisesti niissäkin sijaa.
  Testausjärjestelmän avulla toteutettava web-sovelluksien päätön hyväksymistestaus mahdollistetaan Xvfb-näyttöpalvelimen tarjoaman virtualisoinnin avulla.
  Xvfb-näyttöpalvelin on kuitenkin saatavilla vain UNIX-ympäristöihin, mikä rajaa testausjärjestelmään lisättävien verkkoselaimien saatavuutta.
  Xvfb-näyttöpalvelimelle voitaisiin jatkokehityksenä etsiä monialustaisempi vaihtoehto tai ainakin vastine Windows-ympäristöön, jonka avulla myös vain Windows-alustalle saatavat verkkoselaimet olisi mahdollista lisätä järjestelmään.

  Yksi priorisointimenetelmään liittyvä rajoite on käyttöliittymän näkymä- ja siirtymäperusteinen priorisointi, joka asettaa näkymät vastaamaan testikokoelmia.
  Jatkokehityksenä voitaisiin tutkia näkymäperusteisuuden mukaan tehtävän priorisoinnin muuntamisen mahdollisuutta käyttötapausperusteiseksi.
  Käyttötapausperusteisesti luotava verkko vastaisi parhaimmillaan oikeita käyttäjien tarpeisiin tarkoitettuja toiminallisuuksia ja voisi parantaa priorisointia.
  Priorisointimenetelmä on vahvasti matemaattinen, joka mahdollistaa sen muuntamisen tietokoneohjelmaksi kohtalaisella vaivalla.
  Priorisointimenetelmän rakentaminen automaattisen tietokoneohjelman muotoon olisi erittäin järkevää, ja laskisi menetelmän käyttöönottamiseen tarvittavaa vaivannäköä huomattavasti.
  Lisäksi priorisointimenetelmän näkymäpohjaisten graafien, eli painotettujen verkkojen, visualisointi voitaisiin hoitaa tietokoneohjelman yhteydessä.

\chapter{Yhteenveto} \label{ch:11_yhteenveto}
  <TODO: diplomityön tavoite saavutettu....>

<TODO: tavoitteena oli xxxxxx ....>

<TODO: tutkimuskysymykseen T1 vastattiin luvussa X .....>

<TODO: tutkimuskysymykseen T2 vastattiin luvussa X .....>

<TODO: tutkimuskysymykseen T3 vastattiin luvussa X .....>

<TODO: tutkimuskysymykseen T4 vastattiin luvussa X .....>

<TODO: konkreettisista tuloksista hyötyä......>

<TODO: diplomityö loi perustan asiakasyrityksessä tarvittavan testiautomaation rakentamiseen...>
\printbibliography[heading=bibintoc]

%%%%%%%%%%%%%%%%%%%%%%%%%%%%%%%%%%%%%%%%%%%%%%%%%%%%%%%%%%%
%% Appendices
%%%%%%%%%%%%%%%%%%%%%%%%%%%%%%%%%%%%%%%%%%%%%%%%%%%%%%%%%%%
\begin{appendices}
\chapter{Esimerkki testitapauksesta Robot Framework:illä} \label{ch:12_liite_robot_testitapaus}
  \input{./tex/12_liite_robot_testitapaus.tex}
\chapter{Dijkstran algoritmi pseudokoodina} \label{ch:13_liite_dijkstran_algoritmi}
  \input{./tex/13_liite_dijkstran_algoritmi.tex}
\end{appendices}
\end{document}