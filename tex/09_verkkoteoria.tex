Tässä luvussa käsitellään työhön keskeisesti kuuluvan verkkoteorian perusteet käydää huolellisesti läpi erityisesti työssä käytettävät osat.
Työssä sovelletaan erityisesti verkkoteorian painotettua verkkoa sekä verkkoteoriassa esiintyvän lyhimmän polun ongelmaan kehitettyä Djikstran algoritmia.
Verkkoteoria itsessään on osa diskeettiä matematiikkaa.

\section{Matemaattisten verkkojen tarkoitus} \label{ch:09_matemaattisten_verkkojen_tarkoitus}

  Matemaattisten verkkojen tarkoituksena on mallintaan parittaisia riippuvuuksia verkkomaisessa objektijoukossa.
  Verkkoteoriassa peruskäsitteitä ovat itse \emph{verkko} eli \emph{graafi}, joka muodostuu \emph{solmuista} ja niiden välisiä riippuvuuksia esittävistä \emph{kaarista} tai \emph{nuolista}.
  Verkkoteorialla on lukuisia käytännön sovellutuksia. Verkkoteoriaa sovelletaan muun muassa tietokonetieteissä, kielitieteissä, fysiikan ja kemian sovellutuksissa, sosiaalisissa tieteissä ja biologiassa.
  Alun perin verkkoteoria katsotaan syntyneen 1700-luvulla esiintyneestä niin sanotusta Königsbergin siltaongelmasta, johon Leonhard Euler esitti todistuksensa.

\section{Perusmerkinnät ja käsitteet} \label{ch:09_perusmerkinnat_ja_kasitteet}

  Verkkoteoriassa käytetään seuraavia perusmerkintöjä:
  \begin{itemize}
    \item \(V := \{v_1, v_2, v_3\}\) Solmujoukko joka sisältää \emph{solmut} \(v_1\), \(v_2\) ja \(v_3\).
    \item \(E := \{e_1, e_2, e_3\}\) Kaarijoukko joka sisältää \emph{kaaret} \(e_1\), \(e_2\) ja \(e_3\).
    \item \(\phi(e_1) := \langle v_1, v_2 \rangle\) Kaariparin \(v_1\) ja \(v_2\) yhdistävän \emph{kaaren} \(e_1\) kuvaaja.
  \end{itemize}

  Verkkojen solmujen välisiä yhteyksiä, eli kaaria esitetään usein myös yhteys- tai painomatriisina.
  \[
    M_G = (a_{ij})_{3\times3} =
    \bordermatrix{
      G & v_1 & v_2 & v_3 \cr
      v_1 & 1 & 1 & 2 \cr
      v_2 & 1 & 0 & 0 \cr
      v_3 & 2 & 0 & 0 \cr
    }
  \]

  Verkkoteoriassa käytetään myös muun muassa seuraavia käsitteitä:
  \begin{itemize}
    \item Solmun asteluku, \(d_G(x)\), eli solmuun liittyvien \emph{kaarten} määrä.
    \item Aliverkko, \(G_2 \subset G_1\), eli \emph{verkko} \(G_2\) joka koostuu osasta \emph{verkon} \(G_1\) \emph{solmuja} ja \emph{kaaria}.
    \item Verkon komplementti, \(G'\), eli sellainen \emph{verkko}, jossa on kaikki ne \emph{kaaret} joita \emph{verkossa} \(G\) ei esiinny.
    \item Verkon yhtenäisyys, \(v_1 \neq v_2, v_1 \rightarrow v_2\) , eli jokaiselle solmuparille \(v_1 \neq v_2\) on olemassa niitä yhdistävä \emph{kaari}.
    \item Polku, \(P = \{v_0, v_1, ..., v_n\}, v_0 \rightarrow v_n\), eli \emph{suunnattu solmujono} jota pitkin voidaan kulkea \emph{solmusta} \(v_0\) \emph{solmuun} \(v_n\).
    \item Sykli, \(P = \{v_0, v_1, ..., v_n| e \in E_P, e \notin \{E_P \setminus \{e\} \}\}\) eli \emph{polku}, jonka aloitus \(v_0\) ja lopetussolmu \(v_n\) on sama, mutta polun jokaista kaarta \(e\) kuljetaan vain kerran.
    \item Eristetty solmu, \(d_G(v_1) = 0\), eli \emph{solmu} jonka \emph{asteluku} on nolla.
    \item Silta, \(v_1 \rightarrow v_2, d_G(v_1) = 1 \lor d_G(v_2) = 1\), eli \emph{kaari} johon yhdistyvän \emph{solmun asteluku} on yksi ja jonka poistaminen epäyhteinäistää \emph{verkon}.
    \item Silmukka, \(v_x \rightarrow v_x\), eli \emph{kaari} jonka \emph{aloitussolmut} ja \emph{lopetussolmu} ovat sama \emph{solmu}.
    \item Nuoli, \(\rightarrow\), eli \emph{suunnatussa verkossa} esiintyvä \emph{suunnattu kaari}.
  \end{itemize}

\section{Painotettu verkko ja leikkaaminen} \label{ch:09_painotettu_verkko_ja_leikkaaminen}

  \begin{itemize}
    \item \(\alpha := V(G), E(G) \rightarrow \mathbb{N}\) Painofunktion yleinen kuvaus.
    \item TODO: esitä leikaaminen matemaattisesti
  \end{itemize}

\section{Lyhimmän polun ongelma} \label{ch:09_lyhimman_polun_ongelma}

  \begin{itemize}
    \item \(d_G^\alpha(v_1, v_2) = min\{\alpha(P) | P:v_1 \rightarrow v_2 | v_1, v_2 \in V(G)\}\) Lyhimmän polun ongelma.
    \item Ongelman ratkaiseminen Djikstran algoritmilla
  \end{itemize}
