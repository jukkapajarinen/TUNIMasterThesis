Tämän diplomityön tavoitteena oli jo aluksi laaditun tutkimusasetelmankin mukaisesti kehittää hyväksymistestausjärjestelmä ja toistettavissa oleva menetelmä web-käyttöliittymien hyväksymistestauksessa tarvittavien testitapauksien priorisointiin.
Lisäksi tavoitteena oli tarjota selkeä, eheä ja helposti ymmärrettävä kokonaisuus hyväksymistestauksen toteuttamiseen ja priorisoimiseen testausjärjestelmää ja kehitettyä priorisointimenetelmää käyttäen.
Tutkimusta varten laadittiin neljä tutkimuskysymystä, joihin vastaaminen asetettiin myös yhdeksi työn tavoitteeksi.

Tutkimuskysymykseen \emph{T1} vastattiin kokonaisuutena priorisointi painotetun verkon avulla luvussa \ref{ch:09_priorisointi_painotetun_verkon_avulla}, eli esitetään ratkaisuna toistettavissa oleva menetelmä testitapauksien priorisoimiseen.
Priorisointiin vaikuttavat muuttujat luvussa \ref{ch:10_priorisointiin_vaikuttavat_muuttujat} esitetään myös suora vastaus tutkimuskysymykseen \emph{T2}.
Lisäksi painofunktiot \ref{ch:10_painofunktiot_priorisointiin} ja verkon karsiminen \ref{ch:10_verkon_karsiminen} esittää vastaukset tutkimuskysymykseen \emph{T3}.
Lopuksi verkon ja testitapauksien yhteys \ref{ch:10_verkon_ja_testitapauksien_yhteys} antaa suoran vastauksen tutkimuskysymykseen \emph{T4}.

Tässä diplomityössä kehitettiin hyväksymistestausjärjestelmä sekä siihen rakennettavien testitapauksien priorisointimenetelmä.
Hyväksymistestausjärjestelmä laadittiin käyttäen Robot Framework:iä, Selenium:ia, Xvfb-näyttöpalvelinta ja Docker:ia.
Lisäksi jatkuvan integroinnin tarve otettiin huomioon käyttäen GoCD-palvelinta hyväksymistestausjärjestelmän kanssa.
Testitapauksien priorisointimenetelmä kehitettiin itsenäisesti käyttäen matemaattista painotettua verkkoa.
Painotetun verkon avulla tehtävään priorisointiin liittyivät tärkeimpinä priorisointiin vaikuttavat muuttujat, painofunktiot ja verkon karsimiseksi tehtävät leikkaukset.
Painotetun verkon avulla tehtävä priorisointi todettiin toimivaksi esimerkin ja aidosta sovelluksesta tehdyn yksinkertaistetun mallin avulla.

Edellä mainitut diplomityön tavoitteet ovat nyt saavutettu ja sen tekeminen myös loi tarvittavan perustan WordDivellä tarvittavan web-sovelluksen hyväksymistestauksen testiautomaation rakentamiseen priorisointimenetelmää hyödyntäen.
Lopuksi vielä toivon, että tässä diplomityössä esitetystä hyväksymistestauksen testiautomaation lähestymistavasta ja erityisesti sen priorisointia varten kehitetystä priorisointimenetelmästä olisi mahdollisimman paljon hyötyä sen käyttämistä harkitseville tai käyttäville tahoille.