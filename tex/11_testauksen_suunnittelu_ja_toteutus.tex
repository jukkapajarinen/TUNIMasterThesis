Tässä luvussa esitetään työn perusteella tehty esimerkkitoteutus sekä käytännön sovelluskehyksen ja työkalut.
Tämän luvun tarkoituksena on todistaa menetelmän toimivuus oikeassa ohjelmistotuotannon ympäristössä toteuttaen samalla testiautomaatio asiakasyritykselle.

\section{Käyttöliittymän näkymät ja siirtymät} \label{11_kayttoliittyman_nakymat_ja_siirtymat}

% Teksti tähän

\section{Painotetun verkon rakentaminen} \label{11_painotetun_verkon_rakentaminen}

% Teksti tähän

\section{Painotetun verkon käyttö priorisointiin} \label{11_painotetun_verkon_kaytto_priorisointiin}

% Teksti tähän

\section{Sovelluskehykset ja työkalut} \label{11_sovelluskehykset_ja_tyokalut}

  % Teksti tähän

  \subsection{Docker} \label{11_docker}

  % Teksti tähän

  \subsection{GoCD} \label{11_gocd}

  % Teksti tähän

  \subsection{Robot Framework} \label{11_robot_framework}

  % Teksti tähän

  \subsection{Selenium} \label{11_selenium}

  % Teksti tähän

\section{Testitapauksien toteuttaminen} \label{11_testitapauksien_toteuttaminen}

% Teksti tähän

\section{Testitapauksien suorittaminen} \label{11_testitapauksien_suorittaminen}

Tässä kappaleessa esitetään vastausta tutkimuskysymykseen \emph{T4}, keskittyen jatkuvaan integroinnin ja testitapauksien priorisoinnin yhteyteen.

% Teksti tähän
