Tässä luvussa esitetään työn perusteella tehty esimerkkitoteutus sekä käytännön sovelluskehyksen ja työkalut.
Tämän luvun tarkoituksena on todistaa menetelmän toimivuus oikeassa ohjelmistotuotannon ympäristössä toteuttaen samalla testiautomaatio asiakasyritykselle.

\section{Käyttöliittymän näkymät ja siirtymät} \label{ch:11_kayttoliittyman_nakymat_ja_siirtymat}

  % Teksti tähän

\section{Painotetun verkon rakentaminen} \label{ch:11_painotetun_verkon_rakentaminen}

  % Teksti tähän

\section{Painotetun verkon käyttö priorisointiin} \label{ch:11_painotetun_verkon_kaytto_priorisointiin}

  % Teksti tähän

\section{Sovelluskehykset ja työkalut} \label{ch:11_sovelluskehykset_ja_tyokalut}

  % Teksti tähän

  \subsection{Docker} \label{ch:11_docker}

    % Teksti tähän

  \subsection{GoCD} \label{ch:11_gocd}

    % Teksti tähän

  \subsection{Robot Framework} \label{ch:11_robot_framework}

    % Teksti tähän

  \subsection{Selenium} \label{ch:11_selenium}

    % Teksti tähän

\section{Testitapauksien toteuttaminen} \label{ch:11_testitapauksien_toteuttaminen}

  % Teksti tähän

\section{Testitapauksien suorittaminen} \label{ch:11_testitapauksien_suorittaminen}

  Tässä kappaleessa esitetään vastausta tutkimuskysymykseen \emph{T4}, keskittyen jatkuvaan integroinnin ja testitapauksien priorisoinnin yhteyteen.

  % Teksti tähän
