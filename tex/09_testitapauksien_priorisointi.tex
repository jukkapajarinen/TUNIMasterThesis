Tässä luvussa esitetään tutkimuksen tärkein sisältö, eli toistettavissa oleva menetelmä testitapauksien priorisoimiseen.
Priorisointia varten esitetään harkintaa käyttäen lähdeaineistosta suodatetut priorisointiin vaikuttavat muuttujat, painofunktio, testitapauksien näkymäperusteinen koostaminen ja painotetun verkon laatiminen.
Lisäksi menetelmää käyttäen tuotetun painotetun verkon sisältämää informaatiota käytetään prioriteeteiltaan tärkeiden polkujen löytämiseen ja testikattavuuden arviointiin.

\section{Priorisointiin vaikuttavat muuttujat}

\begin{itemize}
  \item Liiketoiminnallinen arvo
  \item Projektin muuttumisen volatiliteetti
  \item Kehittämisen kompleksisuus
  \item Vaatimusten taipumus virheellisyyteen
\end{itemize}

\section{Painofunktio priorisointiin}

Painofunktion yleinen kuvaus.
\[\alpha := E(G) \rightarrow \mathbb{N}\]

Painofunktio yksittäiselle solmulle \(v\) tai kaarelle \(e\).
\[\alpha(v|e) = value - volatility - complexity - errorness\]

Painofunktion polulle \(P\) solmusta \(v_1\) solmuun \(v_2\).
\[\alpha(P) = \sum_{v \in P} \alpha(v) + \sum_{e \in P} \alpha(e)\]

\section{Käyttöliittymän näkymät ja siirtymät}

\section{Painotetun verkon rakentaminen}

\section{Painotetun verkon karsiminen}

Painotetun verkon karsiminen eli leikkaaminen on prioriteeillä painotetun verkon tärkeä ominaisuus.
Verkkoteorian soveltaminen prioriteettien avulla painotettuun verkkoon on erityisen hyödyllistä, kun verkon kaarissa korkea paino tarkoittaa suurta prioriteettia.
Tällaisessa tapauksessa on mahdollista soveltaa lyhimmän polun ongelman ratkaisemiseen kehitettyjä algoritmeja, jolloin ne toimivat etsien alhaisimman prioriteetin polkuja.
Lyhimmän polun etsimiseen on tarkoiksenmukaista valita aina aloitus ja lopetuspisteet, joiden välille lyhin polku verkossa voidaan etsiä.
Prioriteetein painotetun verkon karsimistarkoitukseen olisi järkevää valita sellaiset aloitus- ja lopetuspisteet, joiden välillä ei näyttäisi olevan korkean prioriteetin solmuja.
Voidaan kuitenkin menetellä myös siten, että valitaan aloitus- ja lopetuspisteeksi sellaiset solmut, jotka ovat painoltaan verkon alhaisimmat \(v_1 = min(V)\) ja \(v_2 = min(V \setminus \{v_1\})\) ja verrata niiden lyhimmän polun kokonaisprioriteettia muuhun verkkoon.

  \subsection{Dijkstran algoritmin soveltaminen}

  \begin{itemize}
    \item Pienimmän prioriteetin solmuparin etsiminen.
    \item Dijkstran algoritmin käyttö lyhimmän (prioriteetiltaan pienimmän) polun löytämiseen.
    \item Leikkauksien tekeminen ja toistaminen \(x\)-kertaa.
  \end{itemize}

  \subsection{Leikkauksien tekeminen}

  \begin{itemize}
    \item Poistetaan yksittäiseen solmuun johtavat sillat.
    \item Poistetaan yksittäiset eristetyt solmut.
    \item Poistetaan Dijkstran lyhimmän polun kaaret.
  \end{itemize}

\section{Testitapauksien muodostaminen}
