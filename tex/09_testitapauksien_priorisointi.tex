Tässä luvussa esitetään tutkimuksen tärkein sisältö, eli toistettavissa oleva menetelmä testitapauksien priorisoimiseen.
Priorisointia varten esitetään harkintaa käyttäen lähdeaineistosta suodatetut priorisointiin vaikuttavat muuttujat, painofunktio, testitapauksien näkymäperusteinen koostaminen ja painotetun verkon laatiminen.
Lisäksi menetelmää käyttäen tuotetun painotetun verkon sisältämää informaatiota käytetään prioriteeteiltaan tärkeiden polkujen löytämiseen ja testikattavuuden arviointiin.

\section{Priorisointiin vaikuttavat muuttujat}

\begin{itemize}
  \item Liiketoiminnallinen arvo
  \item Projektin muuttumisen volatiliteetti
  \item Kehittämisen kompleksisuus
  \item Vaatimusten taipumus virheellisyyteen
\end{itemize}

\section{Painofunktio priorisointiin}

\[f(a,b,c,d) = a+b+c+d\]

\section{Testitapauksien koostaminen näkymistä}

<Lisää teksti tähän>

\section{Painotetun verkon rakentaminen}

<Lisää teksti tähän>

\section{Korkeiden prioriteettien polut}

<Lisää teksti tähän>

\section{Testikattavuuden arviointi}

<Lisää teksti tähän>
