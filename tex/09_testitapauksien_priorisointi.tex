Tässä luvussa esitetään tutkimuksen tärkein sisältö, eli toistettavissa oleva menetelmä testitapauksien priorisoimiseen.
Priorisointia varten esitetään harkintaa käyttäen lähdeaineistosta suodatetut priorisointiin vaikuttavat muuttujat, painofunktio, testitapauksien näkymäperusteinen koostaminen ja painotetun verkon laatiminen.
Lisäksi menetelmää käyttäen tuotetun painotetun verkon sisältämää informaatiota käytetään prioriteeteiltaan tärkeiden polkujen löytämiseen ja testikattavuuden arviointiin.

\section{Priorisointiin vaikuttavat muuttujat}

\begin{itemize}
  \item Liiketoiminnallinen arvo
  \item Projektin muuttumisen volatiliteetti
  \item Kehittämisen kompleksisuus
  \item Vaatimusten taipumus virheellisyyteen
\end{itemize}

\section{Painofunktio priorisointiin}

Painofunktion yleinen kuvaus.
\[\alpha := E(G) \rightarrow \mathbb{N}\]

Painofunktio yksittäiselle solmulle \(v\) tai kaarelle \(e\).
\[\alpha(v|e) = value - volatility - complexity - errorness\]

Painofunktion polulle \(P\) solmusta \(v_1\) solmuun \(v_2\).
\[\alpha(P) = \sum_{v \in P} \alpha(v) + \sum_{e \in P} \alpha(e)\]

\section{Käyttöliittymän näkymät ja siirtymät}

<Lisää teksti tähän>

\section{Painotetun verkon rakentaminen}

<Lisää teksti tähän>

\section{Painotetun verkon karsiminen}

<Lisää teksti tähän>

  \subsection{Dijkstran algoritmin soveltaminen}

  \begin{itemize}
    \item Pienimmän prioriteetin solmuparin etsiminen.
    \item Dijkstran algoritmin käyttö lyhimmän (prioriteetiltaan pienimmän) polun löytämiseen.
    \item Leikkauksien tekeminen ja toistaminen \(x\)-kertaa.
  \end{itemize}

  \subsection{Leikkauksien tekeminen}

  \begin{itemize}
    \item Poistetaan yksittäiseen solmuun johtavat sillat.
    \item Poistetaan yksittäiset eristetyt solmut.
    \item Poistetaan Dijkstran lyhimmän polun kaaret.
  \end{itemize}

\section{Testitapauksien muodostaminen}

<Lisää teksti tähän>
