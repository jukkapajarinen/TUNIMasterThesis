Tässä luvussa esitetään perusteet ja tarvittavat tiedot hyväksymistestauksesta, johon testauksen tasoista tässä diplomityössä keskitytään.
Ensin esitetään hyväksymistestauksen tarkoitus, jonka jälkeen keskitytään hyväksymistestausvetoiseen kehitykseen ja sen esittelemiseen ohjelmistotuotannollisena menetelmänä.

Hyväksymistestausvetoisen kehityksen jälkeen käydään läpi tässä diplomityössä käytettyä ja lähes de facto testialustaa, Robot frameworkia, hyväksymistestauksen testitapauksien rakentamiseen.
Robot frameworkin perusteiden jälkeen esitetään hyväksymistestauksen testitapauksien laatiminen käyttäen Robot frameworkia sekä esitetään web-sovelluksien erityispiirteitä jotka on huomioitava hyväksymistestauksessa.
Hyväksymistestaksen perusteiden ymmärtämistä tarvitaan työn toteutusvaiheessa, jossa esitetään korkealla tasolla asiakasyritykselle toteutettua hyväksymistestausta ja sen automaatiota.

Lopuksi esitetään yleisestikin ottaen testitapauksiin tärkeästi liittyvä priorisointiongelma ja käydään läpi sen eri ratkaisumalleja, keskittyen erityisesti tässä diplomityössä myöhemmin esitettävään priorisointiin painotetun verkon avulla.

\section{Hyväksymistestauksen tarkoitus} \label{ch:08_hyvaksymistestauksen_tarkoitus}

  Hyväksymistestauksen tarkoituksena on varmistaa toteutettavan ohjelmiston vaatimusten toimivuus erityisesti käytännön tilanteissa siten, että voidaan varmistaa vastaako ohjelmisto loppukäyttäjän tarpeita.
  Hyväksymistestaus antaa vastauksen siihen, toimiiko toteutettu järjestelmä loppukäyttäjän tarpeiden mukaisesti ja loppukäyttäjän näkökulmasta oikein.
  Hyväksymistestauksen sanotaan olevan muodollista testaamista, jossa käyttäjän tarpeet, vaatimukset ja liiketoimintaprosessit otetaan huomioon selvittäessä täyttääkö järjestelmä hyväksymisen kriteerit ja sallii käyttäjän, asikkaiden tai muun autorisoidun tahon päättää hyväksytäänkö järjestelmä \parencite{istqb_glossary_nodate}.
  Ohjelmistotestauksen tekniikoiden näkökulmasta hyväksymistestaus on mustalaatikkotestausta, eli sitä testataan tietämättä sen teknisestä toteutuksesta.
  Hyväksymistestauksen painoarvo on asiakaperusteisessa vaatimusmäärittelyssä ja loppukäyttäjän tarpeiden kartoittamisessa.
  Testiautomaation osalta hyväksymistestausta varten voidaan rakentaa testitapaukset, joiden avulla voidaan keskittyä varmistamaan loppukäyttäjille tarpeellisten toimintojen toteutuminen testitapauksien suorittamisen jälkeen.
  Hyväksymistestauksen osalta testitapauksia voidaan toteuttaa niin sanotulla päästä päähän -periaatteella, jossa testattavaa järjestlemää testataan siten kuin loppukäyttäjä sitä käyttää.
  Hyväksymistestauksessa ei anneta suurta painoarvoa kosmeettisille tai kirjoitusvirheille, vaan pyritään selvittämään loppukäyttäjille oleellisten ja tarpeellisten toimintojen toteutuminen.

  Hyväksymistestaus on aiemmin esitetyistä testauksen tasoista \ref{ch:07_testauksen_tasot} viimeinen ja sen suorittamisen jälkeen saadaan tieto siitä onko järjestlemä toteutuksen osalta sellaisenaan valmis julkaistavaksi.
  Perinteisesti hyväksymistestauksen lähtökohtia ovat selvät hyväksymisvaatimukset sekä julkaisukelpoinen toteutus joka voi sisältää vain kosmeettisia virheitä.
  Hyväksymisvaatimukset voivat olla esimerkiksi liiketoiminnallisia käyttötapauksia, prosessivirtauskaavioita sekä ohjelmiston vaatimusmäärittely.
  Testiautomaatiota varten käytettävästä testialustasta riippuen hyväksymistestauksen käyttötapaukset voidaan muodostaa joko osittain tai suoraan testitapauksiksi.
  Hyväksymistestaukseen usein osallistuu ohjelmistokehittäjien lisäksi myös muut sidosryhmät ja loppukäyttäjät.
  Keskeistä on, että loppukäyttäjiltä hankitaan tieto tarvittavista ja toteutettavista ominaisuuksista, kun taas muut sidosryhmät kuten esimerkiksi johtoryhmä voivat tehdä liiketoiminnallisia päätöksiä hyväksymistestauksen onnistumisen osalta ja esimerkiksi peruuttaa julkaisun.
  Hyväksymistestaus antaa mahdollisuuden korjata usein liiketoiminnalisestakin näkökulmasta merkittävät toiminalliset virheet ennen järjestelmän julkaisua loppukäyttäjille.

  Kehittäjien käsitys järjestelmän toiminnallisuudesta ja sen vaatimuksista voi olla usein hyvinkin erilainen kuin loppukäyttäjien.
  Hyväksymistestauksen avulla voidaan tätä lievittää tätä ongelmaa, ja saattaa ohjelmistokehittäjät loppukäyttäjien kanssa vaatimusmäärittelyn suhteen samalle sivulle.
  Testiautomaation avulla toteutettavalla toistuvalla hyväksymistestauksella varmistetaan, että järjestelmä toteuttaa loppukäyttäjän tarpeet vielä järjestlemään tehtyjen muutoksien jälkeenkin.
  Hyväksysmistestauksen testitapaukset tarkoituksenmukaisesti heijastavat suoraan loppykäyttäjien tarpeita, joka on iso etu sillä sen avulla ohjelmistokehittäjät ja muut sidosryhmät voivat tehokkaasti varmistaa järjestlemän valmiuden ja tilan.
  Hyväksymistestauksella siis saadaan katsaus ohjelmiston valmiudesta sen vaatimuksiin ja loppukäyttäjien toiminnallisiin tarpeisiin nähden.

\section{Hyväksymistestausvetoinen kehitys} \label{ch:08_hyvaksymistestausvetoinen_kehitys}

  Hyväksymistestausvetoisen kehityksen (englanniksi: ATDD, acceptance test driven development) tarkoituksena, kuten testausvetoisessakin kehityksessä \ref{ch:07_testausvetoinen_kehitys} on toteuttaa ohjelmistotuotannollinen prosessi laatien toistettavasti suoritettavat testitapaukset ennen ohjelmiston varsinaista toteutusta.
  Hyväksymistestausvetoisessa kehityksessä tämä tarkoittaa käytännössä sitä, että ennen toteutusta luodaan tarvittavat ohjelmiston asiakasvaatimuksia palvelevat hyväksymistestit, jotka ohjelmiston on tarkoitus läpäistä sen julkaisemisen hyväksymiseksi.
  Hyväksymistestausvetoisen kehityksen sanotaan olevan yhteistyöhön perustuva lähestymistapa kehitykseen, jossa tiimi ja asiakkaat käyttävät asiakkaiden oman ympäristön kieltä ymmärtääkseen heidän vaatimukset, jotka muodostavat pohjan komponentin tai järjestelmän testaamiseen \parencite{istqb_glossary_nodate}.
  Tarvittavat ohjelmiston hyväksymistestit suoritetaan iteratiivisesti ohjelmistokehitysprosessin aikana ja se tarkoittaa käytännössä jatkuvan integraation \ref{ch:07_jatkuva_integrointi} ottamista käyttöön ohjelmistokehityksessä.
  Hyväksysmistestausvetoinen kehitys on erittäin hyödyllinen ohjelmistokehityksessä käytetty menetelmä, sillä kehitysvaiheessa on aina tarkasti tiedossa vastaako ohjelmiston tila asiakasvaatimuksia ja kuinka hyvin se niiden täyttämisessä onnistuu.

  \begin{figure}[H]
    \centering
    \includegraphics[width=0.8\textwidth]{assets/hyvaksymistestausvetoinen-kehitys.png}
    \caption{Hyväksymistestausvetoisen kehityksen vaiheet}
    \label{fig:hyvaksymistestausvetoinen-kehitys}
  \end{figure}

  Hyväksymistestausvetoinen kehitys voidaan luokitella ketteräksi ohjelmistokehitysmenetelmäksi, kuten sen yläkäsitteenä oleva testausvetoinen kehityskin \ref{ch:07_testausvetoinen_kehitys}.
  Hyväksymistestausvetoinen kehitys on testausvetoisen kehityksen kanssa perusperiaatteeltaan samanlainen, mutta ennen ohjelmistokehityksen aloitusta asiakasvaatimukset kartoitetaan ja ohjelmiston hyväksyttävyys määritetään.
  Hyväksymistestitapaukset kirjoitetaan testausvetoisen kehityksen mukaisesti ensin ja ohjelmistokehitys itsessään noudattaa iteratiivisesti testausvetoista kehitystä, vaikkakin hyväksymistestaus itsessään on perinteisesti vaatinut lähes valmista järjestlemää.
  Asiakasvaatimukset määritetään usein käyttötapauksien muotoon, ja riippuen testialustasta ne voidaan kirjottaa testitapauksien muotoon niitä vahvasti hyödyntäen.
  Hyväksymistestausvetoisessa kehityksessä ohjelmistokehitystä siis ohjaavat asiakasvaatimukset ja loppukäyttäjien tarpeiden toteutuminen, jotka ovat hyvin usein toiminallisia vaatimuksia.
  Hyväksymistestausvetoisessa kehityksessä mitataan jatkuvasti iteroiden käyttötapauksien muodossa validoitavien haluttujen ominaisuuksien toteutumista.
  Perusperiaate on kirjoittaa asiakasvaatimus tai käyttötapaus testitapauksen muotoon, toteuttaa testitapaus, ajaa testitapaus läpäisemättömänä, toteuttaa ominaisuus, ajaa testitapaus läpäisevänä, refaktoroida toteutus ja siirtyä takaisin seuraavaan käyttötapaukseen.
  Esimerkki käyttötapauksesta voi olla: \emph{käyttäjänä, haluan voida avata premium ominaisuudet tekemällä sovelluksensisäisen oston}.

  Hyväksymistestausvetoisessa kehityksessä hyväksymistestit on hyödyllistä pilkkoa pieniin hallittaviin kokonaisuuksiin, jolloin voidaan iteratiivisesti toteuttaa valmiiksi tietyn testitapauksen mukainen ominaisuus, joka vastaa jotakin käyttötapausta tai loppukäyttäjän tarvetta.
  Hyväksymistestauksessa testitapaus voi olla esimerkiksi käyttäjän tietojen muuttuminen varmistaminen, kuten tason läpäiseminen pelisovelluksessa, joka muuttaa käyttäjän edistystä.
  Hyväksymistestausvetoisen kehityksen tarkoituksena menetelmänä on onnistua vastaamaan loppykäyttäjän tarpeisiin tehokkaasti ja hyvin ottamalla tarpeet huomioon jo ennen toteutuksen aloittamista.
  Menetelmän avulla myös luodaan ymmärrystä ohjelmistotuotteen valmiuden määritelmästä kun eri sidosryhmän voidaan saada sen suhteen samalle aaltopituudelle.
  Hyväksymistestausvetoinen kehitys on lisäksi erittäin hyödyllistä, sillä jatkuva testaaminen antaa mahdollisuuden haluttujen ominaisuuksien toteutumisen validoimiselle menetelmän jokaisen iteraation koontiversiossa.

\section{Robot Framework} \label{ch:08_robot_framework}

  Robot framework on geneerinen avoimen lähdekoodin testausalusta hyväksymistestaukseen, hyväksymistestausvetoiseen kehitykseen ja robotisten prosessien automaatioon.

  \begin{figure}[H]
    \centering
    \includegraphics[width=0.4\textwidth]{assets/robot-arkkitehtuuri.png}
    \caption{Robot framework alustan arkkitehtuuri}
    \label{fig:robot-architecture}
  \end{figure}

  \begin{itemize}
    \item Helposti ymmärrettävä, luettava ja selkeä avainsanaperustainen syntaksi
    \item Etuna helposti lähestyttävyys. Helppo asentaa, ymmärtää ja ottaa käyttöön.
    \item Testikehystä voi helppouden ja avainsanaperustaisuuden vuoksi käyttää muutkin kuin sovelluskehittäjät.
    \item Tuki ulkoisille kirjastoille ja useita käyttövalmiita ulkoisia kirjastoja
    \item Tukee muuttujia testitapauksissa, joilla voi lisätä hieman kompleksisuutta testitapauksiin
    \item Tukee dataperustaisia testitapauksia, joille annetaan eri syötteitä sisältävää testidataa
    \item Testitapauksia voidaan ryhmitellä tageillä.
    \item Kattavat ja selkeät testiraportit ajetuille testitapauksille.
    \item Heikkoutena tuen puuttuminen ohjelmointikieliperustaisissa testikehyksissä löytyville kontrollirakenteille, joita esiintyy esimerkiksi yksikkötestaukseen tarkoitetuissa testikehyksissä.
    \item käyttötapaus: 1) tilanne, 2) motivaatio, 3) haluttu lopputulos (esim. Kun tämä, niin haluan tätä, jotta saavutan tämän)
  \end{itemize}

\section{Testitapauksien määrittäminen} \label{ch:08_testitapauksien_maarittaminen}

  \begin{itemize}
    \item Testitapaus on testiautomaation näkökulmasta, määritelty toimenpiteiden, ehtojen ja muuttujien joukko, joka suorittamalla voidaan verifioida ominaisuus tai toiminnallisuus ohjelmistosta.
    \item Testisuite tai testikokoelma on samaan kontekstiin kuuluvista testitapauksista muodostettu joukko.
    \item Testitapaukset kirjoitetaan hyväksymistestauksen mukaisesti käyttötapauksien muodossa.
    \item Testiformaatti: 1) Oletetaan tämä ja tätä (setup) => 2) Kun tämä tapahtuu (trigger) => 3) Niin tämä seuraa (verification) (todo: tee tästä kuva)
    \item Yleisiä tavoitteita: yksinkertaisuus, läpinäkyvyys, käyttäjätietoisuus, epätoistuvuus, olettamattomuus, kattavuus, tunnistettavuus, jälkensä puhdistava, toistettava, syvyyttömyys, atomisuus.
    \item Taulukon/listan muodossa esimerkkejä käyttöliittymien testitapauksista.
    \item Robot framework: avainsana, käyttäytyminen tai data-pohjainen kirjoitustyyli testitapauksille
    % \item <todo: lisää kuva/koodi esimerkkitestitapauksessa robot frameworkillä>
    \item https://github.com/robotframework/HowToWriteGoodTestCases/blob/master/HowToWriteGoodTestCases.rst
  \end{itemize}

\section{Web-käyttöliittymien erityispiirteet} \label{ch:08_webkayttoliittymien_erityispiirteet}

  Web-käyttöliittymillä on myös omia erityispiirteitä, jotka vaikuttavat testitapauksien laatimiseen.

  \begin{itemize}
    \item Käyttöliittymä ja DOM
    \item Hosting
    \item Näyttöresoluutiot
    \item Navigointi
    \item Syötteet
    \item Syntaksi
    \item Selainasetukset
    \item Moniselaimellinen testaus
    \item Päätteetön testaus
    \item Selenium
  \end{itemize}

\section{Priorisointiongelma} \label{ch:08_priorisointiongelma}

  Testitapauksien priorisointi on kustannussyistä tai resurssien optimoinnin kannalta erittäin tärkeää.
  Ohjelmistotestauksessa on hyvä tiedostaa, että ohjelmistotuotetta ei usein voida testata täydellisesti, joka nostaa esiin tarpeen tärkeimpien testitapauksien löytämisestä.
  Testitapauksia voidaan priorisoida monella tavalla, joihin tämä diplomityö tuo yhden uudenlaisen painottua verkkoa hyödyntävän lähestymistavan.

  \begin{itemize}
    \item Painotetun verkon hyödyntäminen
    \item Muut priorisointitavat
    \item heuristinen priorisointi
    \item moscow menetelmä
  \end{itemize}