Tässä luvussa esitetään perusteet ja tarvittavat tiedot hyväksymistestauksesta..

\section{Testiautomaatio prosessina}

% Testiautomaation prosessiin kuuluu erilaisia artifakteja, joita luodaan testausprosessin eri vaiheissa.
% Eri vaiheita ovat kronologisessa järjestyksessä ovat muun muassa testisuunnitelma, skenaariot, testitapaukset ja seuranta.

\section{Testausvetoinen kehitys}

% Teksti tähän

\section{Hyväksymistestausvetoinen kehitys}

Hyväksymistestausvetoisen kehityksen (englanniksi: ATDD, acceptance test driven development) tarkoituksena, kuten testausvetoisessakin kehityksessä on toteuttaa ohjelmistotuotannollinen prosessi testaaminen edellä.
Tämä tarkoittaa käytännössä, sitä että ohjelmistokehittäjät laativat ohjelmiston vaatimusten ja suunnitelman mukaisia iteratiivisesti suoritettavia testitapauksia, ennen niitä käyttävän varsinaisen ohjelmakoodin toteuttamista.
Hyväksymistestausvetoisessa kehityksessä luodaan ennen toteutusta tarvittavat ohjelmiston asiakasvaatimuksia palvelevat hyväksymistestit, joiden ohjelmiston on tarkoitus läpäistä.
Tarvittavat ohjelmiston hyväksymistestit suoritetaan iteratiivisesti ohjelmistokehitysprosessin aikana, ja se tarkoittaa käytännössä jatkuvan integraation ottamista käyttöön ohjelmistokehityksessä.
Hyväksysmistestausvetoinen kehitys on erittäin hyödyllinen ohjelmistokehityksessä käytetty menetelmä, sillä kehitysvaiheessa on aina tarkasti tiedossa vastaako ohjelmiston silloinen tila asiakasvaatimuksia ja kuinka hyvin.
Hyväksymistestausvetoisessa kehityksessä toteutettavat hyväksymistestit testaavat ohjelmistoa kokonaisena järjestelmänä tarkoituksenamukaisesti, siten kuten se esiintyy loppukäyttäjille.

\section{Testiautomaatio ja jatkuva integraatio}

% Teksti tähän