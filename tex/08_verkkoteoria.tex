Tässä luvussa käsitellään työhön keskeisesti kuuluvan verkkoteorian perusteet käydää huolellisesti läpi erityisesti työssä käytettävät osat.
Työssä sovelletaan erityisesti verkkoteorian painotettua verkkoa sekä verkkoteoriassa esiintyvän lyhimmän polun ongelmaan kehitettyjä ratkaisualgoritmeja.
Verkkoteoria itsessään on osa diskeettiä matematiikkaa.

\section{Matemaattisten verkkojen tarkoitus}

Matemaattisten verkkojen tarkoituksena on mallintaan parittaisia riippuvuuksia verkkomaisessa objektijoukossa.
Verkkoteoriassa peruskäsitteitä ovat itse \textit{verkko} eli \textit{graafi}, joka muodostuu \textit{solmuista} ja niiden välisiä riippuvuuksia esittävistä \textit{kaarista} tai \textit{nuolista}.
Verkkoteorialla on lukuisia käytännön sovellutuksia. Verkkoteoriaa sovelletaan muun muassa tietokonetieteissä, kielitieteissä, fysiikan ja kemian sovellutuksissa, sosiaalisissa tieteissä ja biologiassa.
Alun perin verkkoteoria katsotaan syntyneen 1700-luvulla esiintyneestä niin sanotusta Königsbergin siltaongelmasta, johon Leonhard Euler esitti todistuksensa.

\section{Suuntaamaton verkko}

<Lisää teksti tähän>

\section{Suunnattu verkko}

<Lisää teksti tähän>

\section{Painotettu verkko}

<Lisää teksti tähän>

\section{Ketjut ja polut}

<Lisää teksti tähän>

\section{Lyhimmän polun ongelma}

<Lisää teksti tähän>

  \subsection{Dijikstran algoritmi}

  <Lisää teksti tähän>

  \subsection{Bellman–Ford algoritmi}

  <Lisää teksti tähän>

  \subsection{A* algoritmi}

  <Lisää teksti tähän>
