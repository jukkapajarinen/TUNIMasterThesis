Tässä luvussa käsitellään työhön keskeisesti kuuluvan verkkoteorian perusteet käydää huolellisesti läpi erityisesti työssä käytettävät osat.
Työssä sovelletaan erityisesti verkkoteorian painotettua verkkoa sekä verkkoteoriassa esiintyvän lyhimmän polun ongelmaan kehitettyjä ratkaisualgoritmeja.
Verkkoteoria itsessään on osa diskeettiä matematiikkaa.

\section{Matemaattisten verkkojen tarkoitus}

Matemaattisten verkkojen tarkoituksena on mallintaan parittaisia riippuvuuksia verkkomaisessa objektijoukossa.
Verkkoteoriassa peruskäsitteitä ovat itse \textit{verkko} eli \textit{graafi}, joka muodostuu \textit{solmuista} ja niiden välisiä riippuvuuksia esittävistä \textit{kaarista} tai \textit{nuolista}.
Verkkoteorialla on lukuisia käytännön sovellutuksia. Verkkoteoriaa sovelletaan muun muassa tietokonetieteissä, kielitieteissä, fysiikan ja kemian sovellutuksissa, sosiaalisissa tieteissä ja biologiassa.
Alun perin verkkoteoria katsotaan syntyneen 1700-luvulla esiintyneestä niin sanotusta Königsbergin siltaongelmasta, johon Leonhard Euler esitti todistuksensa.

\section{Perusmerkinnät ja käsitteet}

Verkkoteoriassa käytetään seuraavia perusmerkintöjä:
\begin{itemize}
  \item \(V := \{v_1, v_2, v_3\}\) Solmujoukko joka sisältää solmut \(v_1\), \(v_2\) ja \(v_3\).
  \item \(E := \{e_1, e_2, e_3\}\) Kaarijoukko joka sisältää kaaret \(e_1\), \(e_2\) ja \(e_3\).
  \item \(\phi(e_1) := \langle v_1, v_2 \rangle\) Solmuja \(v_1\) ja \(v_2\) yhdistävän kaaren \(e_1\) kuvaaja.
\end{itemize}

Verkkojen solmujen välisiä yhteyksiä, eli kaaria esitetään usein myös yhteys- tai painomatriisina.
\[
  M_G = (a_{ij})_{3\times3} =
  \bordermatrix{
    G & v_1 & v_2 & v_3 \cr
    v_1 & 1 & 1 & 2 \cr
    v_2 & 1 & 0 & 0 \cr
    v_3 & 2 & 0 & 0 \cr
  }
\]

Verkkoteoriassa on käytetään myös seuraavia käsitteitä:
\begin{itemize}
  \item Solmujen vierekkäisyys
  \item Eristetty solmu
  \item Surkastunut verkko
  \item Äärellinen verkko
  \item Täydellinen verkko
  \item Yksinkertainen verkko
  \item Aliverkko
  \item Verkon komplementti
  \item Verkon diagonaali
  \item Verkon yhtenäisyys
  \item Silta
  \item Verkon isomorfisuus
\end{itemize}

\section{Suuntaamaton ja suunnattu verkko}

\begin{itemize}
  \item Kaarien rinnakkaisuus
  \item Kaarien vierekkäisyys
  \item Kaari on silmukka
  \item Ketju
  \item Solmun asteluku
  \item Polku
  \item Lähtösolmu
  \item Maalisolmu
  \item Päätesolmu
  \item Vahva rinnakkaisuus
  \item Vastaikkaisuus
  \item Lähtöaste
  \item Maaliaste
  \item Nuoli
  \item Suljettu polku
\end{itemize}

\section{Syklinen ja asyklinen verkko}

<Lisää teksti tähän>

\section{Painotettu verkko}

\begin{itemize}
  \item \(\alpha := E(G) \rightarrow \mathbb{N}\) Painofunktion yleinen kuvaus.
\end{itemize}

\section{Verkon leikkaaminen}

<Lisää teksti tähän>

\section{Lyhimmän polun ongelma}

\begin{itemize}
  \item \(d_G^\alpha(v_1, v_2) = min\{\alpha(P) | P:v_1 \rightarrow v_2 | v_1, v_2 \in V(G)\}\) Lyhimmän polun ongelma.
\end{itemize}

  \subsection{Dijkstran algoritmi}

  <Lisää teksti tähän>
