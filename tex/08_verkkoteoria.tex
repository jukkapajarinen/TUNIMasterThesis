Tässä luvussa käsitellään työhön keskeisesti kuuluvan verkkoteorian perusteet käydää huolellisesti läpi erityisesti työssä käytettävät osat.
Työssä sovelletaan erityisesti verkkoteorian painotettua verkkoa sekä verkkoteoriassa esiintyvän lyhimmän polun ongelmaan kehitettyjä ratkaisualgoritmeja.
Verkkoteoria itsessään on osa diskeettiä matematiikkaa.

\section{Matemaattisten verkkojen tarkoitus}

Matemaattisten verkkojen tarkoituksena on mallintaan parittaisia riippuvuuksia verkkomaisessa objektijoukossa.
Verkkoteoriassa peruskäsitteitä ovat itse \textit{verkko} eli \textit{graafi}, joka muodostuu \textit{solmuista} ja niiden välisiä riippuvuuksia esittävistä \textit{kaarista} tai \textit{nuolista}.
Verkkoteorialla on lukuisia käytännön sovellutuksia. Verkkoteoriaa sovelletaan muun muassa tietokonetieteissä, kielitieteissä, fysiikan ja kemian sovellutuksissa, sosiaalisissa tieteissä ja biologiassa.
Alun perin verkkoteoria katsotaan syntyneen 1700-luvulla esiintyneestä niin sanotusta Königsbergin siltaongelmasta, johon Leonhard Euler esitti todistuksensa.

\section{Perusmerkinnät ja käsitteet}

Verkkoteoriassa käytetään seuraavia perusmerkintöjä:
\begin{itemize}
  \item \(V := \{v_1, v_2, v_3\}\) Solmujoukko joka sisältää solmut \(v_1\), \(v_2\) ja \(v_3\).
  \item \(E := \{e_1, e_2, e_3\}\) Kaarijoukko joka sisältää kaaret \(e_1\), \(e_2\) ja \(e_3\).
  \item \(\phi(e_1) := \langle v_1, v_2 \rangle\) Solmuja \(v_1\) ja \(v_2\) yhdistävän kaaren \(e_1\) kuvaaja.
\end{itemize}

Verkkojen solmujen välisiä yhteyksiä, eli kaaria esitetään usein myös yhteys- tai painomatriisina.
\[
  M_G = (a_{ij})_{3\times3} =
  \bordermatrix{
    G & v_1 & v_2 & v_3 \cr
    v_1 & 1 & 1 & 2 \cr
    v_2 & 1 & 0 & 0 \cr
    v_3 & 2 & 0 & 0 \cr
  }
\]

Verkkoteoriassa käytetään myös muun muassa seuraavia käsitteitä:
\begin{itemize}
  \item Solmun asteluku, \(d_G(x)\), eli solmuun liittyvien \emph{kaarten} määrä.
  \item Surkastunut verkko, \(E = \emptyset\), eli verkko jossa ei ole \emph{kaaria}.
  \item Äärellinen verkko, \(V = \{1,2,3,...,n\}, E = \{1,2,3,...,n\} \), eli \emph{solmujoukko} ja \emph{kaarijoukko} ovat äärellisiä.
  \item Täydellinen verkko, \(G\), eli jokaista solmuparia yhdistää ainakin yksi kaari.
  \item Yksinkertainen verkko, \(G\), eli verkossa ei ole silmukoita eikä rinnakkaisia kaaria.
  \item Aliverkko, \(G_2 /subset G_1\), eli verkko \(G_2\) joka koostuu osasta verkon \(G_1\) solmuja ja kaaria.
  \item Verkon komplementti, \(G'\), eli verkko jossa on kaikki ne kaaret joita verkossa \(G\) ei esiinny.
  \item Verkon yhtenäisyys, \(v_1 \neq v_2, v_1 \rightarrow v_2\) , eli jokaiselle solmuparille \(v_1 \neq v_2\) on olemassa niitä yhdistävä kaari.
\end{itemize}

\section{Suuntaamaton ja suunnattu verkko}

\begin{itemize}
  \item Lähtösolmu
  \item Maalisolmu
  \item Päätesolmu
  \item Eristetty solmu
  \item Solmujen vierekkäisyys
  \item Kaarien rinnakkaisuus
  \item Kaarien vierekkäisyys
  \item Silmukka
  \item Ketju ja polku
  \item Nuoli
  \item Silta
  \item Vahva rinnakkaisuus
  \item Vastaikkaisuus
  \item Suljettu polku
\end{itemize}

\section{Painotettu verkko}

\begin{itemize}
  \item \(\alpha := V(G), E(G) \rightarrow \mathbb{N}\) Painofunktion yleinen kuvaus.
\end{itemize}

\section{Verkon leikkaaminen}

\section{Lyhimmän polun ongelma}

\begin{itemize}
  \item \(d_G^\alpha(v_1, v_2) = min\{\alpha(P) | P:v_1 \rightarrow v_2 | v_1, v_2 \in V(G)\}\) Lyhimmän polun ongelma.
\end{itemize}

  \subsection{Dijkstran algoritmi}
