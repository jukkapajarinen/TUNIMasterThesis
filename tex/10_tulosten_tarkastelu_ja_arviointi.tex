Tässä kappaleessa esitetään yhteenveto tutkimuksen tuloksista ja evaluoidaan testausjärjestelmän ja priorisointimenetelmän toteutuksien onnistumista.
Ensin evaluoidaan diplomityössä suunnitellun ja asiakasyritykselle toteutetun testausjärjestelmän positiivisia sekä negatiivisia puolia.
Seuraavaksi evaluoidaan diplomityössä kehitetyn priorisointimenetelmän positiivisia ja negatiivisia puolia ja muun muassa pohditaan kuinka hyvin soveltuva ja toistettavissa oleva kyseinen kehitetty priorisointimenetelmä on.
Lisäksi lopussa vielä esitetään toteutuksen jälkeen esiin tulleita jatkokehitysehdotuksia testausjärjestelmälle sekä priorisointimenetelmälle.

\section{Tutkimuksen konkreettiset tulokset} \label{ch:12_tutkimuksen_konkreettiset_tulokset}

  Työn tuloksena kehitetty  web-käyttöliittymien hyväksymistestauksen automatisoimisen mahdollistava testausjärjestelmä on integroitu onnistuneesti osaksi GoCD-palvelimen avulla suoritettavaa jatkuvaa integrointia.
  Lyhyesti sanottuna testausjärjestelmän osalta konkreettinen tulos koostuu järjestelmästä, joka mahdollistaa testitapauksien luomisen Robot Framework -alustalle käyttäen Selenium-kirjastoa, Xvfb-virtualisointipalvelinta ja Docker-säiliöintiohjelmistoa.
  Testausjärjestelmän toimivuus käytännössä todettiin esimerkkitestitapauksien muodossa oikeassa ympäristössään ja testausjärjestelmän mahdollistamat ominaisuudet ovat jo itsessään oikeassa ja lopullisessa käyttöympäristössä tarvittavia.

  Web-sovelluksien näkymä ja siirtymäperustainen painotettua verkkoa hyödyntävä priorisointimenetelmä on myös todettu toimivaksi lähestysmistavaksi priorisointiin.
  Lyhyesti sanottuna priorisointimenetelmän tuloksena on painotettuja verkkoja hyödyntävä menetelmä, jossa määritetään priorisointiin vaikuttavat muuttujat, painofunktiot, painomatriisi, prioriteetteihin perustuvien leikkauksien tekeminen ja prioriteettien löytäminen sekä lukeminen verkosta.
  Priorisointimenetelmän toimivuus käytännössä todettiin aidosta ympäristöstä yksinkertaistaen poimitusta web-sovelluksesta.
  Priorisointimenetelmän avulla todettiin, että priorisoinnin aikana toteutetut painotetun verkon leikkaukset olivat juuri niitä näkymiä, jotka vaistonvaraisesti ilman menetelmänkin käyttöä karsittaisiin.

\section{Toteutuksen evaluointi} \label{ch:12_toteutuksen_evaluointi}

  Kokonaisuutena hyväksymistestausjärjestelmän toteutus onnistui erittäin hyvin ja sen avulla on mahdollista jopa geneerisesti rakentaa web-sovelluksesta riippumattomasti hyväksymistestaus testauskohteena olevalle web-sovellukselle.

  Testausjärjestelmän positiivisia puolia ovat muun muassa Docker-säiliöinnin avulla saatava tuki myös manuaaliselle testitapauksien ajamiselle.
  Docker-säiliö, joka mahdollistaa testitapauksien ajamisen voidaan pystyttää periaatteessa mihin tahansa ympäristöön, jossa Docker on saatavilla.
  Docker-säiliöinnin avulla myös ohjelmistokehittäjät saavat valmiin hyväksymistestausjärjestelmän helposti käyttöönsä.
  Docker-säiliöinnin ja Docker-compose:n avulla rakennettu järjestelmä ei myöskään ole sidottu mihinkään ennalta määritettyyn jatkuvan integroinnin palvelimeen, joka huomattavasti helpottaa testausjärjestelmän käyttöönottoa osaksi uusia tai muuttuvaa ohjelmistotuotannon prosessia.
  Testausjärjestelmä mahdollistaa päätteettömän testauksen virtuaalisen Xfvb-näyttöpalvelimen avulla, joka on itsessään erittäin tarvittu ominaisuus jatkuvan integroinnin ja testausjärjestelmän yhdistämiseen.
  Xvfb-näyttöpalvelimen tarjoaman virtualisoinnin avulla voidaan myös uusia WebDriver rajapinnan toteuttavia verkkoselaimia lisätä päätteettömän testauksen alaisuuteen erittäin helposti ja käytännössä rajoituksitta.
  Ainoa vaatimus on, että verkkoselain on saatavilla siihen ympäristöön, jossa Xvfb-näyttöpalvelinta ajetaan.

  Testausjärjestelmässä on kuitenkin myös negatiivisia puolia, joiden osalta järjestelmän käyttö on rajattua.
  Xvfb-näyttöpalvelimen avulla voidaan päätteettömästi testata periaatteessa mitä tahansa GUI-ohjelmia, mutta rajoitteena on kuitenkin, että niiden täytyy olla saatavilla siihen ympäristöön, jossa Xvfb-näyttöpalvelinta ajetaan.
  Tämä tarkoittaa käytännössä sitä, että web-sovelluksien hyväksymistestauksen automatisoimisesta on jätettävä pois vain Window-ympäristöön saatavien verkkoselainten, kuten Internet Explorer verkkoselaimen testaaminen.
  Robot Framework takaa helpon testitapauksien luettavuuden kenelle tahansa, mutta ohjelmistokehittäjille se voi olla turhan rajatun tuntuinen.
  Ohjelmistokehittäjänä testitapauksien laatimisen yhteyteen olisi hyvä saada mahdollisuus yksikkötestauskehyksissä käytettävistä ohjelmointikielistä tuttuihin kontrollirakenteisiin, joilla testitapauksien monipuolisuutta voisi kasvattaa perinteisesti yksikkötestauksessa mahdollisien rakenteiden tasolle.
  Tämä ei kuitenkaan ole mahdollista Robot Framework:issä, jossa testitapauksien laatimiseen käytetään Robot Framework:in omaa, rajattua syntaksia.

\section{Menetelmän evaluointi} \label{ch:12_menetelman_evaluointi}

  Tässä diplomityössä kehitetyn testitapauksien priorisointimenetelmän kehittäminen onnistui myös erittäin hyvin ja sen käyttämisellä saavutetaan lisäarvoa etenkin keskisuurien ja suurien web-sovelluksien käyttöliittymien hyväksymistestaukseen.

  Priorisointimenetelmän positiivisia puolia ovat muun muassa sen ominaisuuksiin liittyviä asioista kuten priorisointimenetelmän toistettavuus ja mahdollisuus priorisoida käyttöliittymien näkymiä ja siirtymiä.
  Näkymä ja siirtymäperustaisen priorisoinnin tarkoituksena on mahdollistaa näkymiin perustuvien testikokoelmien priorisoiminen, jolloin niiden tärkeysjärjestys saadaan selville ja testitapauksien kirjoittaminen voidaan aloittaa prioriteetiltaan tärkeimmästä näkymästä.
  Menetelmän käyttäminen on tehokkainta kun testitapaukset kategorisoidaan näkymittäin laadittuihin testikokoelmiin, sillä menetelmän kehittämisen taustalla on ollut ajatus jossa näkymät vastaavat testikokoelmia.
  Priorisointimenetelmä perustuu matemaattisiin painotettuihin verkkoihin, jotka tuovat hyötynä lyhimmän polun ongelman ratkaisemiseen kehitettyjen algoritmien käyttämisen mahdollistamisen prioriteetiltaan korkeimpien polkujen löytämiseen kahden solmun, eli näkymän välille.
  Lisäksi painotetun verkon ja matemaattisen lähestymistavan käyttäminen tuo hyötynä sen, että menetelmä on kohtalaisen pienellä vaivalla muunnettavissa tietokoneohjelmaksi.
  Painotettujen verkkojen käyttäminen priorisointiin pakottaa myös menetelmän käyttäjät piirtämään näkymä ja siirtymäperustaisen painotetun verkon, jolloin se kasvattaa käyttäjien ymmärrystä testauskohteena olevasta järjestelmästä.
  Priorisointimenetelmä on tässä diplomityössä esitetyn esimerkin \ref{tab:esimerkki_verkon_priorisointi_muuttujat} mukaan todettavissa toimivaksi ja sen avulla on suoritettu priorisointi varsinaisesta testauskohteesta yksinkertaistetulle käyttöliittymälle.

  Myös priorisointimenetelmässä on negatiivisiakin puolia, mutta ne eivät tässäkään tapauksessa ylitä menetelmän käytöstä saatavaa hyötyä.
  Menetelmässä esittävien toistuvien leikkauksien määräää rajaavan testikattavuuden päättäminen näkymä ja siirtymäperusteisesti voi olla haastavaa.
  Toinen menetelmään kohdistuva kritiikki koskee menetelmän geneerisyyttä, eli käyttöönottamisen mahdollisuutta ilman muutoksia, jota on vaikea arvioida.
  Priorisointimenetelmässä käytettävät priorintiin vaikuttavat muuttujat ovat varsin subjektiivisia ja voivat olla testausta toteuttavan tahon mukaan muuttuvia, jonka takia muuttujiin joudutaan mahdollisesti tekemään muutoksia.
  Negatiivista on myös se että menetelmän käyttö on soveltuva painotettujen verkkojen luonteen mukaisesti käyttöliittymien tapauksessa soveltuvat vain kokonaisia näkymiä peilaavien testikokoelmien priorisointiin yksittäisten testitapauksien sijaan.
  Priorisointimenetelmän käyttö voi olla turhan aikaa vievää jos käyttöliittymä on yksinkertainen.

\section{Jatkokehitysehdotukset} \label{ch:12_jatkokehitysehdotukset}

  Tämän diplomityön konkreettisina tuloksina syntyneet testausjärjestelmä ja priorisointimenetelmä ovat sellaisenaan käyttövalmiita ja toimivaksi todettuja, mutta jatkokehittelylle on luonnollisesti niissäkin sijaa.
  Testausjärjestelmän avulla toteuttava web-sovelluksien päätön hyväksymistestaus mahdollisestaan Xvfb-näyttöpalvelimen tarjoaman virtualisoinnin avulla.
  Xvfb-näyttöpalvelin on kuitenkin saatavilla vain UNIX-ympäristöihin, joka rajaa testausjärjestelmään lisättävien verkkoselaimien saatavuutta.
  Xvfb-näyttöpalvelimelle voitaisiin jatkokehityksenä etsiä monialustaisempi vaihtoehto tai ainakin vastine Window-ympäristöön, jonka avulla myös vain Window-alustalle saatavat verkkoselaimet olisi mahdollista lisätä järjestelmään.

  Yksi priorisointimenetelmään liittyvä rajoite on käyttöliittymän näkymä ja siirtymäperustainen priorisointi, joka asettaa näkymät vastaamaan testikokoelmia.
  Jatkokehityksenä voitaisiin tutkia näkymäperusteisuuden mukaan tehtävän priorisoinnin muuntamisen mahdollisuutta käyttötapausperustaiseksi.
  Käyttötapausperustaisesti luotava verkko parhaimmillaan vastaisi oikeita käyttäjien tarpeisiin tarkoitettuja toiminallisuuksia ja voisi parantaa priorisointia.
  Priorisointimenetelmä on vahvasti matemaattinen, joka mahdollistaa sen muuntamisen kohtalaisella vaivalla tietokoneohjelmaksi.
  Priorisointimenetelmän rakentaminen automaattisen tietokoneohjelman muotoon olisi erittäin järkevää ja laskisi menetelmän käyttööottamiseen tarvittavaa vaivannäköä huomattavasti.
  Lisäksi priorisointimenetelmän näkymäpohjaisten graafien, eli painotettujen verkkojen visualisointi voitaisiin hoitaa tietokoneohjelman yhteydessä.
