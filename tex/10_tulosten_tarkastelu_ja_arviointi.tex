Tässä kappaleessa esitetään yhteenveto tutkimuksen tuloksista ja muun muassa pohditaan kuinka hyvin soveltuva ja toistettavissa oleva kyseinen kehitetty menetelmä on.

\section{Tutkimuksen konkreettiset tulokset} \label{ch:12_tutkimuksen_konkreettiset_tulokset}

  \begin{itemize}
    \item Web-käyttöliittymien hyväksymistestauksen automatisoimisen mahdollistava järjestelmä
    \item Näkymäperustainen painotettua verkkoa hyödyntävä toistettvissa oleva priorisointimenetelmä
    % \item Ohjelma, joka tekee menetelmää noudattaen priorisoinnin painomatriisin ja kattavuuden perusteella.
  \end{itemize}

\section{Menetelmän evaluointi} \label{ch:12_menetelman_evaluointi}

  \begin{itemize}
    \item Priorisointimenetelmän toistettavuus
    \item Menetelmän avulla on mahdollista priorisoida näkymiä ja siirtymiä
    \item Dijkstran algoritmin avulla voidaan selvittää verkosta prioriteetiltaan korkein polku, eli näkymät joiden testikokoelmat tulisi suorittaa ensimmäisenä.
    \item Painotettu verkko voidaan piirtää, joka kasvattaa ymmärrystä järjestelmästä
    \item Testikattavuuden päättäminen näkymä- ja siirtymäperusteisesti voi olla haastavaa
    \item Ei välttämättä ainakaan muokkauksia tekemättä geneerinen menetelmä
    \item Menetelmän käyttö soveltuu testikokoelmien, ei testitapauksien, priorisointiin
    \item Menetelmän käyttö on tehokkainta kun testitapaukset kategorisoidaan näkymittäin testikokoelmiin
    \item Priorisointimenetelmän käyttö voi olla turhan aikaa vievää jos käyttöliittymä on yksinkertainen
  \end{itemize}

\section{Toteutuksen evaluointi} \label{ch:12_toteutuksen_evaluointi}

  \begin{itemize}
    \item Docker säiliönnin avulla tuki myös manuaaliselle testitapauksien ajamiselle
    \item Docker säiliönnin avulla sovelluskehittäjät saavat valmiin hyväksymistestausjärjestelmän helposti käyttöönsä
    \item Docker säiliöinnin takia toteutus ei ole sidottu CI palvelimeen
    \item Xvfb virtualisoinnin avulla voidaan uusia verkkoselaimia lisätä helposti
    \item Xvfb virtuallisoinnin avulla ei voida testata vain Windows ympäristöön saatavia GUI-ohjelmia
    \item Robot framework takaa helpon luettavuuden kenelle tahansa, mutta ohjelmistokehittäjälle rajatun tuntuinen
    \item Hyvä skaalautuvuus, docker säilöitä on helppo rakentaa ja GoCD agentteja lisätä
  \end{itemize}

\section{Jatkokehitysehdotukset} \label{ch:12_jatkokehitysehdotukset}

  \begin{itemize}
    \item Xvfb:lle vastineen löytäminen Window ympäristöön
    \item Parempi priorisointi muuttamalla näkymäperusteisuus käyttötapausperustaiseksi
    \item Näkymägraafien visualisointi ja muu priorisointiohjelman jatkokehittely
  \end{itemize}