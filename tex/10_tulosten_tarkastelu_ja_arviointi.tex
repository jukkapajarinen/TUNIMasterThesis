Tässä kappaleessa esitetään yhteenveto tutkimuksen tuloksista, ja evaluoidaan testausjärjestelmän ja priorisointimenetelmän toteutuksien onnistumista.
Ensin evaluoidaan diplomityössä suunnitellun ja asiakasyritykselle toteutetun testausjärjestelmän positiivisia sekä negatiivisia puolia.
Seuraavaksi evaluoidaan diplomityössä kehitetyn priorisointimenetelmän positiivisia ja negatiivisia puolia ja pohditaan muun muassa sitä, että kuinka hyvin soveltuva ja toistettavissa oleva kyseinen kehitetty priorisointimenetelmä on.
Lopuksi vielä esitetään toteutuksen jälkeen esiin tulleita jatkokehitysehdotuksia testausjärjestelmälle sekä priorisointimenetelmälle.

\section{Tutkimuksen konkreettiset tulokset} \label{ch:12_tutkimuksen_konkreettiset_tulokset}

  Työn tuloksena kehitetty  web-käyttöliittymien hyväksymistestauksen automatisoimisen mahdollistava testausjärjestelmä on integroitu onnistuneesti osaksi GoCD-palvelimen avulla suoritettavaa jatkuvaa integrointia.
  Lyhyesti sanottuna testausjärjestelmän osalta konkreettinen tulos koostuu järjestelmästä, joka mahdollistaa testitapauksien luomisen Robot Framework -alustalle käyttäen Selenium-kirjastoa, Xvfb-virtualisointipalvelinta ja Docker-säiliöintiohjelmistoa.
  Testausjärjestelmän toimivuus todettiin käytännössä esimerkkitestitapauksien muodossa oikeassa ympäristössään, ja testausjärjestelmän mahdollistamat ominaisuudet ovat jo itsessään oikeassa ja lopullisessa käyttöympäristössä tarvittavia.

  Web-sovelluksien näkymä- ja siirtymäperustainen, painotettua verkkoa hyödyntävä priorisointimenetelmä on myös todettu toimivaksi lähestysmistavaksi priorisointiin.
  Lyhyesti sanottuna priorisointimenetelmän tuloksena on painotettuja verkkoja hyödyntävä menetelmä, jossa määritetään priorisointiin vaikuttavat muuttujat, painofunktiot, painomatriisi, prioriteetteihin perustuvien leikkauksien tekeminen sekä prioriteettien löytäminen ja niiden lukeminen verkosta.
  Priorisointimenetelmän toimivuus käytännössä todettiin aidosta ympäristöstä yksinkertaistaen poimitusta web-sovelluksesta.
  Priorisointimenetelmän avulla todettiin, että priorisoinnin aikana toteutetut painotetun verkon leikkaukset olivat juuri niitä näkymiä, jotka vaistonvaraisesti myös ilman menetelmän käyttöä karsittaisiin.

\section{Toteutuksen evaluointi} \label{ch:12_toteutuksen_evaluointi}

  Kokonaisuutena hyväksymistestausjärjestelmän toteutus onnistui erittäin hyvin ja sen avulla on mahdollista jopa geneerisesti rakentaa web-sovelluksesta riippumattomasti hyväksymistestaus testauskohteena olevalle web-sovellukselle.

  Testausjärjestelmän positiivisia puolia ovat muun muassa Docker-säiliöinnin avulla saatava tuki myös manuaaliselle testitapauksien ajamiselle.
  Docker-säiliö, joka mahdollistaa testitapauksien ajamisen, voidaan pystyttää periaatteessa mihin tahansa ympäristöön, jossa Docker on saatavilla.
  Docker-säiliöinnin avulla ohjelmistokehittäjät myös saavat valmiin hyväksymistestausjärjestelmän helposti käyttöönsä.
  Docker-säiliöinnin ja Docker-composen avulla rakennettu järjestelmä ei myöskään ole sidottu mihinkään ennalta määritettyyn jatkuvan integroinnin palvelimeen, mikä huomattavasti helpottaa testausjärjestelmän käyttöönottoa osaksi uutta tai muuttuvaa ohjelmistotuotannon prosessia.
  Testausjärjestelmä mahdollistaa päätteettömän testauksen virtuaalisen Xfvb-näyttöpalvelimen avulla, joka on itsessään erittäin tarvittu ominaisuus jatkuvan integroinnin ja testausjärjestelmän yhdistämiseen.
  Xvfb-näyttöpalvelimen tarjoaman virtualisoinnin avulla voidaan myös lisätä uusia WebDriver-rajapinnan toteuttavia verkkoselaimia päätteettömän testauksen alaisuuteen erittäin helposti ja käytännössä rajoituksetta.
  Ainoa vaatimus on, että verkkoselain on saatavilla siihen ympäristöön, jossa Xvfb-näyttöpalvelinta ajetaan.

  Testausjärjestelmässä on kuitenkin myös negatiivisia puolia, joiden osalta järjestelmän käyttö on rajattua.
  Xvfb-näyttöpalvelimen avulla voidaan päätteettömästi testata periaatteessa mitä tahansa GUI-ohjelmia, mutta rajoitteena on kuitenkin se, että niiden täytyy olla saatavilla siihen ympäristöön, jossa Xvfb-näyttöpalvelinta ajetaan.
  Tämä tarkoittaa käytännössä sitä, että web-sovelluksien hyväksymistestauksen automatisoimisesta on jätettävä pois vain Window-ympäristöön saatavien verkkoselainten, kuten Internet Explorer -verkkoselaimen testaaminen.
  Robot Framework takaa helpon testitapauksien luettavuuden kenelle tahansa, mutta ohjelmistokehittäjille se voi olla turhan rajatun tuntuinen.
  Ohjelmistokehittäjänä testitapauksien laatimisen yhteyteen olisi hyvä saada mahdollisuus yksikkötestauskehyksissä käytettävistä ohjelmointikielistä tuttuihin kontrollirakenteisiin, joilla testitapauksien monipuolisuutta voisi kasvattaa perinteisesti yksikkötestauksessa mahdollisten rakenteiden tasolle.
  Tämä ei kuitenkaan ole mahdollista Robot Frameworkissä, jossa testitapauksien laatimiseen käytetään Robot Frameworkin omaa, rajattua syntaksia.

\section{Menetelmän evaluointi} \label{ch:12_menetelman_evaluointi}

  Tässä diplomityössä kehitetyn testitapauksien priorisointimenetelmän kehittäminen onnistui erittäin hyvin ja sen käyttämisellä saavutetaan lisäarvoa etenkin keskisuurien ja suurien web-sovelluksien käyttöliittymien hyväksymistestaukseen.

  Priorisointimenetelmän positiivisia puolia ovat muun muassa sen ominaisuuksiin liittyviä asioita, kuten priorisointimenetelmän toistettavuus, ja mahdollisuus priorisoida käyttöliittymien näkymiä ja siirtymiä.
  Näkymä- ja siirtymäperusteisen priorisoinnin tarkoituksena on mahdollistaa näkymiin perustuvien testikokoelmien priorisoiminen, jolloin niiden tärkeysjärjestys saadaan selville, ja testitapauksien kirjoittaminen voidaan aloittaa prioriteetiltaan tärkeimmästä näkymästä.
  Esimerkkinä menetelmän käyttämisen tuomasta resurssien säästöstä voidaan ottaa tarkasteluun kattavuudet \(c=90\) ja \(c=50\), keskisuurelle käyttöliittymälle, jossa on yhteensä kymmenen erilaista näkymää, ja joista jokaista varten laadittaisiin esimerkin omaisesti kolme testitapausta.
  Tällaisessa tilanteessa kaikkien testitapauksien lukumääräksi tulee kolmekymmentä, joista prioriteetein painotettua verkkoa karsimalla voidaan kuitenkin tiputtaa alhaisimman prioriteetin näkymien avulla osa testitapauksista pois.
  Tämä tarkoittaa kattavuudella \(c=90\) yhden alhaisimmalla prioriteetilla painotetun näkymän jättämistä pois testauksesta, jolloin myös kolme testitapausta voidaan jättää pois toteutuksesta.
  Vastaavasti varsin häikäilemättömällä kattavuudella \(c=50\) voitaisiin jättää pois peräti viisi näkymää, mikä tarkoittaa esimerkissä viittätoista testitapausta, jotka jätettäisiin alhaisen prioriteettinsa myötä pois toteutuksesta.
  Testitapauksien toteuttamatta jättämisellä voidaan väistämättä säästää resursseja, mutta sitä ei kuitenkaan voida tehdä täysin ilman priorisointia.
  Yksi tämän menetelmän suurimmista hyödyistä tulee nimenomaan priorisoimisesta, joka mahdollistaa resurssien säästämisen edellä mainitulla tavalla.

  Menetelmän käyttäminen on tehokkainta, kun testitapaukset kategorisoidaan näkymittäin laadittuihin testikokoelmiin, sillä menetelmän kehittämisen taustalla on ollut ajatus, jossa näkymät vastaavat testikokoelmia.
  Priorisointimenetelmä perustuu matemaattisiin painotettuihin verkkoihin, jotka tuovat hyötynä lyhimmän polun ongelman ratkaisemiseen kehitettyjen algoritmien käyttämisen mahdollistamisen prioriteetiltaan korkeimpien polkujen löytämiseen kahden solmun, eli näkymän, välille.
  Lisäksi painotetun verkon ja matemaattisen lähestymistavan käyttäminen tuo hyötynä sen, että menetelmä on kohtalaisen pienellä vaivalla muunnettavissa tietokoneohjelmaksi.
  Painotettujen verkkojen käyttäminen priorisointiin pakottaa myös menetelmän käyttäjät piirtämään näkymä- ja siirtymäperusteisen painotetun verkon, jolloin se kasvattaa käyttäjien ymmärrystä testauskohteena olevasta järjestelmästä.
  Priorisointimenetelmä on tässä diplomityössä esitetyn esimerkin \ref{tab:esimerkki_verkon_priorisointi_muuttujat} mukaan todettavissa toimivaksi, ja sen avulla on suoritettu priorisointi varsinaisesta testauskohteesta yksinkertaistetulle käyttöliittymälle.

  Priorisointimenetelmässä on myös negatiivisia puolia, mutta ne eivät tässäkään tapauksessa ylitä menetelmän käytöstä saatavaa hyötyä.
  Menetelmässä esittävien toistuvien leikkauksien määrää rajaavan testikattavuuden päättäminen näkymä- ja siirtymäperusteisesti voi olla haastavaa.
  Toinen menetelmään kohdistuva kritiikki koskee menetelmän geneerisyyttä, eli käyttöönottamisen mahdollisuutta ilman muutoksia, jota on vaikea arvioida.
  Priorisointimenetelmässä käytettävät priorisointiin vaikuttavat muuttujat ovat varsin subjektiivisia ja voivat olla testausta toteuttavan tahon mukaan muuttuvia, minkä takia muuttujiin joudutaan mahdollisesti tekemään muutoksia.
  Lisäksi menetelmässä esitetty, priorisointiin vaikuttavia muuttujia hyödyntävä, funktio \(p(v)\) kokonaisprioriteetin laskemiseen ei ota muuttujien määrittämisessä mahdollisesti esiintyvää epälineaarisuutta lainkaan huomioon.
  Tämä tarkoittaa käytännössä sitä, että kokonaisprioriteetin määrittämisen on voitava olla ilmaistavissa siihen vaikuttavien osiensa summana, mikä rajoittaa menetelmän käyttöä epälineaarisissa tapauksissa.
  Negatiivista on myös se, että menetelmän käyttö on soveltuva painotettujen verkkojen luonteen mukaisesti käyttöliittymien tapauksessa soveltuvat vain kokonaisia näkymiä peilaavien testikokoelmien priorisointiin yksittäisten testitapauksien sijaan.
  Priorisointimenetelmän käyttö voi olla turhan aikaa vievää, jos käyttöliittymä on yksinkertainen.

\section{Jatkokehitysehdotukset} \label{ch:12_jatkokehitysehdotukset}

  Tämän diplomityön konkreettisina tuloksina syntyneet testausjärjestelmä ja priorisointimenetelmä ovat sellaisenaan käyttövalmiita ja toimivaksi todettuja, mutta jatkokehittelylle on luonnollisesti niissäkin sijaa.
  Testausjärjestelmän avulla toteutettava web-sovelluksien päätön hyväksymistestaus mahdollistetaan Xvfb-näyttöpalvelimen tarjoaman virtualisoinnin avulla.
  Xvfb-näyttöpalvelin on kuitenkin saatavilla vain UNIX-ympäristöihin, mikä rajaa testausjärjestelmään lisättävien verkkoselaimien saatavuutta.
  Xvfb-näyttöpalvelimelle voitaisiin jatkokehityksenä etsiä monialustaisempi vaihtoehto tai ainakin vastine Windows-ympäristöön, jonka avulla myös vain Windows-alustalle saatavat verkkoselaimet olisi mahdollista lisätä järjestelmään.

  Yksi priorisointimenetelmään liittyvä rajoite on käyttöliittymän näkymä- ja siirtymäperusteinen priorisointi, joka asettaa näkymät vastaamaan testikokoelmia.
  Jatkokehityksenä voitaisiin tutkia näkymäperusteisuuden mukaan tehtävän priorisoinnin muuntamisen mahdollisuutta käyttötapausperusteiseksi.
  Käyttötapausperusteisesti luotava verkko vastaisi parhaimmillaan oikeita käyttäjien tarpeisiin tarkoitettuja toiminallisuuksia ja voisi parantaa priorisointia.
  Priorisointimenetelmä on vahvasti matemaattinen, joka mahdollistaa sen muuntamisen tietokoneohjelmaksi kohtalaisella vaivalla.
  Priorisointimenetelmän rakentaminen automaattisen tietokoneohjelman muotoon olisi erittäin järkevää, ja laskisi menetelmän käyttöönottamiseen tarvittavaa vaivannäköä huomattavasti.
  Lisäksi priorisointimenetelmän näkymäpohjaisten graafien, eli painotettujen verkkojen, visualisointi voitaisiin hoitaa tietokoneohjelman yhteydessä.
