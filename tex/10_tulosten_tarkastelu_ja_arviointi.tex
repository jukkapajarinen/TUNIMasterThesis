Tässä kappaleessa esitetään yhteenveto tutkimuksen tuloksista ja evaluoidaan testausjärjestelmän ja priorisointimenetelmän toteutuksien onnistumista.
Ensin evaluoidaan diplomityössä suunnitellun ja asiakasyritykselle toteutetun testausjärjestelmän positiivisia sekä negatiivisia puolia.
Seuraavaksi evaluoidaan diplomityössä kehitetyn priorisointimenetelmän positiivisia ja negatiivisia puolia ja muun muassa pohditaan kuinka hyvin soveltuva ja toistettavissa oleva kyseinen kehitetty priorisointimenetelmä on.
Lisäksi lopussa vielä esitetään esiin tulleita jatkokehitysehdotuksia testausjärjestelmälle sekä priorisointimenetelmälle.

\section{Tutkimuksen konkreettiset tulokset} \label{ch:12_tutkimuksen_konkreettiset_tulokset}

  Työn tuloksena kehitetty  web-käyttöliittymien hyväksymistestauksen automatisoimisen mahdollistava testausjärjestelmä on integroitu onnistuneesti osaksi GoCD-palvelimen avulla suoritettavaa jatkuvaa integrointia.
  Lyhyesti sanottuna testausjärjestelmän osalta konkreettinen tulos koostuu järjestelmästä, joka mahdollistaa testitapauksien luomisen Robot Framework alustalle käyttäen Selenium kirjastoa, Xvfb virtualisointipalvelinta ja Docker säiliöintiohjelmistoa.
  Testausjärjestelmän toimivuus käytännössä todettiin esimerkkitestitapauksien muodossa oikeassa ympäristössään ja testausjärjestelmän mahdollistamat ominaisuudet ovat jo itsessään oikeassa ja lopullisessa käyttöympäristössä tarvittavia.

  Web-sovelluksien näkymä ja siirtymäperustainen painotettua verkkoa hyödyntävä priorisointimenetelmä on myös todettu toimivaksi lähestysmistavaksi priorisointiin.
  Lyhyesti sanottuna priorisointimenetelmän tuloksena on painotettuja verkkoja hyödyntävä menetelmä, jossa määritetään priorisointiin vaikuttavat muuttujat, painofunktiot, painomatriisi, prioriteetteihin perustuvien leikkauksien tekeminen ja prioriteettien löytäminen sekä lukeminen verkosta.
  Priorisointimenetelmän toimivuus käytännössä todettiin aidosta ympäristöstä yksinkertaistaen poimitusta web-sovelluksesta.
  Priorisointimenetelmän avulla todettiin, että priorisoinnin aikana toteutetut painotetun verkon leikkaukset olivat juuri niitä näkymiä, jotka vaistonvaraisesti ilman menetelmänkin käyttöä karsittaisiin.

\section{Toteutuksen evaluointi} \label{ch:12_toteutuksen_evaluointi}

  Docker säiliönnin avulla tuki myös manuaaliselle testitapauksien ajamiselle

  Docker säiliönnin avulla sovelluskehittäjät saavat valmiin hyväksymistestausjärjestelmän helposti käyttöönsä

  Docker säiliöinnin takia toteutus ei ole sidottu CI palvelimeen

  Xvfb virtualisoinnin avulla voidaan uusia verkkoselaimia lisätä helposti

  Hyvä skaalautuvuus, Docker-säilöitä on helppo rakentaa ja GoCD agentteja lisätä



  Xvfb virtuallisoinnin avulla ei voida testata vain Windows ympäristöön saatavia GUI-ohjelmia

  Robot Framework takaa helpon luettavuuden kenelle tahansa, mutta ohjelmistokehittäjälle rajatun tuntuinen

\section{Menetelmän evaluointi} \label{ch:12_menetelman_evaluointi}

  Priorisointimenetelmän toistettavuus

  Menetelmän avulla on mahdollista priorisoida näkymiä ja siirtymiä

  Dijkstran algoritmin avulla voidaan selvittää verkosta prioriteetiltaan korkein polku, eli näkymät joiden testikokoelmat tulisi suorittaa ensimmäisenä.

  Painotettu verkko voidaan piirtää, joka kasvattaa ymmärrystä järjestelmästä

  Menetelmän käyttö on tehokkainta kun testitapaukset kategorisoidaan näkymittäin testikokoelmiin



  Testikattavuuden päättäminen näkymä- ja siirtymäperusteisesti voi olla haastavaa

  Ei välttämättä ainakaan muokkauksia tekemättä geneerinen menetelmä

  Menetelmän käyttö soveltuu testikokoelmien, ei testitapauksien, priorisointiin

  Priorisointimenetelmän käyttö voi olla turhan aikaa vievää jos käyttöliittymä on yksinkertainen

\section{Jatkokehitysehdotukset} \label{ch:12_jatkokehitysehdotukset}

  Xvfb:lle vastineen löytäminen Window ympäristöön

  Parempi priorisointi muuttamalla näkymäperusteisuus käyttötapausperustaiseksi

  Näkymägraafien visualisointi ja muu priorisointiohjelman jatkokehittely
