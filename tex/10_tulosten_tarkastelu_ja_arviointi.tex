Tässä kappaleessa esitetään yhteenveto tutkimuksen tuloksista ja evaluoidaan testausjärjestelmän ja priorisointimenetelmän toteutuksien onnistumista.
Ensin evaluoidaan diplomityössä suunnitellun ja asiakasyritykselle toteutetun testausjärjestelmän positiivisia sekä negatiivisia puolia.
Seuraavaksi evaluoidaan diplomityössä kehitetyn priorisointimenetelmän positiivisia ja negatiivisia puolia ja muun muassa pohditaan kuinka hyvin soveltuva ja toistettavissa oleva kyseinen kehitetty priorisointimenetelmä on.
Lisäksi lopussa vielä esitetään toteutuksen jälkeen esiin tulleita jatkokehitysehdotuksia testausjärjestelmälle sekä priorisointimenetelmälle.

\section{Tutkimuksen konkreettiset tulokset} \label{ch:12_tutkimuksen_konkreettiset_tulokset}

  Työn tuloksena kehitetty  web-käyttöliittymien hyväksymistestauksen automatisoimisen mahdollistava testausjärjestelmä on integroitu onnistuneesti osaksi GoCD-palvelimen avulla suoritettavaa jatkuvaa integrointia.
  Lyhyesti sanottuna testausjärjestelmän osalta konkreettinen tulos koostuu järjestelmästä, joka mahdollistaa testitapauksien luomisen Robot Framework alustalle käyttäen Selenium kirjastoa, Xvfb virtualisointipalvelinta ja Docker säiliöintiohjelmistoa.
  Testausjärjestelmän toimivuus käytännössä todettiin esimerkkitestitapauksien muodossa oikeassa ympäristössään ja testausjärjestelmän mahdollistamat ominaisuudet ovat jo itsessään oikeassa ja lopullisessa käyttöympäristössä tarvittavia.

  Web-sovelluksien näkymä ja siirtymäperustainen painotettua verkkoa hyödyntävä priorisointimenetelmä on myös todettu toimivaksi lähestysmistavaksi priorisointiin.
  Lyhyesti sanottuna priorisointimenetelmän tuloksena on painotettuja verkkoja hyödyntävä menetelmä, jossa määritetään priorisointiin vaikuttavat muuttujat, painofunktiot, painomatriisi, prioriteetteihin perustuvien leikkauksien tekeminen ja prioriteettien löytäminen sekä lukeminen verkosta.
  Priorisointimenetelmän toimivuus käytännössä todettiin aidosta ympäristöstä yksinkertaistaen poimitusta web-sovelluksesta.
  Priorisointimenetelmän avulla todettiin, että priorisoinnin aikana toteutetut painotetun verkon leikkaukset olivat juuri niitä näkymiä, jotka vaistonvaraisesti ilman menetelmänkin käyttöä karsittaisiin.

\section{Toteutuksen evaluointi} \label{ch:12_toteutuksen_evaluointi}

  Kokonaisuutena hyväksymistestausjärjestelmän toteutus onnistui erittäin hyvin ja sen avulla on mahdollista jopa geneerisesti rakentaa web-sovelluksesta riippumattomasti hyväksymistestaus testauskohteena olevalle web-sovellukselle.

  Testausjärjestelmän positiivisia puolia ovat muun muassa Docker säiliöinnin avulla saatava tuki myös manuaaliselle testitapauksien ajamiselle.
  Docker säiliö, joka mahdollistaa testitapauksien ajamisen voidaan pystyttää periaatteessa mihin tahansa ympäristöön, jossa Docker on saatavilla.
  Docker säiliöinnin avulla myös ohjelmistokehittäjät saavat valmiin hyväksymistestausjärjestelmän helposti käyttöönsä.
  Docker säiliöinnin ja Docker-compose:n avulla rakennettu järjestelmä ei myöskään ole sidottu mihinkään ennalta määritettyyn jatkuvan integroinnin palvelimeen, joka huomattavasti helpottaa testausjärjestelmän käyttöönottoa osaksi uusia tai muuttuvaa ohjelmistotuotannon prosessia.
  Testausjärjestelmä mahdollistaa päätteettömän testauksen virtuaalisen Xfvb-näyttöpalvelimen avulla, joka on itsessään erittäin tarvittu ominaisuus jatkuvan integroinnin ja testausjärjestelmän yhdistämiseen.
  Xvfb-näyttöpalvelimen tarjoaman virtualisoinnin avulla voidaan myös uusia WebDriver rajapinnan toteuttavia verkkoselaimia lisätä päätteettömän testauksen alaisuuteen erittäin helposti ja käytännössä rajoituksitta.
  Ainoa vaatimus on, että verkkoselain on saatavilla siihen ympäristöön jossa Xvfb-näyttöpalvelinta ajetaan.

  Testausjärjestelmässä on kuitenkin myös negatiivisia puolia, joiden osalta järjestelmän käyttö on rajattua.
  Xvfb-näyttöpalvelimen avulla voidaan päätteettömästi testata periaatteessa mitä tahansa GUI-ohjelmia, mutta rajoitteena on kuitenkin, että niiden täytyy olla saatavilla siihen ympäristöön, jossa Xvfb-näyttöpalvelinta ajetaan.
  Tämä tarkoittaa käytännössä sitä, että web-sovelluksien hyväksymistestauksen automatisoimisesta on jätettävä pois vain Window-ympäristöön saatavien verkkoselainten, kuten Internet Explorer selaimen testaaminen.
  Robot Framework takaa helpon testitapauksien luettavuuden kenelle tahansa, mutta ohjelmistokehittäjille se voi olla turhan rajatun tuntuinen.
  Ohjelmistokehittäjänä testitapauksien laatimisen yhteyteen olisi hyvä saada mahdollisuus yksikkötestauskehyksissä käytettävistä ohjelmointikielistä tuttuihin kontrollirakenteisiin, joilla testitapauksien monipuolisuutta voisi kasvattaa perinteisesti yksikkötestauksessa mahdollisien rakenteiden tasolle.
  Tämä ei kuitenkaan ole mahdollista Robot Framework:issä, jossa testitapauksien laatimiseen käytetään Robot Framework:in omaa, rajattua syntaksia.

\section{Menetelmän evaluointi} \label{ch:12_menetelman_evaluointi}

  Priorisointimenetelmän toistettavuus

  Menetelmän avulla on mahdollista priorisoida näkymiä ja siirtymiä

  Dijkstran algoritmin avulla voidaan selvittää verkosta prioriteetiltaan korkein polku, eli näkymät joiden testikokoelmat tulisi suorittaa ensimmäisenä.

  Painotettu verkko voidaan piirtää, joka kasvattaa ymmärrystä järjestelmästä

  Menetelmän käyttö on tehokkainta kun testitapaukset kategorisoidaan näkymittäin testikokoelmiin



  Testikattavuuden päättäminen näkymä- ja siirtymäperusteisesti voi olla haastavaa

  Ei välttämättä ainakaan muokkauksia tekemättä geneerinen menetelmä

  Menetelmän käyttö soveltuu testikokoelmien, ei testitapauksien, priorisointiin

  Priorisointimenetelmän käyttö voi olla turhan aikaa vievää jos käyttöliittymä on yksinkertainen

\section{Jatkokehitysehdotukset} \label{ch:12_jatkokehitysehdotukset}

  Xvfb:lle vastineen löytäminen Window ympäristöön

  Parempi priorisointi muuttamalla näkymäperusteisuus käyttötapausperustaiseksi

  Näkymägraafien visualisointi ja muu priorisointiohjelman jatkokehittely
