Tässä luvussa esitetään diplomityön tutkimuskysymykset sekä käytetty tutkimusmenetelmä.
Tutkimuskysymykset liittyvät vahvasti yhteiseen priorisoinnin teemaan, johon tässä työssä erityisesti paneudutaan.
Lisäksi työn lopussa on myös toteutuksellinen osuus, joka on tehty diplomityön asiakasyrityksen tarpeita varten.
Toteutuksellisessa osuudessa on paljon muutakin sisältöä, joka on priorisointiteeman ulkopuolella, mutta pysyy kuitenkin työn kokonaiskontekstissa.

\section{Tutkimuskysymykset}

Tutkimuksen tarkoituksena on pohjimmiltaan tarkoitus löytää ja kehittää toistettavissa oleva menetelmä hyväksymistestauksen testitapauksien priorisoimiseen.

Työlle asetettiin seuraavat tutkimuskysymykset:
\begin{itemize}
	\item T1: \emph{Miten painotettua verkkoa voidaan käyttää testitapauksien priorisoimiseen?}
  \item T2: \emph{Mitkä muuttujat vaikuttavat web-käyttöliittymän hyväksymistestauksen testitapauksien priorisointiin?}
  \item T3: \emph{Kuinka prioriteetein painotetusta verkosta valitaan toteutettavat testitapaukset?}
  \item T4: \emph{Miten painotetun verkon avulla tehty priorisointi liitetään yhteen jatkuvan integraation ja testiautomaation kanssa?}
\end{itemize}

\section{Tutkimusmenetelmä}

Tutkimuskysymyksiin vastaamiseksi työn tutkimusmenetelmäksi valittiin kirjallisuusanalyysi.

<perustutkimus, aikaisemmin saatavilla oleva tieto koontiin>

<menetelmä>

<laadullinen>

<kerätty aineisto>

<teoreettinen kehys>

Aineston hallintaan käytettiin tietokoneohjelmistoa, jossa aineisto kategorisoitiin eri loogisiin kokonaisuuksiin muun muassa priorisoinnin ja painetetun verkon osalta.
Kerätyn aineiston avulla pyrittiin luomaan mahdollisimman vahva teoreettinen pohja tutkimuskysymyksiin vastaamiseksi mahdollisimman kattavasti.