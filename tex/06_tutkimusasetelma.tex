Tässä luvussa esitetään diplomityön taustaa, tutkimuskysymykset, tutkimusmenetelmä, tutkimuksen rajaus sekä tavoitteet.
Tutkimuskysymykset liittyvät vahvasti yhteiseen priorisoinnin teemaan, johon tässä työssä erityisesti paneudutaan.
Lisäksi työn lopussa on myös toteutuksellinen osuus, joka on tehty diplomityön asiakasyrityksen tarpeita varten.
Toteutuksellisessa osuudessa on paljon muutakin sisältöä, joka on varsinaisen priorisointiteeman ulkopuolella, mutta pysyy kuitenkin työn kokonaiskontekstissa.

\section{Tausta}

Diplomityö tehtiin WordDive nimiselle yritykselle. WordDive on Tampereella toimiva, suomalainen kieltenoppimiseen keskittyvä yritys. WordDivellä oli kirjoitushetkellä kieltenoppimissovellus sekä mobiilialustalle että web-alustalle. Tämän diplomityön sisältö koskettaa web-alustalla toimivaa sovellusta. Hyväksymistestauksen osalta mobiilisovellukselle oli yrityksessä toteutettu testiautomaatio, mutta web-alustalle sitä ei vielä oltu tehty.

% Lisää tekstiä tähän

\section{Tutkimuskysymykset} \label{tutkimuskysymykset}

Tutkimuksen tarkoituksena on pohjimmiltaan tarkoitus löytää ja kehittää toistettavissa oleva menetelmä hyväksymistestauksen testitapauksien priorisoimiseen.
Testitapauksien laatimisen yleisenä ongelmanakohtana on erityisesti niiden priorisointi, joka usein johtaa liian suppean tai ylikattavan testiautomaation rakentamiseen.
Tutkimuskysymykset on laadittu siten, että niihin vastaaminen antaa ratkaisun edellä mainittuun testiautomaation ongelmaan.

Työlle asetettiin seuraavat tutkimuskysymykset:
\begin{itemize}
  \item T1: \emph{Miten painotettua verkkoa voidaan käyttää testitapauksien priorisoimiseen?}
  \item T2: \emph{Mitkä muuttujat vaikuttavat web-käyttöliittymän hyväksymistestauksen testitapauksien priorisointiin?}
  \item T3: \emph{Kuinka prioriteetein painotetusta verkosta valitaan toteutettavat testitapaukset?}
  \item (T4: \emph{Miten painotetun verkon avulla tehty priorisointi liitetään yhteen jatkuvan integraation ja testiautomaation kanssa?)}
\end{itemize}

\section{Tutkimusmenetelmä}

Tutkimuskysymyksiin vastaamiseksi työn tutkimusmenetelmäksi valittiin diplomitöissä yleisesti käytetty design science menetelmä.

Tarkoituksena oli muodostaa uudenlainen toistettavissa oleva menetelmä tutkimuksen kohteena olevan ongelman ratkaisemiseksi.

Tutkimusidean hahmottelemisen ja ratkaisua kaipaavan ongelman identifoinnin jälkeen, valittua tutkimusmenetelmää käyttäen ensin määriteltiin tutkimuskysymykset \ref{tutkimuskysymykset}.

Seuraavaksi kartoitettiin ratkaisuvaihtoehto tutkimuskysymyksiin ja esitetään perustelut siihen päätymiseen.

Asiakasyrityksen ohjelmistotuotetta silmälläpitäen toteutettiin kokonaisratkaisu hyödyntäen kehitettyä ratkaisumenetelmää.

Lopuksi vielä evaluoitiin ratkaisun toimivuus ja esitetään yhteenveto tutkimuksesta.

% Tutkimusaiheen suunnittelemisen alkuvaiheessa huomattiin, että olemassa olevat käsitteelliset mallit ja teoria eivät olleet kovin vakiintunutta ja jäsenneltyä, joka antoi lisää painoarvoa työn tekemiselle.
% Tutkittavaan asiaan liittyvää aineistoa muun muassa priorisoinnin ja painotettujen verkkojen osalta on saatavilla runsaasti, joten aineiston valitseminen perustui harkintaan.
% Aineston hallintaan käytettiin tietokoneohjelmistoa, jossa aineisto kategorisoitiin eri loogisiin kokonaisuuksiin muun muassa priorisoinnin ja painotetun verkon osalta.
% Kerätyn aineiston avulla pyrittiin luomaan mahdollisimman vahva teoreettinen pohja tutkimuskysymyksiin vastaamiseksi mahdollisimman kattavasti.

\section{Tutkimuksen rajaus}

Ohjelmistotestauksen tasojen osalta tutkimus rajoittuu hyväksymistestaukseen.

Testitapauksien osalta on olemassa lukuisia eri testausalustoja, joita hyödyntäen testitapauksia voidaan toteuttaa. Tässä työssä testitapauksien toteuttaminen rajataan tietylle ennalta määräytyneelle Robot Framework alustalle.

Jatkuvan integroinnin osalta tutkimus rajoittuu perusteisiin ja painoarvo pidetään testitapauksien toteuttamisessa ja niiden priorisoinnissa.

Verkkoteorian osalta tutkimus rajoittuu perusteisiin ja painotettua verkkoa sekä kehitettyä menetelmää tukeviin käsitteisiin.

Priorisointiin vaikuttavien muuttujien osalta tutkimus rajoittuu muuttujien kartoittamiseen, mutta niiden määrittäminen jätetään työn ulkopuolelle. Tämä tarkoittaa käytännössä sitä, että jokainen menetelmää hyödyntävä taho hankkii itse varsinaiset numeeriset arvot muuttujille. Esimerkiksi liiketoiminnallisen vision arvo on menetelmää käyttävän yrityksen vastuulla.

Painotetun verkon lyhimmän polun etsimiseen on olemassa lukuisa määrä erilaisia algoritmeja, mutta tässä työssä hyödynnetään vain perinteistä Djikstran algoritmia.

\section{Tavoitteet}

Tutkimuksen tavoitteena on löytää toistettavissa oleva menetelmä hyväksymistestauksessa tarvittavien testitapauksien priorisoimiseen.

Lisäksi tavoitteena on pystyä todentamaan kehitetyn menetelmän toimivuus käytännössä asiakasyritykselle tehtyä toteutusta evaluoimalla.
