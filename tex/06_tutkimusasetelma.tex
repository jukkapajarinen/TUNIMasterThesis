Tässä luvussa esitetään diplomityön tutkimuskysymykset sekä käytetty tutkimusmenetelmä.
Tutkimuskysymykset liittyvät vahvasti yhteiseen priorisoinnin teemaan, johon tässä työssä erityisesti paneudutaan.
Lisäksi työn lopussa on myös toteutuksellinen osuus, joka on tehty diplomityön asiakasyrityksen tarpeita varten.
Toteutuksellisessa osuudessa on paljon muutakin sisältöä, joka on varsinaisen priorisointiteeman ulkopuolella, mutta pysyy kuitenkin työn kokonaiskontekstissa.

\section{Tutkimuskysymykset}

Tutkimuksen tarkoituksena on pohjimmiltaan tarkoitus löytää ja kehittää toistettavissa oleva menetelmä hyväksymistestauksen testitapauksien priorisoimiseen.
Testitapauksien laatimisen yleisenä ongelmanakohtana on erityisesti niiden priorisointi, joka usein johtaa liian suppean tai ylikattavan testiautomaation rakentamiseen.
Tutkimuskysymykset on laadittu siten, että niihin vastaaminen antaa ratkaisun edellä mainittuun testiautomaation ongelmaan.

Työlle asetettiin seuraavat tutkimuskysymykset:
\begin{itemize}
  \item T1: \emph{Miten painotettua verkkoa voidaan käyttää testitapauksien priorisoimiseen?}
  \item T2: \emph{Mitkä muuttujat vaikuttavat web-käyttöliittymän hyväksymistestauksen testitapauksien priorisointiin?}
  \item T3: \emph{Kuinka prioriteetein painotetusta verkosta valitaan toteutettavat testitapaukset?}
  \item (T4: \emph{Miten painotetun verkon avulla tehty priorisointi liitetään yhteen jatkuvan integraation ja testiautomaation kanssa?)}
\end{itemize}

\section{Tutkimusmenetelmä}

Tutkimuskysymyksiin vastaamiseksi työn tutkimusmenetelmäksi valittiin \st{ankkuroitu tutkimus}, jota mielessä pitäen tutkimuksessa käytettävä ainesto kerättiin.
Tarkoituksena oli muodostaa uudenlainen teoreettinen näkökulma tutkivaan asiaan, siihen liittyvää aineistoa tarkastelemalla ja jäsentämällä.
Tutkimusaiheen suunnittelemisen alkuvaiheessa huomattiin, että olemassa olevat käsitteelliset mallit ja teoria eivät olleet kovin vakiintunutta ja jäsenneltyä, joka antoi lisää painoarvoa työn tekemiselle.
Tutkittavaan asiaan liittyvää aineistoa muun muassa priorisoinnin ja painotettujen verkkojen osalta on saatavilla runsaasti, joten kyseessä ei ole niin sanottu kokonaisututkimus, vaan aineiston valitseminen perustuu harkintaan.
Strategisesti työ kuuluu teoreettisen tutkimuksen piiriin ja se on myös niin sanottu perustutkimus, jossa aikaisemmin saatavilla olevaa tietoa kootaan ja jäsennellään uudestaan järkeviin tutkimuskysymyksiin vastaaviin kokonaisuuksiin.
Tutkimusosuus ja teorian muodostaminen diplomityöstä on pyritty pitämään kvalitatiivisena, eli saatavilla olevaa ja teemaan liittyvää aineistoa kerättiin pitäen tutkimuksen paino määrän sijaan laadussa.
Aineston hallintaan käytettiin tietokoneohjelmistoa, jossa aineisto kategorisoitiin eri loogisiin kokonaisuuksiin muun muassa priorisoinnin ja painotetun verkon osalta.
Kerätyn aineiston avulla pyrittiin luomaan mahdollisimman vahva teoreettinen pohja tutkimuskysymyksiin vastaamiseksi mahdollisimman kattavasti.

\section{Tutkimussuunnitelma}

% Teksti tähän