Tässä luvussa esitetään diplomityön tutkimuskysymykset sekä käytetty tutkimusmenetelmä.
Tutkimuskysymykset liittyvät vahvasti yhteiseen priorisoinnin teemaan, johon tässä työssä erityisesti paneudutaan.
Lisäksi työn lopussa on myös toteutuksellinen osuus, joka on tehty diplomityön asiakasyrityksen tarpeita varten.
Toteutuksellisessa osuudessa on paljon muutakin sisältöä, joka on varsinaisen priorisointiteeman ulkopuolella, mutta pysyy kuitenkin työn kokonaiskontekstissa.

\section{Tutkimuskysymykset} \label{tutkimuskysymykset}

Tutkimuksen tarkoituksena on pohjimmiltaan tarkoitus löytää ja kehittää toistettavissa oleva menetelmä hyväksymistestauksen testitapauksien priorisoimiseen.
Testitapauksien laatimisen yleisenä ongelmanakohtana on erityisesti niiden priorisointi, joka usein johtaa liian suppean tai ylikattavan testiautomaation rakentamiseen.
Tutkimuskysymykset on laadittu siten, että niihin vastaaminen antaa ratkaisun edellä mainittuun testiautomaation ongelmaan.

Työlle asetettiin seuraavat tutkimuskysymykset:
\begin{itemize}
  \item T1: \emph{Miten painotettua verkkoa voidaan käyttää testitapauksien priorisoimiseen?}
  \item T2: \emph{Mitkä muuttujat vaikuttavat web-käyttöliittymän hyväksymistestauksen testitapauksien priorisointiin?}
  \item T3: \emph{Kuinka prioriteetein painotetusta verkosta valitaan toteutettavat testitapaukset?}
  \item (T4: \emph{Miten painotetun verkon avulla tehty priorisointi liitetään yhteen jatkuvan integraation ja testiautomaation kanssa?)}
\end{itemize}

\section{Tutkimusmenetelmä}

Tutkimuskysymyksiin vastaamiseksi työn tutkimusmenetelmäksi valittiin diplomitöissä yleisesti käytetty design science menetelmä.
Tarkoituksena oli muodostaa uudenlainen toistettavissa oleva menetelmä tutkimuksen kohteena olevan ongelman ratkaisemiseksi.
Tutkimusidean hahmottelemisen jälkeen, valittua tutkimusenetelmää käyttäen ensin määriteltiin tutkimuskysymykset \ref{tutkimuskysymykset}.
Seuraavaksi kartoitettiin ratkaisuvaihtoehdot tutkimuskysymyksiin ja valittiin sopivimmat ratkaisut perustelut esittäen.
Lisäksi sitten esitettiin yhtenäinen ratkaisu kokonaisuudessaan tutkittavan aiheen osalta.
Asiakasyrityksen ohjelmistotuotetta silmälläpitäen toteutettiin kokonaisratkaisu.
Lopuksi vielä evaluoitiin ratkaisun toimivuus ja esitetään yhteenveto tutkimuksesta.

Tutkimusaiheen suunnittelemisen alkuvaiheessa huomattiin, että olemassa olevat käsitteelliset mallit ja teoria eivät olleet kovin vakiintunutta ja jäsenneltyä, joka antoi lisää painoarvoa työn tekemiselle.
Tutkittavaan asiaan liittyvää aineistoa muun muassa priorisoinnin ja painotettujen verkkojen osalta on saatavilla runsaasti, joten aineiston valitseminen perustui harkintaan.
Aineston hallintaan käytettiin tietokoneohjelmistoa, jossa aineisto kategorisoitiin eri loogisiin kokonaisuuksiin muun muassa priorisoinnin ja painotetun verkon osalta.
Kerätyn aineiston avulla pyrittiin luomaan mahdollisimman vahva teoreettinen pohja tutkimuskysymyksiin vastaamiseksi mahdollisimman kattavasti.

\section{Tutkimussuunnitelma}

% Teksti tähän