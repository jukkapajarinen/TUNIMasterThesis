Tässä luvussa esitetään diplomityön taustaa, tutkimuskysymykset, käytetty tutkimusmenetelmä, tutkimuksen rajaus sekä tavoitteet.
Tutkimuskysymykset liittyvät vahvasti yhteiseen priorisoinnin teemaan, johon tässä työssä erityisesti paneudutaan.
Tutkimus on soveltavaa ja sen tarkoituksena on muodostaa selvitys tutkimusongelman ratkaisemiseksi.
Tässä työssä se tarkoittaa erityisesti matemaattisen, toistettavissa olevan menetelmän kehittämistä tutkimusongelman ratkaisemiseksi.
Tutkimuskysymyksistä itsessään voi päätellä tutkimuksen tarkoitusta ja tavoitteita, mutta tämä esitetään myös yksityiskohtaisemmin tavoitteet luvussa \ref{06_tavoitteet}.
Lisäksi työn lopussa on myös toteutuksellinen osuus, joka on tehty diplomityön asiakasyrityksen tarpeita varten.
Toteutuksellisessa osuudessa on paljon muutakin sisältöä, joka on varsinaisen priorisointiteeman ulkopuolella, mutta pysyy kuitenkin työn kokonaiskontekstissa.

\section{Tausta} \label{06_tausta}

Diplomityö tehtiin WordDive nimiselle yritykselle. WordDive on vuonna 2009 perustettu Tampereella toimiva, suomalainen kieltenoppimiseen keskittyvä yritys. WordDivellä oli kirjoitushetkellä kieltenoppimissovellus mobiilialustalle sekä web-alustalle.
Tämän diplomityön sisältö koskettaa vain web-alustalla toimivaa sovellusta.
Hyväksymistestauksen osalta mobiilisovellukselle oli yrityksessä jo toteutettu testiautomaatio, mutta web-alustalle sitä ei vielä oltu tehty.

Allekirjoittanut aloitti työt kyseisessä yrityksessä 2018 vuoden loppupuolella, jolloin diplomityön aihetta ei vielä ollut.
Tarkoituksena oli tuolloin ensin töitä tekemällä tutustua yrityksen web-alustalla toimivaan sovellukseen ja yrityksen ohjelmistotuotantoprosessiin.
Diplotyön aihe alkoi muotoutua vasta vuoden 2019 alkupuolella, kun tarvittava tietämys ohjelmistotuotteesta ja prosessista oli saavutettu.
Asiakasyrityksessä sai hyvinkin vapaasti löytää itseään kiinnostavan, varsinaisten töiden ohella tehtävän, mutta kaikille osapuolille hyödyllisen aiheen.
Aiheen löytämisen taustalla olivat hyvinkin konkreettiset tarpeet, jotka ohjelmistotuotannon työssä tulivat esille.

Uusien ominaisuuksien ja koodimuutoksien tekemisen yhteydessä oli jatkuvasti tarve huolelliselle testaamiselle ja erityisesti asiakkaan näkökulmasta tärkeimpien sovelluksen ominaisuuksien toiminnan varmistamiselle.
Tämä sai diplomityön aiheen suuntautumaan testiautomaatioon ja erityisesti hyväksymistestaukseen.
Lisäksi yrityksessä oli jo toteutettuna päivittäisessä käytössä oleva hyväksymistestaus mobiilialustalle, joka auttoi hahmottamaan web-sovelluksen testiautomaation integroimista osaksi yrityksen ohjelmistotuotantoprosessia.
Mobiilialustalle tehtyä hyväksymistestausta varten oli yrityksessä jo valittu tietyt hyväksi todetut sovelluskehykset ja työkalut testiautomaatiota varten, joten tässä työssä ei enää ollut tarvetta evaluoida eri työkaluja tarvittavan testiautomaation toteuttamiseksi.
Web-sovelluksen testiautomaatio toteutettiin käyttäen pääpiirteittäin samoja sovelluskehyksiä ja työkaluja \ref{11_sovelluskehykset_ja_tyokalut} kuin mobiilisovellukselle.
Tästä syystä diplomityön aihetta ja tutkimusongelmaa lähdettiin etsimään muualta.

Tutkimusongelmaan miettiin aihioita etenkin alkuvaiheessa polkutestauksen ja ohjelmistotestaukseen liittyvien heuristiikkojen osalta.
Polkutestaus oli yhtenä vaihtoehtona, mutta siitä luovuttiin, koska sen todettiin olevan paremmin soveltuvampi hyväksymistestausta alemmille testauksen tasoille.
Heuristiikkojen hyödyntämistä mietittiin kahteen eri ongelmaan; testitapauksien muodostamiseen sekä kriittisten testitapauksien määrittämiseen.
Näistä kahdesta, tässä työssä sivutaan heuristiikkojen käyttämistä testitapauksien muodostamiseen luvussa \ref{08_testitapauksien_maarittaminen}, mutta sitä ei käsitellä tutkimusongelmana.

Lopullisen tutkimusongelman löytämiseen vaikuttivat erityisesti konkreettiset tarpeet, jotka tulivat esiin vasta web-sovelluksen testiautomaation suunnitteluvaiheessa.
Hyväksymistestauksen testiautomaatiota varten oli ensin määritettävä mitä testauksen kohteena olevasta sovelluksesta tulisi testata.
Testiautomaation rakentamiseen allokoitavia resursseja oli rajallinen määrä, jonka lisäksi testikattavuuden suppeus sekä ylikattavuus nähtiin selkeänä ongelmana.
Tämä ongelma voidaan esittää yksinkertaisemmin testitapauksien priorisointiongelmana, joka myös lopulta muotoutui työn tutkimusongelmaksi.
Priorisointiongelman valitsemiseen ratkaisevasti johtavia asioita olivat kaksi allekirjoittaneen oivallusta aiheesta.
Ensimmäiseksi hyväksymistestattavaa web-sovellusta keksittiin ajatella käyttöliittymän näkymä ja siirtymätasolla matemaattisena prioriteetein painotettuna verkkona.
Toiseksi oivallukseksi keksittiin käyttää lyhimmän polun ongelmaan kehitettyjä algoritmeja prioriteetein painotetun verkon karsimiseen, jolloin alhaisen prioriteetin solmuja saatiin leikattua pois.
Nämä oivallukset vaikuttivat lopulta varsinaisen tutkimusongelman eli testitapauksien priorisointiongelman valitsemiseen, koska ne loivat järkevän ja mielenkiintoisen pohjan tutkimusongelmaan vastaamiselle.
Kokonaisuutena diplomityön aihe saatiin muodostettua sellaiseksi, että se esittää yleishyödyllisen menetelmän tutkimusongelman ratkaisemiseen sekä sitä hyödyntävän toteutuksen suunnittelemisen ja rakentamisen asiakasyritykselle.

\section{Tutkimuskysymykset} \label{06_tutkimuskysymykset}

Tutkimuksen tarkoituksena on pohjimmiltaan tarkoitus löytää ja kehittää toistettavissa oleva menetelmä hyväksymistestauksen testitapauksien priorisoimiseen.
Testitapauksien laatimisen yleisenä ongelmanakohtana on erityisesti niiden priorisointi, joka usein johtaa liian suppean tai ylikattavan testiautomaation rakentamiseen.
Tutkimuskysymykset on laadittu siten, että niihin vastaaminen antaa ratkaisun tähän edellä mainittuun testiautomaation ongelmaan.

Työlle asetettiin seuraavat tutkimuskysymykset:
\begin{itemize}
  \item T1: \emph{Miten painotettua verkkoa voidaan käyttää testitapauksien priorisoimiseen?}
  \item T2: \emph{Mitkä muuttujat vaikuttavat web-käyttöliittymän hyväksymistestauksen testitapauksien priorisointiin?}
  \item T3: \emph{Kuinka prioriteetein painotetusta verkosta valitaan toteutettavat testitapaukset?}
  \item T4: \emph{Miten painotetun verkon avulla tehty priorisointi liitetään yhteen jatkuvan integraation ja testiautomaation kanssa?}
\end{itemize}

\section{Tutkimusmenetelmä} \label{06_tutkimusmenetelma}

Tutkimuskysymyksiin vastaamiseksi työn tutkimusmenetelmäksi valittiin tietotekniikan diplomitöissä yleisesti käytetty Design Science menetelmä.
Menetelmän tarkoituksena on tuottaa teknologiaa hyödyntävä ratkaisu tutkimusongelmaan vastaamiseen.
Design Science menetelmää käyttäessä pyritään tutkimaan uusia ratkaisumalleja ratkaisemattomiin ongelmiin tai kehittämään parempia ratkaisumalleja jo aiemmin ratkaistujen ongelmien tilalle.
Tietotekniikan tutkimuksessa on aikojen saatossa kehitetty uusia tai parempia tietokonearkkitehtuureja, ohjelmointikieliä, algoritmeja, tietorakenteita ja tiedonhallintajärjestelmiä.
Näiden osalta yhteistä on, että niissä on monesti käytetty usein jopa tiedostamatta Design Science menetelmää.

Design Science menetelmän vaiheet ovat yksinkertaistetusti seuraavat:
\begin{enumerate}
  \item Tutkimusongelman tai kysymyksen määrittäminen
  \item Ratkaisuvaihtoehtojen kartoittaminen ja perusteltu valitseminen
  \item Ratkaisun toteuttaminen tai prototypointi
  \item Ratkaisun toimivuuden evaluointi
\end{enumerate}

Tutkimuksen tarkoituksena oli muodostaa uudenlainen toistettavissa oleva menetelmä tutkimuksen kohteena olevan ongelman ratkaisemiseksi.
Tutkimusidean hahmottelemisen ja ratkaisua kaipaavan ongelman identifoinnin jälkeen, valittua tutkimusmenetelmää käyttäen ensin määriteltiin tutkimuskysymykset \ref{06_tutkimuskysymykset}.

Seuraavaksi kartoitettiin ratkaisuvaihtoehto tutkimuskysymyksiin ja työn yleiseen priorisoinnin teemaan vastaamiseksi ja esitetään perustelut matemaattiseen toistettavissa olevaan ratkaisumenetelmään päätymiseen.
Tutkimusta varten kerättiin teoreettista aineistoa, jonka tarkoituksena oli tukea menetelmän kehittämistä ja jonka avulla pyrittiin luomaan lukijalle mahdollisimman vahva teoreettinen pohja tutkimuskysymyksiin vastaavan ratkaisumenetelmän ymmärtämiseksi.
Asiakasyrityksen ohjelmistotuotetta ja ohjelmistotuotantoprosessi huomioiden toteutettiin myös kokonaisratkaisu jossa toteutettiin testiautomaatio hyödyntäen kehitettyä priorisoinnin ratkaisumenetelmää.

Lopuksi vielä evaluoitiin menetelmän eli ratkaisun ja sitä hyödyntävän toteutuksen toimivuus käytännössä ja esitetään yhteenveto tutkimuksesta.

\section{Tutkimuksen rajaus} \label{06_tutkimuksen_rajaus}

Ohjelmistotestauksen tasojen osalta tutkimus rajoittuu hyväksymistestaukseen.
Tämä rajaus pohjautuu ohjelmistotuotannon työssä konkreettisesti havaittuun tarpeeseen sekä yhdenmukaisen testiautomaation toteuttamiseen mobiili- ja web-sovelluksille asiakasyrityksessä.

Testitapauksien osalta on olemassa lukuisia eri testausalustoja, joita hyödyntäen testitapauksia voidaan toteuttaa.
Tässä työssä testitapauksien toteuttaminen rajataan tietylle ennalta määräytyneelle Robot Framework alustalle.
Tämä rajaus pohjautuu asiakasyrityksessä jo aiemmin valittuihin testauksen sovelluskehyksiin ja työkaluihin.
Lisäksi Robot Framework on alustana yleisesti käytetty etenkin hyväksymistestauksen toteuttamiseen.
Robot Framework on myös erityisen hyvin soveltuva tilanteissa, joissa halutaan koodia korkeampaa abstraktiotasoa.

Jatkuvan integroinnin osalta tutkimus rajoittuu perusteisiin ja tutkimuksen painoarvo pidetään testitapauksien toteuttamisessa ja niiden priorisoinnissa.
Jatkuva integrointi on kuitenkin asiakasyrityksessä tärkeä osa testiautomaation ja jatkuvan käyttöönoton toteutuksessa.
Jatkuvan integroinnin osalta ei tässä työssä esitetä muuta kuin testiautomaation toteutusosaan kokonaisuutena erityisesti liittyvät käsitteet ja ratkaisu.
Tämä rajaus pohjautuu tutkimusongelman tarkempaan spesifioimiseen ja tutkimuksen kokonaislaajuuden hallitsemiseen.

Verkkoteorian osalta tutkimus rajoittuu perusteisiin ja painotettua verkkoa sekä kehitettyä menetelmää tukeviin käsitteisiin.
Tämä rajaus pohjautuu työn kohdentamiseen ohjelmistotuotantoon ja diplomityön kirjoitusvaiheessa saatuun ohjauspalautteeseen, jossa matematiikan osuus oli kasvanut liiallisen suureksi.

Priorisointiin vaikuttavien muuttujien osalta tutkimus rajoittuu muuttujien kartoittamiseen, mutta niiden määrittäminen jätetään työn ulkopuolelle.
Tämä tarkoittaa käytännössä sitä, että jokainen menetelmää hyödyntävä taho hankkii itse varsinaiset numeeriset arvot muuttujille.
Esimerkiksi liiketoiminnallisen vision arvo on yksinomaan menetelmää käyttävän tahon päätettävissä.

Painotetun verkon lyhimmän polun etsimiseen on olemassa lukuisa määrä erilaisia algoritmeja, mutta tässä työssä hyödynnetään vain perinteistä Djikstran algoritmia.
Tämä rajaus pohjautuu työssä kehitetyn priorisointimenetelmän käyttämisen perimmäiseen tarkoitukseen, jossa ei algoritmin tehokkuudella tai lisäominaisuuksilla ole suurta merkitystä.
Lisäksi Dijkstran algoritmi on selkeä, paljon tutkittu ja käytetty ratkaisu lyhimmän polun etsimiseen, joka tekee siitä ihanteellisen tässä työssä muodostettavaan painotettuun verkkoon tehtävien leikkauksien identifoimista ajatellen.

\section{Tavoitteet} \label{06_tavoitteet}

Tutkimuksen tavoitteena on löytää toistettavissa oleva menetelmä hyväksymistestauksessa tarvittavien testitapauksien priorisoimiseen.

Lisäksi tavoitteena on pystyä todentamaan kehitetyn menetelmän toimivuus käytännössä asiakasyritykselle tehtyä toteutusta evaluoimalla.
