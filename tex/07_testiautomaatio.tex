Tässä luvussa pyritään esittämään perusteet ja tarvittavat tiedot testiautomaatiosta, jotka liittyvät työn laajempaan teoreettiseen kehykseen.
Testiautomaation perusteiden ymmärtämistä tarvitaan työn myöhemmässä vaiheessa, jossa esitetään varsinainen testitapauksien priorisointi painotetun verkon avulla.

\section{Testiautomaation tarkoitus}

Testiautomaation tarkoitus on pohjimmiltaan mahdollistaa ohjelmistotuotteen jatkuva ja vaivaton laadunvarmistus, nyt ja tulevaisuudessa.
Ohjelmistojen testaamisella itsessään pyritään löytämään ohjelmistotuotteesta virheitä, anomalioita ja varmistamaan että se toimii asetettujen vaatimusten mukaisesti.
Testauksen automatisoiminen vapauttaa henkilöresursseja manuaalisesta testaamisesta muihin tuotantotehtäviin sekä parantaa toistuvien testien luotettavuutta poistamalla manuaalisessa testauksessa tapahtuvat inhimillisen virheet.
Laadunvarmistuksen osalta testiautomaatiolla voidaan kattaa erilaisia ohjelmistotuotteen laadullisia ominaisuuksia, kuten toiminnallisuus, luotettavuus, käytettävyys, tehokkuus, ylläpidettävyys ja siirrettävyys [ISO 9126].

\section{Testauksen menetelmät}

<Poista tai lisää teksti tähän>

\section{Testauksen tasot}

Testauksen tasoja on lukuisia ja usein ohjelmistojen testaamiseksi on suositeltavaa käyttää eri tasojen yhdistelmää.
Ohjelmistotestaus usein jaotellaan kolmeen erilaiseen menetelmään, jotka myös vaikuttavat eri testauksen tasojen käytettävyyteen.
Erilaisia menetelmiä ovat mustalaatikkotestaus, harmaalaatikkotestaus ja valkolaatikkotestaus, jotka eroavat toisistaan yleisesti ottaen siinä, otetaanko tieto ohjelmistotuotteen sisäisestä toteutuksesta mukaan testaamiseen.
Testauksen tasoista esimerkiksi yksikkötestausta ei ole mahdollista toteuttaa valkolaatikkotestauksen viitekehyksessä, sillä sitä varten tarvitaan täydellinen tieto ohjelmiston sisäisten komponenttien toteutuksesta.

  \subsection{Yksikkötestaus}

  <Lisää teksti tähän>

  \subsection{Integraatiotestaus}

  <Lisää teksti tähän>

  \subsection{Järjestelmätestaus}

  <Lisää teksti tähän>

  \subsection{Hyväksymistestaus}

  <Lisää teksti tähän>

\section{Testiautomaatio prosessina}

Testiautomaation prosessiin kuuluu erilaisia artifakteja, joita luodaan testausprosessin eri vaiheissa.
Eri vaiheita ovat kronologisessa järjestyksessä ovat testisuunnitelma, skenaariot, testitapaukset ja seuranta.

\section{Testitapauksien määrittäminen}

<Lisää teksti tähän>

<Yleiset heuristiikat>

\section{Testitapauksien priorisointi}

Testitapauksien priorisointi on kustannussyistä tai resurssien optimoinnin kannalta erittäin tärkeää.
Ohjelmistotestauksessa on hyvä tiedostaa, että ohjelmistotuotetta ei usein voida testata täydellisesti, joka nostaa esiin tarpeen tärkeimpien testitapauksien löytämisestä.
Testitapauksia voidaan priorisoida monella tavalla, joihin tämä diplomityö tuo yhden uuden painottua verkkoa hyödyntävän lähestymistavan.

<Painotetun verkon hyödyntäminen>

<Muut priorisointitavat>

\section{Web-käyttöliittymien erityispiirteet}

<Lisää teksti tähän>

\section{Hyväksymistestausvetoinen kehitys}

<Lisää teksti tähän>
