Tässä luvussa esitetään perusteet ja tarvittavat tiedot testitapauksista..

\section{Mikä on testitapaus?}

% Teksti tähän

\section{Testitapauksien määrittäminen}

Yleisiä testitapauksien määrittämiseen käytettäviä heuristiikkoja ovat muun muassa:
\begin{itemize}
  \item Polut ja tiedostot
  \item Aika ja päivämäärät
  \item Numerot
  \item Merkkijonot
  \item Yleiset rikkeet
  \item Muuttujien analyysi
  \item Kosketuspisteet
  \item Rajat
  \item CRUD-toiminnot
  \item Datan eheys
  \item Konfiguraatiot
  \item Katkokset
  \item Nälkiintyminen
  \item Samanaikaiset käyttäjät
  \item Transaktiotulvat
  \item Riippuvuudet
  \item Rajaehdot
  \item Syötetyypit
  \item Tilan analyysi
  \item Käyttäjät ja skenaariot
\end{itemize}

\section{Web-käyttöliittymien erityispiirteet}

Web-käyttöliittymillä on myös omia erityispiirteitä, jotka vaikuttavat testitapauksien laatimiseen.
\begin{itemize}
  \item Navigointi
  \item Syötteet
  \item Syntaksi
  \item Selainasetukset
\end{itemize}

\section{Priorisointiongelma}

Testitapauksien priorisointi on kustannussyistä tai resurssien optimoinnin kannalta erittäin tärkeää.
Ohjelmistotestauksessa on hyvä tiedostaa, että ohjelmistotuotetta ei usein voida testata täydellisesti, joka nostaa esiin tarpeen tärkeimpien testitapauksien löytämisestä.
Testitapauksia voidaan priorisoida monella tavalla, joihin tämä diplomityö tuo yhden uudenlaisen painottua verkkoa hyödyntävän lähestymistavan.
\begin{itemize}
  \item Painotetun verkon hyödyntäminen
  \item Muut priorisointitavat
\end{itemize}
