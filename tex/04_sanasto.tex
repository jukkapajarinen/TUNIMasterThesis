\newglossaryentry{api}{
  name={API},
  description={Ohjelmointirajapinta, eli englanniksi Application Programming Interface}
}

\newglossaryentry{ci}{
  name={CI},
  description={Jatkuva integrointi, eli englanniksi Continuous Integration}
}

\newglossaryentry{csharp}{
  name={C\#},
  description={Microsoftin kehittämä oliopohjainen ohjelmointikieli}
}

\newglossaryentry{dom}{
  name={DOM},
  description={Verkkosivujen rakenteen kuvaava dokumenttiobjektimalli, eli englanniksi Document Object Model}
}

\newglossaryentry{end2end}{
  name={e2e},
  description={Päästä päähän -testaus, eli englanniksi End-to-end testing}
}

\newglossaryentry{frontend}{
  name={Front-end},
  description={Web-sovelluksien asiakas-palvelin arkkitehtuurissa asiakaspuolella toimiva ohjelma}
}

\newglossaryentry{gocd}{
  name={GoCD},
  description={ThoughtWorks yhtiön kehittämä jatkuvan integroinnin ja julkaisemisen työkalu}
}

\newglossaryentry{html}{
  name={HTML},
  description={Verkkosivuihin käytetty hypertekstin merkintäkieli, eli englanniksi Hypertext Markup Language}
}

\newglossaryentry{ide}{
  name={IDE},
  description={Integroitu ohjelmointiympäristö, eli englanniksi Integrated Development Environment}
}

\newglossaryentry{iso}{
  name={ISO},
  description={Kansainvälinen standardisoimisjärjestö, eli englanniksi International Organization for Standardization}
}

\newglossaryentry{json}{
  name={JSON},
  description={JavaScript ohjelmointikielestä syntynyt tiedostoformaatti, eli englanniksi JavaScript Object Notation}
}

\newglossaryentry{moscow}{
  name={MoSCoW},
  description={Priorisointimenetelmä, joka tulee englannin kielen sanoista: Must, Should, Could ja Would}
}

\newglossaryentry{sql}{
  name={SQL},
  description={Relaatiotietokannan hallitsemiseen tarkoitettu kyselykieli, eli englanniksi Structured Query Language}
}

\newglossaryentry{unix}{
  name={UNIX},
  description={Suosittu tietokoneiden käyttöjärjestelmäperhe, englanniksi Uniplexed Information and Computing Service}
}

\newglossaryentry{atdd}{
  name={ATDD},
  description={Hyväksymistestausvetoinen kehitys, eli englanniksi Acceptance Test Driven Development}
}

\newglossaryentry{xss}{
  name={XSS},
  description={Eräs verkkosivuihin kohdistunut haavoittuvuus, eli englanniksi Cross Site Scripting}
}

\newglossaryentry{xvfb}{
  name={Xvfb},
  description={X-ikkunointijärjestelmän protokollan toteuttava virtualisointipalvelin, eli englanniksi X Virtual Framebuffer}
}

\newglossaryentry{yaml}{
  name={YAML},
  description={Erityisesti konfiguraatiotiedostoissa käytetty merkintäkieli, eli englanniksi YAML Ain’t Markup Language}
}
