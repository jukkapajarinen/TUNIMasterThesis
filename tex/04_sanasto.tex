\newglossaryentry{natural_numbers}{
  name={\ensuremath{\mathbb{N}}},
  description={Luonnolisten lukujen joukko}
}

\newglossaryentry{end2end}{
  name={e2e},
  description={End-to-end, eli päästä päähän testaus}
}

\newglossaryentry{atdd}{
  name={ATDD},
  description={Hyväksymistestausvetoinen kehitys, englanniksi: acceptance test driven development}
}

\newglossaryentry{ab}{
  name={A/B},
  description={A/B-testaus, kahta eri ohjelmarevisiota vertaileva testaus}
}

\newglossaryentry{ci}{
  name={CI},
  description={Continuous Integration, eli jatkuva integrointi}
}

\newglossaryentry{uat}{
  name={UAT},
  description={User Acceptance Testing, eli käyttäjän hyväksyttämistestaus}
}

\newglossaryentry{html}{
  name={HTML},
  description={Hypertext markup language, eli internetin verkkosivuihin käytetty hypertekstin merkintäkieli}
}

\newglossaryentry{dom}{
  name={DOM},
  description={Document object model, eli html-sivujen dokumenttiobjektimalli}
}

\newglossaryentry{xss}{
  name={XSS},
  description={Cross site scripting, eli verkkosivuihin kohdistettava haavoittuvuus}
}

\newglossaryentry{sql}{
  name={SQL},
  description={Structured query langauge, eli tietokantoihin kohdistettava haavoittuvuus}
}

\newglossaryentry{unix}{
  name={UNIX},
  description={Uniplexed Information and Computing Service, yksi suosittu ja vapaa tietokoneen käyttöjärjestelmäperhe}
}

\newglossaryentry{xvfb}{
  name={Xvfb},
  description={X virtual framebuffer, X-ikkunointijärjestelmän protokollan toteuttava virtualisointipalvelin}
}