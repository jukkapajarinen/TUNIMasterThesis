\newglossaryentry{api}{
  name={API},
  description={Application Programming Interface, ohjelmointirajapinta}
}

\newglossaryentry{ci}{
  name={CI},
  description={Continuous Integration, jatkuva integrointi}
}

\newglossaryentry{csharp}{
  name={C\#},
  description={Microsoftin kehittämä oliopohjainen ohjelmointikieli}
}

\newglossaryentry{dom}{
  name={DOM},
  description={Document Object Model, dokumenttiobjektimalli}
}

\newglossaryentry{end2end}{
  name={e2e},
  description={End-to-end, akronyymi päästä päähän -testaukselle}
}

\newglossaryentry{frontend}{
  name={Front-end},
  description={Asiakaspuoli asiakas-palvelin arkkitehtuurissa}
}

\newglossaryentry{gocd}{
  name={GoCD},
  description={Go Continuous Delivery, jatkuvan integroinnin ja julkaisemisen työkalu}
}

\newglossaryentry{gui}{
  name={GUI},
  description={Graphical User Interface, graafinen käyttöliittymä}
}

\newglossaryentry{html}{
  name={HTML},
  description={HyperText Markup Language, hypertekstin merkintäkieli}
}

\newglossaryentry{ide}{
  name={IDE},
  description={Integrated Development Environment, integroitu ohjelmointiympäristö}
}

\newglossaryentry{iso}{
  name={ISO},
  description={International Organization for Standardization, kansainvälinen standardoimisjärjestö}
}

\newglossaryentry{json}{
  name={JSON},
  description={JavaScript Object Notation, JavaScript-kieleen pohjautuva tiedostoformaatti}
}

\newglossaryentry{moscow}{
  name={MoSCoW},
  description={Must, Should, Could ja Would priorisointimenetelmä}
}

\newglossaryentry{sql}{
  name={SQL},
  description={Structured Query Language, kyselykieli relaatiotietokannan hallitsemiseen}
}

\newglossaryentry{unix}{
  name={UNIX},
  description={Yleinen tietokoneiden käyttöjärjestelmäperhe}
}

\newglossaryentry{atdd}{
  name={ATDD},
  description={Acceptance Test-driven Development, hyväksymistestausvetoinen kehitys}
}

\newglossaryentry{xss}{
  name={XSS},
  description={Cross Site Scripting, eräs verkkosivuihin kohdistuva haavoittuvuus}
}

\newglossaryentry{xvfb}{
  name={Xvfb},
  description={X Virtual Framebuffer, X-ikkunointijärjestelmän protokollan toteuttava virtualisointipalvelin}
}

\newglossaryentry{yaml}{
  name={YAML},
  description={Yleinen konfiguraatiotiedostoissa käytetty merkintäkieli}
}
