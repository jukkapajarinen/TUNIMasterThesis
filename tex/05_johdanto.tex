Nykypäivänä web-sovellukset ovat kasvaneet laajuudessa, toiminnallisuudessa ja komp\-lek\-si\-suu\-des\-sa todella suuriksi \parencite{web_app_complexity}.
Web-sovelluksissa on useita eri näkymiä, niiden välisiä siirtymiä sekä niiden sisältämää loppukäyttäjille tarkoitettua toiminnallisuutta.
Web-sovelluksien ja niiden käyttöliittymien sekä toiminnallisuuden laajuuden kasvaessa on erittäin tärkeää, että niiden testaamiseen otetaan mukaan myös hyväksymistestausta joka varmistaa loppukäyttäjille suunnatun toiminnallisuuden toimivuuden.
Hyväksymistestaus voidaan nykypäivänä automatisoida testitapauksien muodossa.
Hyväksymistestauksen testitapauksien kattavuus on yksi haaste, sillä läheskään kaikkia sovelluksen toimintoja ei usein ole järkevää tai edes mahdollista testata \parencite{testing_possibility}.
Tämän lisäksi testitapauksien priorisointi on muun muassa kustannussyistä ja resurssien optimoinnin kannalta äärimmäisen tärkeää.

Tämä diplomityö on yhtenäinen web-käyttöliittymien hyväksymistestauksen automatisoimiseen ja siihen liittyvien testitapauksien priorisoimiseen liittyvä kokonaisuus.
Työstä voidaan erotella kaksi eri osuutta, jotka ovat testiautomaation mahdollistava hyväksymistestausjärjestelmä ja hyväksymistestauksen testitapauksien priorisointimenetelmä.
Hyväksymistestausjärjestelmän tarkoituksena oli luoda järjestelmä web-käyttöliittymien hyväksymistestauksen automatisoimiseen testitapauksien avulla.
Priorisointimenetelmän tarkoituksena puolestaan oli edellä mainittuun järjestelmään rakennettavien testitapauksien priorisoiminen, jotta epäoleelliset testitapaukset voitaisiin jättää toteuttamatta ja keskittyä vain prioriteetiltaan tärkeiden testitapauksien rakentamiseen.

Luvussa kaksi esitetään tutkimusasetelma, joka sisältää tutkimuksen tausta, tutkimuskysymykset joihin diplomityössä etsitään vastausta ja tutkimuksen rajauksen sekä sen tavoitteet.
Luvussa kolme käydään läpi testiautomaation teoriaa, joka on tarpeellista esitietoa työn toteutuksen ymmärtämistä ajatellen.
Testiautomaation teoriasta oleellisena osana käydään läpi ohjelmistotestauksen tasot, jotta diplomityössä hyväksymistestaukseen kohdistunut painopiste tulee hyvin esille.
Lisäksi testiautomaatioon liittyen käydään läpi myös testitapauksien, testikokoelmien ja jatkuvan integroinnin käsitteet, jotka liittyvät oleellisesti diplomityössä toteutettuun hyväksymistestausjärjestelmään.
Luvussa neljä käydään läpi hyväksymistestausta sekä esitetään suunniteltu ja työn tuloksena syntynyt hyväksymistestausjärjestelmä.
Lisäksi käydään läpi testausjärjestelmään rakennettavia testitapauksia sekä alustetaan niiden priorisointiongelmaa.
Luvussa viisi esitetään painotettuun verkkoon perustuvan priorisointimenetelmän toteutus.
Ensin käydään läpi priorisointimenetelmään oleellisesti liittyvää matemaattisten verkkojen teoriaa.
Teorian jälkeen esitetään priorisointiin vaikuttavat muuttujat, painofunktiot priorisointiin, verkon rakentaminen ja verkkoon tehtävä karsinta leikkauksien avulla.
Lisäksi käydään läpi Dijkstran algoritmin hyödyntämistä painotetussa verkossa ja verkon sekä testitapauksien yhteys.
Luvussa kuusi tarkastellaan ja arvioidaan tutkimuksen konkreettiset tulokset, joita ovat hyväksymistestausjärjestelmä sekä kehitetty priorisointimenetelmä.
Hyväksymistestausjärjestelmä sekä priorisointimenetelmä evaluoidaan erillisissä kappaleissa, jonka jälkeen esitetään myös työn toteutuksen jälkeen syntyneitä jatkokehitysehdotuksia.
Luvussa seitsemän esitetään vielä tutkimuksen yhteenveto ja pohditaan tutkimuksen tavoitteiden saavuttamista ja tutkimuskysymyksiin vastaamisen onnistumista.
