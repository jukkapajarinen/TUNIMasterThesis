Web-sovellukset ja niiden sisältämät käyttöliittymät kasvavat jatkuvasti yhä suuremmiksi ja kompleksisemmiksi.
Tämän lisäksi ohjelmistojen testaaminen ja testiautomaation rakentaminen ovat erittäin tärkeitä ja kasvavia tehtäviä nykyaikaisessa ohjelmistokehityksessä.
Testiautomaatiossa tarvittavien testitapauksien priorisointi on kustannussyistä tai resurssien optimoinnin kannalta myös erittäin tärkeää.
Hyväksymistestausta voidaan priorisoida ja rakentaa loppukäyttäjän rooliin asettuen, jolloin päästään keskittymään oikeasti olennaisten toimintojen ja ominaisuuksien testaamiseen, sekä säästämään testiautomaation rakentamiseen tarvittavia resursseja.

Tässä diplomityössä keskityttiin hyväksymistestauksen automatisointiin ja siihen liittyvän testitapauksien priorisointiongelman ratkaisemiseen.
Diplomityö koostuu kahdesta osuudesta, jotka ovat hyväksymistestauksen automatisoimisen mahdollistava hyväksymistestausjärjestelmä, ja hyväksymistestauksen testitapauksien matemaattinen priorisointimenetelmä.

Tutkimusta varten laadittiin neljä tutkimuskysymystä, joihin vastaaminen asetettiin myös yhdeksi työn tavoitteista.
Tutkimuskysymykset liittyivät painotetun verkon hyödyntämiseen testiautomaation priorisoimisessa, priorisointiin vaikuttavien muuttujien kartoittamiseen, hyväksymistestauksen testitapauksien ja painotetun verkon väliseen yhteyteen, sekä painotetun verkon avulla tehtävän priorisoinnin ja testiautomaation yhteenliittämiseen.

Työn tuloksena syntyivät testiautomaation mahdollistava hy\-väk\-sy\-mis\-tes\-taus\-jär\-jes\-tel\-mä web-käyt\-tö\-liit\-ty\-mien toiminnallisuuden testaamiseen, sekä toistettavissa oleva matemaattinen menetelmä hyväksymistestauksen testitapauksien priorisoimiseen.
Työn tuloksena syntynyt priorisointimenetelmä on työn tärkeintä antia, josta on toivottavasti mahdollisimman suurta hyötyä priorisointiongelman ratkaisemista kaipaaville tahoille.
